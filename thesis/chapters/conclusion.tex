\chapter{\sc Conclusion}\label{ch:conclusion}

In this thesis we have investigated fitted front tracking finite element
methods for two-phase incompressible Navier--Stokes flows. In Chapter
\ref{ch:introduction}, after introducing the free-boundary fluid flows, we have
described the Navier--Stokes equations which model the evolution of a two-phase
fluid. In these types of problems, apart from the flow solution in the bulk
domain, the interface dividing the phases needs to be determined. Therefore, we
have briefly overviewed the different ways to treat the unknown interface and we
have detailed the various numerical challenges posed by these types of flows.
In Chapter \ref{ch:geometric_pdes} we have introduced the key ideas of the
front tracking approach, applying it to the simple problem of purely geometric
evolution equations such as mean curvature flow and surface diffusion. Here we
have reported on the finite element approximations and we have performed some
numerical experiments in order to show the property of the schemes and of the
flows. In particular, we have shown that the mesh naturally equidistributes and
that the volume is conserved. In Chapter \ref{ch:stokes} we have proposed a
novel finite element approximation for incompressible two-phase Stokes flow.
After showing an energy bound and the volume conservation for the weak problem,
we have demonstrated that our scheme is unconditionally stable and we have
proved the existence and uniqueness of the discrete approximation. Moreover, we
have investigated the equidistribution property and the volume conservation. We
have also discussed the mesh generation process together with the smoothing and
remeshing procedures used to preserve the mesh quality. In Chapter
\ref{ch:stokes_results} we have presented several numerical experiments in 2d
and 3d to test our method and to allow comparisons with other methods. Using
two exact solutions to the two-phase Stokes flow, we have performed various
convergence tests. Moreover, we have tested that interface mesh points are
naturally equidistributed by our scheme, that our scheme conserves the enclosed
volume and that the surface energy monotonically decays. We have also reported
on a shear flow experiment. In Chapter \ref{ch:navier_stokes} we have extend
our Stokes scheme to solve the incompressible two-phase Navier--Stokes flow.
Here, we have derived a standard finite element approximation and an alternative
finite element approximation based on an antisymmetric rewrite. For the
antisymmetric discretization, we have proved the existence and uniqueness of a
discrete solution and we have demonstrated that this scheme has certain
stability properties. We have also examined the techniques used to interpolate
the velocity. In Chapter \ref{ch:ale} we have presented a different finite
element approximation for incompressible two-phase Navier--Stokes flow, which
uses the Arbitrary Lagrangian Eulerian method. This technique allows to
rewrite the velocity time derivative with respect to a fixed reference
manifold. As a consequence, the velocity does not need to be interpolated
anymore. Finally, in Chapter \ref{ch:ns_results} we have tested our schemes in
several numerical experiments. Using two exact solutions to the two-phase
Navier--Stokes flow, we have performed various convergence tests. Then, we have
reported on two rising bubble experiments which constitute the standard
benchmark for two-phase schemes.
\footnote{Marco: You should conclude something here. Is ALE superior? Does the
standard scheme perform as well? Were the downsides of the standard scheme
expressed at the beginning of the ALE chapter observed in practice? What scheme
would you advertise to be used?}
