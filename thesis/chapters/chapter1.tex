\chapter{\sc Geometric Evolution Equations}
\label{ch:1}

\section[Geometric PDEs]{Geometric PDEs}
As starting point, in order to describe the key ideas of the front tracking
approach, we consider the simple problem of purely geometric evolution
equations. A geometric evolution equation defines the motion of a hypersurface
by prescribing its normal velocity in terms of its geometric quantities. These
problems are part of the more general time-dependent interface evolution
problems category, where the normal velocity depends also on field quantities
evaluated on the analysed hypersurface. A detailed description and analysis can
be found in the review article \cite{DeckelnickDE05}.

Interface evolution problems are everywhere in modern physics and engineering.
Several typical applications can be found in materials science such as the
mathematical modelling of the morphology of microstructure in order to
correctly evaluate the mechanical properties of the material or in void
electro-stress migration where small voids or cracks contained in metallic
materials can change their location and shape according to the presence of
surface diffusion and electro-stress loading. Other typical applications which
can be modelled as time-dependent interface evolution problems are the motion of
grain boundaries which separate differing orientations of the same crystalline
phase, or solid-liquid interfaces exhibiting dendritic structures in
under-cooled solidification. Another research fields where these models can be
applied is image processing to detect the separation of dark regions from a
brighter background and to identify separating contours in order to correctly
cluster the objects in the image. Instead, as explained in Chapter
\ref{ch:introduction}, we apply these techniques to incompressible two-phase
flows problem.

A general geometric evolution equation has the following formulation
\begin{equation}\label{eq:geometric_pde}
\vec{\mathcal{V}}\,.\,\vec\nu=f(\vec{z}\,,\vec{\nu}\,,\varkappa)
\qquad\mbox{on }\Gamma(t),
\end{equation}
which prescribes the normal velocity $\vec{\mathcal{V}}\,.\,\vec\nu$ of the
interface $\Gamma(t)$ as a function depending on the position $\vec z$, its
normal direction $\vec{\nu}$ and the sum of its principal curvatures
$\varkappa$.

The simplest geometric PDE is the one which arises from motion by mean
curvature,
\begin{equation}\label{eq:mean_curvature}
\vec{\mathcal{V}}\,.\,\vec\nu=\varkappa\qquad\mbox{on }\Gamma(t).
\end{equation}
This equation describes a surface evolving in such a way that its own normal
velocity is equal to the sum of the $d-1$ principal curvatures of $\Gamma(t)$.

Another important geometric PDE is the one which arises from motion by
surface diffusion
\begin{equation}\label{eq:surface_diff}
\vec{\mathcal{V}}\,.\,\vec\nu=-\Delta_s \varkappa \qquad\mbox{on }\Gamma(t)\,,
\end{equation}
where we use the Laplace--Beltrami operator $\Delta_s$ defined by
(\ref{eq:surface_laplacian}). In this case the surface normal velocity matches
the surface Laplacian of the mean curvature.

For both models, (\ref{eq:mean_curvature}) and (\ref{eq:surface_diff}), it is
necessary to prescribe the initial interface $\Gamma(0)=\Gamma_0$ in order to
have a well posed problem.

\section[Geometric Analysis]{Geometric analysis}
The aim of this section is to collect some useful definitions and results from
differential geometry. Again we refer to \cite{DeckelnickDE05} which cover the
subject in depth.

A subset $\Gamma \subset \R^d$ is called a $C^2$-hypersurface if for each point
$\vec z_0 \in \Gamma$ there exists an open set $G \subset \R^d$ containing
$\vec z_0$ and a function $g \in C^2(G)$ such that
\begin{equation}
G \cap \Gamma = \{ \vec z \in G \, | \, g(\vec z) = 0 \}
\qquad \mbox{ and } \qquad \nabla \, G(\vec z) \neq 0\,,
\quad \forall\ \vec z \in G \cap \Gamma \, .
\end{equation}
The tangent space $T_{\vec z} \Gamma$ is then the $(d-1)$-dimensional linear
subspace of $\R^d$ that is orthogonal to $\nabla \, g(\vec z)$. It does not
depend on the particular function $g$ which is chosen to describe $\Gamma$. A
$C^2$-hypersurface $\Gamma \in \R^d$ is called orientable if there exists a
vector-valued function $\vec\nu \in C^1(\Gamma, \R^d)$, i.e.~$\vec\nu \in C^1$
in an open neighbourhood of $\Gamma$, such that $\vec\nu(\vec z) \perp T_{\vec
z} \Gamma$ and $|\vec{\nu}(\vec z)| = 1$ for all $\vec z \in \Gamma$. In what
follows, we shall assume that $\Gamma \subset \R^d$ is an orientable
$C^2$-hypersurface.

In order to prove some property of the surface gradient
(\ref{eq:surface_gradient}), we employ the notation
\begin{equation}
\nabla_s \, g(\vec z) = (\underline{D}_1 g(\vec z), \hdots, \underline{D}_{d}
g(\vec z))
\end{equation}
for the $d$ components of the surface gradient. Since $\nabla_s \, g(\vec z)$
is the orthogonal projection of $\nabla \, g(\vec z)$ onto $T_{\vec z} \Gamma$,
it depends only on the values of $g$ on $\Gamma$. Moreover, from
(\ref{eq:surface_gradient}) follows that
\begin{equation}\label{eq:surface_gradient_comp}
\nabla_s \, g(\vec z) \,.\, \vec{\nu}(\vec z)=0\,, \qquad \vec z \in \Gamma\,.
\end{equation}
Similarly, if $g$ is twice differentiable in an open neighbourhood of
$\Gamma$, the surface Laplacian (\ref{eq:surface_laplacian}) can be written as
\begin{equation}\label{eq:surface_laplacian_comp}
\Delta_s g(\vec z) = \nabla_s\, . \,\nabla_s \, g(\vec z) =
\sum_{i = 1}^d \underline{D}_i \underline{D}_i g(\vec z) \, ,
\qquad \vec z \in \Gamma \, ,
\end{equation}
while, if $\vec g$ is a differentiable vector field, the surface divergence
can be rewritten as
\begin{equation}
\nabla_s \,.\, \vec g(\vec z) = \sum_{j = 1}^{d} \underline{D}_j \vec g_j
(\vec z)\,, \qquad \vec z \in \Gamma\,.
\end{equation}

We assume $\vec{\nu} \in C^1$ in a neighbourhood of $\Gamma$ so that we
may introduce the matrix $\mat H(\vec z)$ with components
\begin{equation}\label{eq:matrixHjk}
[\mat H (\vec z)]_{jk} = - \underline{D}_j \nu_k (\vec z)\,, \qquad \vec z \in
\Gamma \, ,
\end{equation}
with $1\leq j,k\leq n$. It can be shown that $\mat H(\vec z)$ is symmetric.
Furthermore,
\begin{equation}
\sum_{k = 1}^{d} [\mat H(\vec z)]_{jk} \nu_k (\vec z) =
\sum_{k = 1}^{d} - \underline{D}_j \nu_k (\vec z) \nu_k (\vec z) =
- \tfrac{1}{2} \underline{D}_j |\vec{\nu}|^2 (\vec z) = 0 \, ,
\end{equation}
since $|\vec{\nu}| = 1$ on $\Gamma$. Thus, $\mat H(\vec z)$ has one
eigenvalue which is equal to zero with corresponding eigenvector
$\vec\nu(\vec z)$. The remaining $d - 1$ eigenvalues $\varkappa_1 (\vec z),
\hdots, \varkappa_{d - 1} (\vec z)$ are called the principal curvatures of
$\Gamma$ at the point $\vec z$. The mean curvature of $\Gamma$ at $\vec z$
can then be defined as the trace of the matrix $\mat H(\vec z)$, that is
\begin{equation}\label{eq:definitionMC}
\varkappa (\vec z) = \sum_{j = 1}^{d} [\mat H(\vec z)]_{jj} = \sum_{j = 1}^{d -
1} \varkappa_j (\vec z) \,.
\end{equation}
Note that (\ref{eq:definitionMC}) differs from the more common
definition of mean curvature,
$\varkappa = \frac{1}{d - 1} \sum_{j = 1}^{d - 1} \varkappa_j$.
From (\ref{eq:matrixHjk}) we derive the following expression for mean
curvature:
\begin{equation}\label{eq:definition2MC}
\varkappa (\vec z)=-\nabla_s \,.\, \vec{\nu}(\vec z) \qquad \vec z \in \Gamma\,.
\end{equation}
In particular, if $\Gamma$ is the unite sphere, $\Gamma = \mathcal{S}^{d - 1}$,
and the unit normal field is chosen to point away from $\Gamma$,
i.e.~$\vec\nu(\vec z) = \vec z$, we obtain that $\varkappa = -
(d - 1)$, on considering the particular function $g(\vec z) = z_j$, $j \in \{
1, \hdots, d \}$ and observing that $\underline{D}_i z_j = \delta_{ij} - \nu_j
\nu_i$. This shows that the mean curvature $\varkappa$ is positive if $\Gamma$
is curved in the direction of the normal.

Moreover, while the sign of $\varkappa$ depends on the choice of the normal
$\vec\nu$, the mean curvature vector $\varkappa \vec\nu$ is an invariant. By
choosing again the particular function $g(\vec z) = z_j$, $j \in \{ 1,
\hdots, d \}$ in (\ref{eq:surface_laplacian_comp}) and recalling the application
of the Laplace-Beltrami operator to each independent variable $z_j$, we deduce
that
\begin{equation}
\Delta_s z_j = - \sum_{i = 1}^{d} \underline{D}_i (\nu_j \nu_i) =
- (\nabla_s \,.\, \vec\nu) \nu_j - \nabla_s \, \nu_j \,.\, \vec\nu = \varkappa
\nu_j\, ,
\end{equation}
which leads to the identity
\begin{equation} \label{eq:LBop}
\Delta_s\, \vec \id = \varkappa\, \vec\nu \qquad \mbox{on $\Gamma(t)$}\,.
\end{equation}
This identity of differential geometry will be useful at a later stage to
obtain a weak formulation of the surface diffusion and mean curvature problems.

\section[Front tracking approach]{Front tracking approach}
We always treat the interface using a front tracking approach which involves
seeking a parametrization of the unknown interface over a reference manifold.
More formally, we assume that $(\Gamma(t))_{t\in [0,T]}$ is a sufficiently
smooth evolving hypersurface without boundary which is parametrized by
$\vec x(\cdot,t):\Upsilon\to\R^d$, where $\Upsilon\subset \R^d$ is a given
reference manifold of the same topological type of the evolving hypersurface
$\Gamma(t)$, therefore
\begin{equation}\label{eq:interface_parametrization}
\Gamma(t) = \vec x(\Upsilon,t)\,.
\end{equation}
The position vector $\vec x(\cdot,t)$, for every time $t$, maps a certain point
$\vec{q}$ of the reference manifold $\Upsilon$ to its actual position
$\vec{z}$ on $\Gamma(t)$. Therefore, from (\ref{eq:interface_parametrization}),
we can define the velocity $\vec{\mathcal{V}}$ of $\Gamma(t)$ as
\begin{equation} \label{eq:V}
\vec{\mathcal{V}}(\vec z, t) := \vec x_t(\vec q, t) \qquad
\forall\ \vec z = \vec x(\vec q,t) \in \Gamma(t)\,.
\end{equation}

The position vector $\vec x(\cdot,t)$ is one unknown of the problem and, once
computed, the evolution of $\Gamma(t)$ is fully determined. Moreover, all the
geometrical quantities of the hypersurface, e.g. curvature, can be expressed as
derivatives of the parametrization.

It is worth to notice that, at discrete level, the reference manifold
$\Upsilon$ is never used. Indeed, instead of computing the position vector
$\vec x(\cdot,t)$, the unknown variable is the displacement that the previous
discrete interface is subject to. This can be viewed as setting, at every time
step, a new reference manifold which correspond to the actual hypersurface
configuration.

\section[Finite element discretization]{Finite element discretization}
The finite element discretization which we use is based on the seminal paper
\cite{Dziuk91} and we refer to the ones described in
\cite{triplej,triplejMC,gflows3d}. We use a different notation in order to be
consistent with \cite{spurious} and \cite{stokesfitted}.

Putting together the mean curvature flow (\ref{eq:mean_curvature}) with the
identity (\ref{eq:LBop}) we obtain the following system of PDEs
\begin{subequations}
\begin{align}
&\vec{x}_t\,.\,\vec{\nu}=\varkappa\,,\label{eq:mean_curvature_a}\\
&\varkappa\,\vec{\nu}=\Delta_s\,\vec{x}\,,\label{eq:mean_curvature_b}
\end{align}
\end{subequations}
where we have used (\ref{eq:V}) from the front tracking approach in order to
describe the hypersurface velocity $\vec{\mathcal{V}}$.

Analogously, the surface diffusion problem (\ref{eq:surface_diff}) can be
rewritten as a system of second order equations
\begin{subequations}
\begin{align}
&\vec{x}_t\,.\,\vec{\nu}=-\Delta_s\,\varkappa\,,\label{eq:surf_diff_a}\\
&\varkappa\,\vec{\nu}=\Delta_s\,\vec{x}\,.\label{eq:surf_diff_b}
\end{align}
\end{subequations}
In both problems (\ref{eq:mean_curvature_a}--b) and (\ref{eq:surf_diff_a}--b)
we impose the initial condition $\Gamma(0)=\Gamma_0$.

We consider the partitioning  $0= t_0 < t_1 < \ldots < t_{M-1} < t_M = T$ of
$[0,T]$ into possibly variable time steps
$\tau_m := t_{m+1}-t_m$, $m=0,\ldots, M-1$.

In order to define the parametric finite element spaces on $\Gamma^m$, we
assume that $\Gamma^m=\bigcup_{j=1}^{J_\Gamma} \overline{\sigma^m_j}$, where
$\{\sigma^m_j\}_{j=1}^{J_\Gamma}$ is a family of mutually disjoint open
$(d-1)$-simplices with vertices $\{\vec{q}^m_k\}_{k=1}^{K_\Gamma}$. We also
define the function space
\begin{equation}
\Vh := \{\vec\chi \in [C(\Gamma^m)]^d:\vec\chi\!\mid_{\sigma^m_j}
\in \mathcal{P}_1(\sigma^m_j), j=1,\ldots, J_\Gamma\} =: [\Wh]^d\,,
\end{equation}
where $\Wh \subset H^1(\Gamma^m)$ is the space of scalar continuous
piecewise linear functions on $\Gamma^m$, with $\{\chi^m_k\}_{k=1}^{K_\Gamma}$
denoting the standard basis of $\Wh$ and with $\mathcal{P}_k(\sigma^m)$
denoting the space of polynomials of degree $k$ on $\sigma^m$.

The new surface $\Gamma^{m+1}$ is parametrized respect $\Gamma^m$ using a
parametrization $\vec{X}^{m+1} \in \Vh$, so that $\Gamma^{m+1} =
\vec{X}^{m+1}(\Gamma^m)$.

Then the finite element approximation of the mean curvature flow, which is
based on the variational formulation (\ref{eq:mean_curvature_a}--b), is given
as follows. Let $\Gamma^0$ be an approximation to $\Gamma(0)$. For $m=0,\ldots,
M-1$, find $(\vec{X}^{m+1}, \kappa^{m+1}) \in \Vh \times \Wh$
such that
\begin{subequations}
\begin{align}
&\left\langle \frac{\vec{X}^{m + 1} - \vec{X}^{m}}{\tau_m},
\chi\vec{\nu}^m\right\rangle_{\Gamma^m}^h - \left\langle\kappa^{m+1}, \chi
\right\rangle_{\Gamma^m}^h=0\quad \forall\ \chi \in
\Wh\,,\label{eq:fem_mean_curvature_a}\\
&\left\langle\kappa^{m+1}\vec{\nu}^m, \vec{\eta}\right\rangle_{\Gamma^m}^{h} +
\left\langle\nabla_s\vec{X}^{m + 1},
\nabla_s\vec{\eta}\right\rangle_{\Gamma^m}=0\quad \forall\ \vec\eta \in \Vh,
\label{eq:fem_mean_curvature_b}
\end{align}
\end{subequations}
and set $\Gamma^{m+1} = \vec{X}^{m+1}(\Gamma^m)$. We observe that
(\ref{eq:fem_mean_curvature_a}--b) is a linear scheme in that it leads to a
linear system of equations for the unknowns $(\vec{X}^{m+1}, \kappa^{m+1})$ at
each time level. Here $\langle\cdot,\cdot\rangle_{\Gamma^m}$ is the $L^2$ inner
product over the current polyhedral surface
\begin{equation}\label{}
\langle u,v\rangle_{\Gamma^m} =
\int_{\Gamma^m}uv\,\mathrm{d}\mathcal{H}^{d-1}\,, \qquad\forall u,v \in
L^2(\Gamma^m),
\end{equation}
where $\mathcal{H}^{d-1}$ is the $(d-1)$-dimensional Hausdorff measure, and
similarly for $\vec{u},\vec{v}\in L^2(\Gamma^m,\mathbb{R}^d)$. In addition,
$\langle \cdot,\cdot\rangle_{\Gamma^m}^h$ is a mass lumped inner product with
the vertices of each triangle of the mesh as quadrature points. If $v,w \in
L^\infty(\Gamma^m)$ are piecewise continuous, with possible jumps
across the edges of $\{\sigma_j^m\}_{j=1}^{J_\Gamma}$, we define the mass
lumped inner product $\langle\cdot,\cdot\rangle_{\Gamma^m}^h$ as
\begin{equation} \label{eq:masslump}
\left\langle v, w \right\rangle^h_{\Gamma^m} :=
\tfrac1d \sum_{j=1}^{J_\Gamma} \mathcal{H}^{d-1}(\sigma^m_j)\,
\sum_{k=1}^{d} (v\,w)((\vec{q}^m_{j_k})^-),
\end{equation}
where $\{\vec{q}^m_{j_k}\}_{k=1}^{d}$ are the vertices of $\sigma^m_j$, and
where we define the limit $v((\vec{q}^m_{j_k})^-)
:= \underset{\sigma^m_j\ni \vec{p}\to \vec{q}^m_{j_k}}{\lim}\, v(\vec{p})$. We
naturally extend (\ref{eq:masslump}) to vector valued functions. We notice that
the second term of (\ref{eq:fem_mean_curvature_b}) is indicated as an exact
integration. This is indeed the case because it is the product of two
piecewise constant functions, since they are the gradient of piecewise linear
functions, therefore the mass lumped inner product (\ref{eq:masslump}) is
exact.

Analogously, the finite element approximation for the surface diffusion problem
(\ref{eq:surface_diff}--b) is analogue to the one obtained for the mean
curvature flow with the linear system (\ref{eq:fem_mean_curvature_a}--b)
changed as
\begin{subequations}
\begin{align}
&\left\langle \frac{\vec{X}^{m + 1} - \vec{X}^{m}}{\tau_m},
\chi\vec{\nu}^m\right\rangle_{\Gamma^m}^{h} - \left\langle\nabla_s\kappa^{m+1},
\nabla_s\chi\right\rangle_{\Gamma^m}=0\quad \forall\ \chi \in
\Wh\,,\label{eq:fem_surf_diff_a}\\
&\left\langle\kappa^{m+1}\vec{\nu}^m, \vec{\eta}\right\rangle_{\Gamma^m}^{h} +
\left\langle\nabla_s\vec{X}^{m + 1},
\nabla_s\vec{\eta}\right\rangle_{\Gamma^m}=0\quad \forall\ \vec\eta \in \Vh\,.
\label{eq:fem_surf_diff_b}
\end{align}
\end{subequations}

\section[Algebraic formulation]{Algebraic formulation}
As regards the mean curvature flow, (\ref{eq:fem_mean_curvature_a}--b), the
corresponding algebraic system of equations can be written as

\begin{equation}\label{eq:algebraic_mean_curvature}
\begin{pmatrix}
\tau_m \, M_m & - \, \vec{N}_{m}^{T} \\
\vec{N}_m & \vec{A}_m
\end{pmatrix}
\begin{pmatrix}
\kappa^{m + 1} \\
\delta \vec{X}^{m + 1}
\end{pmatrix}
=
\begin{pmatrix}
0 \\
- \vec{A}_m \, \vec{X}^{m}
\end{pmatrix} ,
\end{equation}
while, for the system of equations describing the surface diffusion,
(\ref{eq:fem_surf_diff_a}--b), the algebraic system is
\begin{equation}\label{eq:algebraic_surf_diff}
\begin{pmatrix}
\tau_m \, A_m & - \, \vec{N}_{m}^{T} \\
\vec{N}_m & \vec{A}_m
\end{pmatrix}
\begin{pmatrix}
\kappa^{m + 1} \\
\delta \vec{X}^{m + 1}
\end{pmatrix}
=
\begin{pmatrix}
0 \\
- \vec{A}_m \, \vec{X}^{m}
\end{pmatrix}.
\end{equation}
We can notice that the only difference between the above systems occurs in the
upper-left entry of the matrix.

The entries of the matrix are defined as
\begin{eqnarray}
\left[ M_m \right]_{kl} & := & \langle \chi_k^m \, , \chi_l^m
\rangle_{\Gamma^m}^h\,,\label{eq:algebraic_entries_a} \\
\left[ \vec{N}_m \right]_{kl} & := & \langle \chi_k^m \, \vec{\nu}^m\,,
\chi_l^m \rangle_{\Gamma^m}^h\,,\label{eq:algebraic_entries_b} \\
\left[ A_m \right]_{kl} & := & \langle \nabla_s \chi_k^m \, , \nabla_s
\chi_l^m \rangle_{\Gamma^m}\,,\label{eq:algebraic_entries_c}
\end{eqnarray}
where $\{\chi_k^m\}$ are the basis functions of the finite element space of
piecewise linear continuous functions $W(\Gamma^m)$ and
\begin{equation}\label{eq:algebraic_entries_d}
[\vec{A}_m]_{kl} := [A_m]_{kl} \mat \id\,.
\end{equation}
We notice that for entries (\ref{eq:algebraic_entries_c}) holds
\begin{equation}
\begin{split}
& \langle \nabla_s \chi_k^m \, , \nabla_s \chi_l^m \rangle_{\Gamma^m} = \langle
\nabla \chi_k^m \, -\,\nabla \chi_k^m\,.\,\vec \nu\,\vec\nu, \nabla
\chi_l^m \, -\,\nabla \chi_l^m\,.\,\vec \nu\,\vec\nu\rangle_{\Gamma^m} = \\
& \langle \nabla \chi_k^m \, , \nabla \chi_l^m \rangle_{\Gamma^m}\,
- 2\, \langle \nabla \chi_k^m \,.\,\vec\nu\,\vec\nu , \nabla \chi_l^m
\rangle_{\Gamma^m}
+\langle \nabla \chi_k^m \,.\,\vec\nu\,\vec\nu , \nabla \chi_l^m
\,.\,\vec\nu\,\vec\nu \rangle_{\Gamma^m} = \\
& \langle \nabla \chi_k^m \, , \nabla \chi_l^m \rangle_{\Gamma^m}
- \langle \nabla \chi_k^m \,.\,\vec\nu\,\vec\nu , \nabla \chi_l^m
\rangle_{\Gamma^m} =
\langle \nabla \chi_k^m \, , \nabla \chi_l^m \rangle_{\Gamma^m}\,,
\end{split}
\end{equation}
where, in the last equality, we have used the fact that $\langle \vec
\nu\,,\,\nabla \chi_k^m  \rangle_{\Gamma^m}$. Therefore, in practice,
(\ref{eq:algebraic_entries_c}) can be assembled as the standard Laplacian
term $\langle \nabla \chi_k^m \, , \nabla \chi_l^m \rangle_{\Gamma^m}$.

The linear systems (\ref{eq:algebraic_mean_curvature}) and
(\ref{eq:algebraic_surf_diff}) are invertible under a suitable assumption on the
triangulation at each time step, see \cite{gflows3d}. The algebraic system
(\ref{eq:algebraic_mean_curvature}) and (\ref{eq:algebraic_surf_diff}) are
very small also for fine meshes therefore they can be solved very efficiently
with a sparse factorization package such as \verb|UMFPACK|, see \cite{Davis04}.

\section[Numerical results]{Numerical results}

It can be shown that both (\ref{eq:mean_curvature}) and
(\ref{eq:surface_diff}) decrease the surface area of $\Gamma(t)$ over time.
Moreover (\ref{eq:surface_diff}) conserves the volume enclosed by $\Gamma(t)$,
which means that a sphere remains exactly the same sphere when it evolves
according to surface diffusion. See \cite{DeckelnickDE05} for more details on
geometric evolution equations and their numerical approximation.

Finally, we need to introduce a mass lumped inner product on $\Gamma^m$, that
is crucial for the desired tangential motion of vertices on $\Gamma^m$. This
induced tangential motion will lead to good interface mesh properties.
