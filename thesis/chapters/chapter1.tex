\chapter{\sc Geometric Evolution Equations}
\label{ch:geometric}

\section[Geometric PDEs]{Geometric PDEs}
As starting point, in order to describe the key ideas of the front tracking
approach, we consider the simple problem of purely geometric evolution
equations. A geometric evolution equation defines the motion of a hypersurface
by prescribing its normal velocity in terms of its geometric quantities. These
problems are part of the more general time-dependent interface evolution
problems category, where the normal velocity depends also on field quantities
evaluated on the analysed hypersurface. A detailed description and analysis can
be found in the review article \cite{DeckelnickDE05}.

Interface evolution problems are everywhere in modern physics and engineering.
Several typical applications can be found in materials science such as the
mathematical modelling of the morphology of microstructure in order to
correctly evaluate the mechanical properties of the material or in void
electro-stress migration where small voids or cracks contained in metallic
materials can change their location and shape according to the presence of
surface diffusion and electro-stress loading. Other typical applications which
can be modelled as time-dependent interface evolution problems are the motion of
grain boundaries which separate differing orientations of the same crystalline
phase, or solid-liquid interfaces exhibiting dendritic structures in
under-cooled solidification. Another research fields where these models can be
applied is image processing to detect the separation of dark regions from a
brighter background and to identify separating contours in order to correctly
cluster the objects in the image. Instead, as explained in Chapter
\ref{ch:introduction}, we apply these techniques to incompressible two-phase
flows problem.

A general geometric evolution equation has the following formulation
\begin{equation}\label{eq:geometric_pde}
\vec{\mathcal{V}}\,.\,\vec\nu=f(\vec{z}\,,\vec{\nu}\,,\varkappa),
\qquad\mbox{on }\Gamma(t),
\end{equation}
which prescribes the normal velocity $\vec{\mathcal{V}}\,.\,\vec\nu$ of the
interface $\Gamma(t)$ as a function depending on the position $\vec z$, its
normal direction $\vec{\nu}$ and the sum of its principal curvatures
$\varkappa$.

The simplest geometric PDE is the one which arises from motion by mean
curvature,
\begin{equation}\label{eq:mean_curvature}
\vec{\mathcal{V}}\,.\,\vec\nu=-\varkappa,\qquad\mbox{on }\Gamma(t).
\end{equation}
This equation describes a surface evolving in such a way that its own normal
velocity is equal to the sum of the $d-1$ principal curvatures of $\Gamma(t)$.

Another important geometric PDE is the one which arises from motion by
surface diffusion
\begin{equation}\label{eq:surface_diff}
\vec{\mathcal{V}}\,.\,\vec\nu=-\Delta_s \varkappa, \qquad\mbox{on }\Gamma(t)
\end{equation}
where we use the Laplace--Beltrami operator $\Delta_s$ defined by
(\ref{eq:surface_laplacian}). In this case the surface normal velocity matches
the surface Laplacian of the mean curvature.

For both models, (\ref{eq:mean_curvature}) and (\ref{eq:surface_diff}), it is
necessary to prescribe the initial interface $\Gamma(0)=\Gamma_0$ in order to
have a well posed problem.

\section[Geometric Analysis]{Geometric analysis}
The aim of this section is to collect some useful definitions and results from
differential geometry. Again we refer to \cite{DeckelnickDE05} which cover the
subject in depth.

\section[Front tracking approach]{Front tracking approach}
We always treat the interface using a front tracking approach which involves
seeking a parametrization of the unknown interface over a reference manifold.
More formally, the evolving hypersurface hypersurface $\Gamma(t)$ is
parametrized as
\begin{equation}\label{eq:parametric_hypersurface}
 \Gamma(t)=\vec{x}(\cdot,t)(M)
\end{equation}
where $M\subset\mathbb{R}^{d}$ is the reference manifold and
\begin{equation}\label{eq:position_vector}
 \vec{x}:M\times[0,T)\rightarrow\mathbb{R}^{d}
\end{equation}
is one unknown of the problem. Finally $\vec{x}(\vec{p},t)$, $\vec{p}\in M$, is
the position vector at a certain time $t$ and at a certain point $\vec{p}$ of
the reference manifold. All the geometrical quantities of the hypersurface (e.g.
curvature) can be expressed as derivatives of the parametrization.

It can be shown that both \eqref{eq:mean_curvature} and
\eqref{eq:surface_diff} decrease the surface area of $\Gamma(t)$ over time.
Moreover \eqref{eq:surface_diff} conserves the volume enclosed by $\Gamma(t)$,
which means that a sphere remains exactly the same sphere when it evolves
according to surface diffusion. See \cite{DeckelnickDE05} for more details on
geometric evolution equations and their numerical approximation.

\section[FEM discretization]{FEM discretization}
The Finite Element Method which I use is based on the seminal paper
\cite{Dziuk91} and I refer to the ones described in
\cite{triplej,triplejMC,gflows3d}. The surface diffusion problem
\eqref{eq:surface_diff} can be rewritten as a system of second order equations
\begin{equation}\label{eq:surf_diff_sys}
 \begin{cases}
  \vec{x}_t\cdot\vec{\nu}=-\Delta_{\Gamma(t)} \varkappa,\\
  \varkappa\vec{\nu}=\Delta_{\Gamma(t)}\vec{x},
 \end{cases}
\end{equation}
where we have used the fact that the normal velocity can be expressed as a
function of the parametrization
\begin{equation}
 V=\vec{x}_t\cdot\vec{\nu}.
\end{equation}
The second equation, instead, is a well-known identity from surface geometry.

The FEM scheme for \eqref{eq:surf_diff_sys} from \cite{gflows3d} assumes the
following formulation
\begin{equation}\label{eq:fem_surf_diff}
 \begin{cases}
  \langle \frac{\vec{X}^{m + 1} - \vec{X}^{m}}{\tau_m},
  \chi\vec{\nu}^m\rangle_{\Gamma^m}^{h} - \langle\nabla_{
  \Gamma^m}\kappa^{m+1}, \nabla_{\Gamma^m}\chi \rangle_{\Gamma^m}^{h} = 0,\\
  \langle\kappa^{m+1}\vec{\nu}^m,
  \vec{\eta}\rangle_{\Gamma^m}^{h} + \langle\nabla_{\Gamma^m}\vec{X}^{m + 1},
  \nabla_{\Gamma^m}\vec{\eta} \rangle_{\Gamma^m}=0
 \end{cases}
\end{equation}
$\forall\vec{\eta}\in\underline{V}(\Gamma^m)$ and $\forall\chi\in W(\Gamma^m)$.
These spaces of test functions are defined as
\begin{equation}\label{eq:space_test_functions}
 \underline{V}(\Gamma^m) := \{\vec{\chi}\in C(\Gamma^m,\mathbb{R}^d) :
\vec{\chi}|_{\sigma_j^m}\textrm{ is linear }\forall j=1\rightarrow J\} :=
[W(\Gamma^m)]^d\subset H^1(\Gamma^m,\mathbb{R}^d)
\end{equation}
where $W(\Gamma^m)\subset H^1(\Gamma^m)$ is the space of scalar continuous
piecewise linear functions on $\Gamma^m$. Here $\Gamma^m$ is the polyhedral
surface which approximates the surface $\Gamma (t^m)$ at time $t^m$ and
$\{\sigma_j^m\}_{j=1}^J$ is a family of mutually disjoint open triangles such
that $\Gamma^m=\cup_{j=1}^J\overline{\sigma_j^m}$.

Moreover $\langle\cdot,\cdot\rangle_{\Gamma^m}$ is the $L^2$ inner product over
the current polyhedral surface
\begin{equation}\label{}
 \langle u,v\rangle_{\Gamma^m} = \int_{\Gamma^m}uv\,\mathrm{d}s,\qquad\forall
  u,v\in L^2(\Gamma^m),
\end{equation}
and similarly for $\vec{u},\vec{v}\in L^2(\Gamma^m,\mathbb{R}^d)$. In addition,
$\langle \cdot,\cdot\rangle_{\Gamma^m}^h$ is a mass lumped inner product with
the vertices of each triangle of the mesh as quadrature points.

For what concerns the mean curvature flow, the system of second order equations
is
\begin{equation}\label{eq:mean_curvature_sys}
 \begin{cases}
  \vec{x}_t\cdot\vec{\nu}=-\varkappa,\\
  \varkappa\vec{\nu}=\Delta_{\Gamma(t)}\vec{x},
 \end{cases}
\end{equation}
and the FEM formulation from \cite{gflows3d} is
\begin{equation}\label{eq:fem_mean_curvature}
 \begin{cases}
  \langle \frac{\vec{X}^{m + 1} - \vec{X}^{m}}{\tau_m},
  \chi\vec{\nu}^m\rangle_{\Gamma^m}^{h} - \langle\kappa^{m+1}, \chi
  \rangle_{\Gamma^m}^{h}=0,\\
  \langle\kappa^{m+1}\vec{\nu}^m, \vec{\eta}\rangle_{\Gamma^m}^{h} +
  \langle\nabla_{\Gamma^m}\vec{X}^{m + 1}, \nabla_{\Gamma^m}\vec{\eta}
  \rangle_{\Gamma^m}=0.
 \end{cases}
\end{equation}

\section[Algebraic formulation]{Algebraic formulation}

As regards the mean curvature flow, \eqref{eq:fem_mean_curvature}, the
corresponding algebraic system of equations can be written as

\begin{equation}\label{eq:algebraic_mean_curvature}
 \begin{pmatrix}
  \tau_m \, M_m & - \, \vec{N}_{m}^{T} \\
  \vec{N}_m & \vec{A}_m
 \end{pmatrix}
 \begin{pmatrix}
  \kappa^{m + 1} \\
  \delta \vec{X}^{m + 1}
 \end{pmatrix}
 =
 \begin{pmatrix}
  0 \\
  - \vec{A}_m \, \vec{X}^{m}
 \end{pmatrix} ,
\end{equation}
while, for the system of equations describing the surface diffusion,
\eqref{eq:fem_surf_diff}, the algebraic system is
\begin{equation}\label{eq:algebraic_surf_diff}
 \begin{pmatrix}
  \tau_m \, A_m & - \, \vec{N}_{m}^{T} \\
  \vec{N}_m & \vec{A}_m
 \end{pmatrix}
 \begin{pmatrix}
  \kappa^{m + 1} \\
  \delta \vec{X}^{m + 1}
 \end{pmatrix}
 =
 \begin{pmatrix}
  0 \\
  - \vec{A}_m \, \vec{X}^{m}
 \end{pmatrix}.
\end{equation}
We can notice that the only difference between the above systems occurs in the
upper-left entry of the matrix.

The entries of the matrix are defined as
\begin{eqnarray}\label{eq:algebraic_entries}
 \left[ M_m \right]_{kl} & := & \langle \phi_k^m \, , \phi_l^m \rangle_m^h, \\
 \left[ \vec{N}_m \right]_{kl} & := & \int_{\Gamma^m} \pi^h \left[ \phi_k^m ,
 \phi_l^m \right] \vec{\nu}^m \mathrm{d}s, \\
 \left[ A_m \right]_{kl} & := & \langle \nabla_s \phi_k^m \, , \nabla_s
 \phi_l^m \rangle_m,
\end{eqnarray}
where $\{\phi_k\}$ are the basis functions of the finite element space of
piecewise linear continuous functions $W(\Gamma^m)$ and
\begin{equation}\label{eq:algebraic_entries_vector}
 [\vec{A}_m]_{kl} := [A_m]_{kl} \vec{Id},
\end{equation}
where $\vec{Id}$ is the identity matrix.

The linear systems \eqref{eq:algebraic_mean_curvature} and
\eqref{eq:algebraic_surf_diff} are invertible under a suitable assumption on the
triangulation at each time step, see \cite{gflows3d}. Hence they can be solved
with a sparse factorization package such as \verb|UMFPACK|, see \cite{Davis04}.

Alternately, the algebraic system \eqref{eq:algebraic_mean_curvature} can be
solved with a Schur complement approach:
\begin{subequations}
 \begin{eqnarray}
  \kappa^{m + 1} & = & \frac{1}{\tau_m} M_m^{-1} \vec{N}^{T}_m \, \delta
  \vec{X}^{m + 1} \label{subeq:schur_mcf_1} , \\
  (\vec{A}_m + \frac{1}{\tau_m} \vec{N}_m M_m^{-1} \vec{N}^{T}_m) \, \delta
  \vec{X}^{m + 1} & = & - \vec{A}_m \, \vec{X}^{m} , \label{subeq:schur_mcf_2}
 \end{eqnarray}
\end{subequations}
where \eqref{subeq:schur_mcf_2} is symmetric and positive definite.

\section[Numerical results]{Numerical results}
