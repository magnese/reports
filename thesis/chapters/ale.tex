\chapter[Two-Phase Navier--Stokes Flow ALE FEM]
{\sc ALE Finite Element Approximation for Two-Phase Navier--Stokes Flow}
\label{ch:ale}

\section{Mathematical model}\label{sec:ale_model}
The Arbitrary Lagrangian Eulerian (ALE) method is an alternative technique to
numerically approximate equations on moving domain. It was first proposed in
the papers \cite{Donea83,Hughes81}. See also
\cite{Nobile99,Formaggia04,NobilePhd} for more details. In this formulation the
partial time derivative is expressed with respect to a reference fixed
configuration. A special homeomorphic map, called the ALE map, associates, at
each time $t$, a point in the current computational domain $\Omega(t)$ to a
point in the reference domain $\D$.

In this way, the PDE system resulting after space discretization actually
describes the evolution of the
solution along trajectories that are at all times contained in the computational
domain. The ALE mapping
is somehow arbitrary, apart from the requirement of conforming to the evolution
of the domain boundary,
which is either a given data or the result of the coupling with other
differential models. The latter is the case,
for instance, when treating fluid-structure interaction problems where the
position of the fluid domain boundary is provided through the interaction with
a mechanical model. The map of the boundary $\partial \D$ of
the reference domain has then to provide, at all $t$, the boundary of the
current configuration $\Omega(t)$.
In a numerical simulation we are concerned with the evolution of the discrete
domain, typically built as
the assembly of the elements of a computational grid. The discrete ALE map
describes the evolution of the
grid during the domain movement. It is indeed at discrete level that the
advantage of the ALE formulation
emerges, as in an ALE setting the time advancing scheme provides directly the
evolution of the unknowns
at mesh nodes, and thus, in a classic finite element setting, that of the
degrees of freedom of the discrete
problem at hand.

The system of governing equations (\ref{eq:ns_full_momentum}--i) is expressed in
term of Eulerian coordinates $\vec z$. However, it is possible to rewrite the
velocity time derivative $\vec u_t$ with respect to the so called Arbitrary
Lagrangian Eulerian (ALE) coordinate $\vec q$.

Analogously to what is done in \S\ref{sec:front_tracking_approach} to
parametrize the interface $\Gamma(t)$, we can define the fixed reference
manifold $\D$ and extend the map (\ref{eq:interface_parametrization}) to
parametrize $\Omega$ as $\Omega=\vec x(\D,t)$. This extended map obviously
still satisfy $\Gamma(t) = \vec x(\Upsilon,t)$, given that $\Upsilon\subset\D$.

Now, let $h:\Omega\times [0,T]\to \R$ be a function defined on the Eulerian
frame, the corresponding function on the ALE frame $\hat h$ is defined as
\begin{equation}\label{eq:h}
\hat h:\D\times [0,T]\to \R,\quad \hat h(\vec q,t)=h(\vec x(\vec q,t),t).
\end{equation}
In order to compute the time derivative of (\ref{eq:h}) with respect to the ALE
frame, using the chain rule, we have
\begin{equation}
\frac{\partial\hat h(\vec q,t)}{\partial t}=\frac{\partial h(\vec x(\vec
q,t),t)}{\partial t}=\frac{\partial h(\vec z,t)}{\partial t}+\vec x_t(\vec q,t)
\cdot \nabla h(\vec z,t),
\end{equation}
therefore it holds that
\begin{equation}
\frac{\partial h(\vec z,t)}{\partial t} =
\frac{\partial\hat h(\vec q,t)}{\partial t}-
\vec x_t(\vec q,t) \cdot \nabla h(\vec z,t).
\end{equation}
Finally, introducing the domain velocity
\begin{equation} \label{eq:W}
\W(\vec z, t) := \vec x_t(\vec q, t) \quad \forall\ \vec z = \vec x(\vec q,t)
\in \Omega
\end{equation}
and the time derivative in the ALE frame
\begin{equation} \label{eq:ale_derivative}
\left.\frac{\partial h(\vec z,t)}{\partial t}\right|_{\D}:=
\frac{\partial\hat h(\vec q,t)}{\partial t} \quad
\forall\ \vec z = \vec x(\vec q,t) \in \Omega
\end{equation}
we obtain
\begin{equation}\label{eq:ale_dt}
h_t =\left.h_t\right|_{\D} -\vec{\mathcal{W}} \cdot \nabla h.
\end{equation}
The identity (\ref{eq:ale_dt}) is naturally extended to vector valued functions.
We stress the fact that the domain velocity $\W$ for the interface points is
consistent with the interface velocity $\V$. Therefore it holds
\begin{equation}
\left.\W \right|_{\Gamma(t)}=\V\,.
\end{equation}

Using (\ref{eq:ale_dt}) in (\ref{eq:ns_full_momentum}), we can rewrite the
momentum equation in the ALE frame as:
\begin{align}
\rho\,(\left.\vec u_t\right|_{\D} +(\vec u - \W)\,.\, \nabla \vec u)\,
-2 \mu\,\nabla\,.\, \mat D(\vec u)+ \nabla\,p & = \vec f
\quad &&\mbox{in } \Omega_\pm(t)\,. \label{eq:ns_full_momentum_ale} \\
\end{align}
We can notice that, with respect to the original formulation, there is a
convective-type term due to the domain movement and the time derivative is
computed in the fixed reference frame $\D$. Obviously, if the domain is fixed,
this additional convective term is zero and the time derivative in the ALE
frame coincide with the usual time derivative in the Eulerian frame. Therefore
$\W=\vec 0$ corresponds to a pure Eulerian method, while $\W=\vec u$ corresponds
to a fully Lagrangian scheme.

\section{Weak formulation}\label{sec:ale_weak}
In order to define the ALE weak formulation, we need to use a different
functional setting for the test functions with respect to the one used for the
standard weak formulation and antisymmetric weak formulation, see
\S\ref{sec:ns_weak}. More precisely we consider the admissible spaces of test
functions on the reference domain
\begin{align*}
\widehat{\uspace{b}} &:= \{\vec \phi\in[H^1_0(\D)]^d:\vec \phi =\vec
g\quad \mbox{on }\partial_1\D\}\,,\\
\hat{\pspace} &:= L^2(\D)\,,\\
\widetilde{\hat\pspace} &:= \{\eta\in\hat\pspace : \int_\D\eta\dL{d}=0 \}\,,
\end{align*}
for a given $\vec b \in [H^1_0(\D)]^d$. We notice that the boundary
of $\Omega$ is not time dependent which implies that
$\partial_1\Omega=\partial_1\D$.

Hence, the ALE weak formulation of (\ref{eq:ns_full_momentum_ale},
\ref{eq:ns_full_mass}--i) is given as follows. Given $\Gamma(0) = \Gamma_0$ and
$\vec u = \vec u_0$, for almost all $t\in(0,T)$ find $\Gamma(t)$ and ${(\vec u,
p, \varkappa)\in \uspace g \times \pnormspace \times H^1(\Gamma(t))}$
such that
\begin{subequations}
\begin{align}
& \left(\rho\,\left.\vec u_t\right|_{\D}, \vec \xi\right) +
\left(\rho\,(\vec u - \W)\,.\, \nabla \vec u,\vec \xi\right)
+ 2\left(\mu\,\mat D(\vec u), \mat D(\vec \xi)\right) \nonumber \\
& - \left(p, \nabla\,.\,\vec \xi\right)
- \gamma\,\left\langle \varkappa\,\vec\nu, \vec\xi\right\rangle_{\Gamma(t)}
= \left(\vec f, \vec \xi\right)\quad \forall\ \vec\xi \in \uspace 0 \,,
\label{eq:ns_weaka_ale}\\
& \left(\nabla\,.\,\vec u, \varphi\right) = 0
\quad \forall\ \varphi \in \pnormspace\,, \label{eq:ns_weakb_ale} \\
&  \left\langle \V
- \vec u, \chi\,\vec\nu \right\rangle_{\Gamma(t)} = 0
\quad \forall\ \chi \in H^1(\Gamma(t))\,, \label{eq:ns_weakc_ale} \\
& \left\langle \varkappa\,\vec\nu, \vec\eta \right\rangle_{\Gamma(t)}
+ \left\langle \nabs\,\vec \id, \nabs\,\vec \eta \right\rangle_{\Gamma(t)}
= 0  \quad \forall\ \vec\eta \in [H^1(\Gamma(t))]^d\,\label{eq:ns_weakd_ale}
\end{align}
\end{subequations}
holds for almost all times $t \in (0,T]$.

\section{Finite element approximation}\label{sec:ale_fem}
