\chapter[Two-Phase Navier--Stokes Flow ALE FEM]
{\sc ALE Finite Element Approximation for Two-Phase Navier--Stokes Flow}
\label{ch:ale}
We now present an alternative finite element approximation for incompressible
two-phase Navier--Stokes that uses the ALE method. This technique allows to
rewrite the velocity time derivative with respect to a fixed reference
manifold. The system of PDEs describing the two-phase fluid,
after space discretization,
becomes the evolution of the solution along trajectories. Thus the velocity
does not need to be interpolated anymore, with the exception of when a full
bulk remeshing is performed. This chapter is extensively based on a paper in
preparation.

The chapter is organised as follows: in \S\ref{sec:ale_model} we reformulate the
mathematical model for the incompressible two-phase Navier--Stokes flow using
the ALE method; in \S\ref{sec:ale_weak} we derive the weak formulation on
which our finite element approximation is going to be based; in
\S\ref{sec:ale_semi_fem} we derive a semidiscrete continuous-in-time finite
element approximation, which leads to a system of ODEs describing the solution
along trajectories; in \S\ref{sec:ale_fem} we present the full fitted finite
element discretization; in \S\ref{sec:ale_solution_method} we explain the
the solution method employed to solve the algebraic system.

\section{Mathematical model}\label{sec:ale_model}
The Arbitrary Lagrangian Eulerian (ALE) method is an alternative technique to
numerically approximate equations on a moving domain. It was first proposed in
the papers \cite{Donea83,Hughes81}. In our presentation we follow closely
\cite{Nobile99,Formaggia04,NobilePhd}. The ALE technique consists in
reformulating
the partial time derivative by expressing it with respect to a fixed reference
configuration. A special homeomorphic map, called the ALE map, associates, at
each time $t$, a point in the current computational domain $\Omega(t)$ to a
point in the reference domain $\D$. Therefore, the system of ODEs resulting
after space discretization actually describes the evolution of the solution
along trajectories that are at all times contained in the computational domain.

More precisely, we want to reformulate the two-phase Navier--Stokes system
(\ref{eq:ns_full_momentum}--i), which is expressed in terms of Eulerian
coordinates $\vec z$, by rewriting the velocity time derivative $\vec u_t$ with
respect to the so called ALE coordinate $\vec q$. Analogously to what is done
in \S\ref{sec:front_tracking_approach} to parametrize the interface
$\Gamma(t)$, we can define the fixed reference manifold $\D$ and extend the map
(\ref{eq:interface_parametrization}) to parametrize $\Omega(t)$ as
$\Omega(t)=\vec x(\D,t)$. This extended map clearly still satisfies $\Gamma(t)
= \vec x(\Upsilon,t)$, given that $\Upsilon\subset\D$. Moreover, we let
$\partial\D$ be partitioned as ${\partial\D=\partial_1\D \cup \partial_2\D}$
with $\partial_1\D \cap \partial_2\D = \emptyset$.

Now, let $h:\Omega(t)\times [0,T]\to \R$ be a function defined on the Eulerian
frame. The corresponding function on the ALE frame $\hat h$ is defined as
\begin{equation}\label{eq:h}
\hat h:\D\times [0,T]\to \R,\quad \hat h(\vec q,t)=h(\vec x(\vec q,t),t).
\end{equation}
In order to compute the time derivative of (\ref{eq:h}) with respect to the ALE
frame, using the chain rule, we have
\begin{equation}
\frac{\partial\hat h(\vec q,t)}{\partial t}=\frac{\partial h(\vec x(\vec
q,t),t)}{\partial t}=\frac{\partial h(\vec z,t)}{\partial t}+\vec x_t(\vec q,t)
\cdot \nabla h(\vec z,t),
\end{equation}
therefore it holds that
\begin{equation}
\frac{\partial h(\vec z,t)}{\partial t} =
\frac{\partial\hat h(\vec q,t)}{\partial t}-
\vec x_t(\vec q,t) \cdot \nabla h(\vec z,t).
\end{equation}
Finally, introducing the domain velocity
\begin{equation} \label{eq:W}
\W(\vec z, t) := \vec x_t(\vec q, t) \quad \forall\ \vec z = \vec x(\vec q,t)
\in \Omega(t)
\end{equation}
and the time derivative in the ALE frame
\begin{equation} \label{eq:ale_derivative}
\left.\frac{\partial h(\vec z,t)}{\partial t}\right|_{\D}:=
\frac{\partial\hat h(\vec q,t)}{\partial t} \quad
\forall\ \vec z = \vec x(\vec q,t) \in \Omega(t)\,,
\end{equation}
we obtain
\begin{equation}\label{eq:ale_dt}
h_t =\left.h_t\right|_{\D} -\vec{\mathcal{W}} \cdot \nabla h.
\end{equation}
The identity (\ref{eq:ale_dt}) is naturally extended to vector valued functions.
We stress the fact that the domain velocity $\W$ for the interface points is
consistent with the interface velocity $\V$. Therefore it holds
\begin{equation}
\left.\W \right|_{\Gamma(t)}=\V\,.
\end{equation}
We also point out that the ALE mapping is somehow arbitrary, apart from the
requirement of conforming to the evolution of the domain boundary. Indeed the
map of the boundary $\partial_i \D$ of the reference domain has then to
provide, at all $t$, the boundary $\partial_i \Omega(t)$ of the current
configuration, with $i=1,\,2$.

Using (\ref{eq:ale_dt}) in the momentum equation (\ref{eq:ns_full_momentum}),
we can rewrite
(\ref{eq:ns_full_momentum}--i) in the ALE frame as:
\begin{subequations}
\begin{alignat}{2}
\rho\,(\left.\vec u_t\right|_{\D} +((\vec u - \W)\,.\, \nabla) \vec u)\, && &
\nonumber \\
 -2 \mu\,\nabla\,.\,\mat D(\vec u)+ \nabla\,p & = \vec f
\quad &&\mbox{in } \Omega_\pm(t)\,, \label{eq:ns_full_momentum_ale} \\
\nabla\,.\,\vec u & = 0 \quad &&\mbox{in } \Omega_\pm(t)\,,
\label{eq:ns_full_mass_ale} \\
\vec u & = \vec g \quad &&\mbox{on } \partial_1\Omega(t)\,,
\label{eq:ns_full_dirichlet_ale} \\
\vec u \,.\,\unitn = 0 \,,\quad \mat\sigma\,\unitn\,.\,\unitt & = 0
\quad\forall \unitt \in \{\unitn\}^\perp \quad && \mbox{on }
\partial_2\Omega(t)\,, \label{eq:ns_full_freeslip_ale}\\
[\vec u]_-^+ & = \vec 0 \quad &&\mbox{on } \Gamma(t)\,,
\label{eq:ns_full_jump_velocity_ale} \\
[2\mu \,\mat D(\vec u)\,.\,\vec\nu - p\,\vec \nu]_-^+
& = -\gamma\,\varkappa\,\vec\nu
\quad &&\mbox{on } \Gamma(t)\,, \label{eq:ns_full_jump_stress_ale} \\
(\V-\vec u)\,.\,\vec \nu & = 0
\quad &&\mbox{on } \Gamma(t)\,,\label{eq:ns_full_velocity_ale}  \\
\Gamma(0) & = \Gamma_0 \,,\label{eq:ns_full_initial_interface_ale} \\
\vec u(0) & = \vec u_0 \,.\label{eq:ns_full_initial_velocity_ale}
\end{alignat}
\end{subequations}
As usual, let $\rho(t) = \rho_+\,\charfcn{\Omega_+(t)}+
\rho_-\,\charfcn{\Omega_-(t)}$, with $\rho_\pm \in \R_{>0}$, be the fluid
density in the two phases, let $\mu(t) = \mu_+\,\charfcn{\Omega_+(t)} +
\mu_-\,\charfcn{\Omega_-(t)}$, with $\mu_\pm \in \R_{>0}$, be the dynamic
viscosities in the two phases, let $\mat D(\vec u):=\frac12\, (\nabla\vec
u+(\nabla\vec u)^T)$ be the rate-of-deformation tensor, let
$\vec f:=\rho\,\vec f_1+\vec f_2$ be a possible forcing term, let $\gamma>0$ be
the surface tension coefficient and let $\varkappa$ be the mean curvature of
$\Gamma(t)$. See Chapter~\ref{ch:introduction} for more details. We can notice
that, with respect to the original formulation, there is a convective-type term
due to the domain movement and the time derivative is computed in the fixed
reference frame $\D$. Moreover, contrary to the original system, not only the
regions $\Omega_\pm(t)$ are time dependent but also the whole domain
$\Omega(t)$ is time dependent.

Obviously, if the domain is fixed, the additional convective term is zero
and the time derivative in the ALE frame coincides with the usual time
derivative in the Eulerian frame. This means that $\W=\vec 0$ corresponds to a
pure Eulerian method, while $\W=\vec u$ corresponds to a fully Lagrangian
scheme. In our case, since we have a fix domain $\Omega$, the ALE
formulation might not seem very useful. But, at the discrete level,
we are concerned with the evolution of the discrete triangulated domains.
Therefore the discrete ALE map describes
the evolution of the grid during the domain movement. It is indeed at the
discrete level that the advantage of the ALE formulation emerges,
as in an ALE setting
the time advancing scheme provides directly the evolution of the unknowns at
mesh nodes and thus that of the degrees of freedom of the discrete solutions.

\section{Weak formulation}\label{sec:ale_weak}
In order to define the ALE weak formulation, we need to use a different
functional setting for the test functions with respect to the one used for the
standard and antisymmetric weak formulation, recall \S\ref{sec:ns_weak}.
The reason is that here it
needs to be defined on the moving domain $\Omega(t)$. Therefore we introduce
the admissible spaces of test functions on the reference domain $\D$
\begin{align*}
\uspaceref b &:= \{\vec \phi\in[H^1(\D)]^d:
\vec \phi =\vec b\quad \mbox{on }\partial_1\D\,,
\quad \vec \phi\,.\,\unitn=0 \quad \mbox{on }\partial_2\D\}\,,\\
\pspaceref &:= L^2(\D)\,,\\
\pnormspaceref &:= \{\eta\in\pspaceref : \int_\D\eta\dL{d}=0 \}\,,
\end{align*}
for a given $\vec b \in [H^1(\D)]^d$. Then, using the ALE mapping, we can
define the admissible spaces of test functions on the moving domain $\Omega(t)$
by setting
\begin{align*}
\uspaceale b &:= \{ \vec\phi:
\bigcup_{t \in [0,T]} \Omega(t)\times \{t\} \to \R^d,\quad
\vec \phi=\vec{\hat\phi}\circ\vec x^{\,-1},\quad\vec{\hat\phi}
\in\uspaceref b\}\,,\\
\pspaceale &:= \{ \eta:
\bigcup_{t \in [0,T]} \Omega(t)\times \{t\} \to \R,\quad
\eta=\hat\eta\circ\vec x^{\,-1},\quad\hat\eta\in\pspaceref \}\,,\\
\pnormspaceale &:= \{ \eta:
\bigcup_{t \in [0,T]} \Omega(t)\times \{t\} \to \R,\quad
\eta=\hat\eta\circ\vec x^{\,-1},\quad\hat\eta\in\pnormspaceref \}\,.
\end{align*}
Note that all the functions in $\uspaceref b$, $\pspaceref$ and
$\pnormspaceref$ are independent of time. Moreover, let
$(\cdot,\cdot)_{\Omega(t)}$ and $\langle \cdot, \cdot
\rangle_{\Gamma(t)}$ denote the $L^2$--inner products on $\Omega(t)$ and
$\Gamma(t)$, respectively. In addition, we let $\vol$ and $\surfvol$ denote the
Lebesgue measure in $\R^d$ and the $(d-1)$-dimensional Hausdorff measure,
respectively.

Hence, the ALE weak formulation of (\ref{eq:ns_full_momentum_ale}),
(\ref{eq:ns_full_mass}--i) is given as follows. Given $\Gamma(0) = \Gamma_0$ and
$\vec u = \vec u_0$, for almost all $t\in(0,T)$ find $\Gamma(t)$ and ${(\vec u,
p, \varkappa)\in \uspaceale g \times \pnormspaceale \times H^1(\Gamma(t))}$
such that
\begin{subequations}
\begin{align}
& \left(\rho\,\left.\vec u_t\right|_{\D}, \vec \xi\right)_{\Omega(t)} +
\left(\rho\,((\vec u - \W)\,.\, \nabla )\vec u,\vec \xi\right)_{\Omega(t)}
+ 2\left(\mu\,\mat D(\vec u), \mat D(\vec \xi)\right)_{\Omega(t)} \nonumber \\
& \qquad - \left(p, \nabla\,.\,\vec \xi\right)_{\Omega(t)}
- \gamma\,\left\langle \varkappa\,\vec\nu, \vec\xi\right\rangle_{\Gamma(t)}
= \left(\vec f, \vec \xi\right)_{\Omega(t)}\quad \forall\ \vec\xi \in
\uspaceale 0 \,, \label{eq:ns_weaka_ale}\\
& \left(\nabla\,.\,\vec u, \varphi\right)_{\Omega(t)} = 0
\quad \forall\ \varphi \in \pnormspaceale\,, \label{eq:ns_weakb_ale} \\
&  \left\langle \V
- \vec u, \chi\,\vec\nu \right\rangle_{\Gamma(t)} = 0
\quad \forall\ \chi \in H^1(\Gamma(t))\,, \label{eq:ns_weakc_ale} \\
& \left\langle \varkappa\,\vec\nu, \vec\eta \right\rangle_{\Gamma(t)}
+ \left\langle \nabs\,\vec \id, \nabs\,\vec \eta \right\rangle_{\Gamma(t)}
= 0  \quad \forall\ \vec\eta \in [H^1(\Gamma(t))]^d\,\label{eq:ns_weakd_ale}
\end{align}
\end{subequations}
holds for almost all times $t \in (0,T]$.

\section{Semidiscrete approximation}\label{sec:ale_semi_fem}
We want to derive a semidiscrete continuous-in-time fitted finite element
approximation of (\ref{eq:ns_weaka_ale}--d). Let ${\cal T}^h$ be a regular
partitioning of the reference domain $\D$ into disjoint open simplices
$\sigmaO^h_j$, $j = 1 ,\ldots, J^h_\D$. From now on, the reference domain $\D$,
which we consider, is the polyhedral domain defined by the triangulation
${\cal T}^h$. On ${\cal T}^h$ we define the finite element spaces
\begin{equation*}
S^h_k := \{\chi \in C(\overline{\D}) : \chi\!\mid_{\sigmaO^h}
\in \mathcal{P}_k(\sigmaO^h) \ \forall\ \sigmaO^h \in {\cal T}^h\}\,,
\quad k \in \mathbb{N}\,,
\end{equation*}
where $\mathcal{P}_k(\sigmaO^h)$ denotes the space of polynomials of degree $k$
on $\sigmaO^h$. Moreover, $S^h_0$ is the space of piecewise constant
functions on ${\cal T}^h$. For later use, we also define $\vec I^h_k$ to be
the standard interpolation operator onto $[S^h_k]^d$.

Let $\uspacesemidiscref{g}{h}\subset\uspacesimpleref(\vec I_k^h\vec g)$ and
$\pspaceref^h\subset\pspaceref$ be the finite element spaces we use for the
approximation of velocity and pressure, and let $\pnormspaceref^h:=
\pspaceref^h\cap \pnormspaceref$. The spaces
$(\uspacesemidiscref{0}{h},\pspaceref^m)$ satisfy the LBB inf-sup condition
if there exists a constant $C_0 \in \R_{>0}$, independent of $\mathcal{T}^h(t)$,
such that
\begin{equation} \label{eq:LBB_ale}
\inf_{\hat \varphi \in \pnormspaceref^h} \sup_{\vec{\hat \xi} \in
\uspacesemidiscref{0}{h}}
\frac{( \hat \varphi, \nabla \,.\,\vec{ \hat \xi})} {\|\hat \varphi\|_0\,
\| \vec{\hat\xi}\|_1} \geq C_0 > 0\,,
\end{equation}
see \cite[p.~114]{GiraultR86} and \S\ref{sec:stokes_fem}. Here $\|\cdot\|_0 :=
(\cdot,\cdot)^\frac12$ and $\|\cdot\|_1 := \|\cdot\|_0 + \|\nabla\,\cdot\|_0$
denote the $L^2$--norm and the $H^1$--norm on $\D$, respectively. Then, for
$d=2$, possible pairs $(\uspacesemidiscref{0}{h},\pspace^m)$ that satisfy
(\ref{eq:LBB_ale}) are P2--P0 and P2--(P1+P0), while for $d=3$ are the
P3--(P2+P0) element and P1$^{\mbox{face bubble}}$--P0. See
\S\ref{sec:stokes_fem} for more details.

Then, using the ALE mapping, we can define the finite element spaces on
the discretized moving domain $\Omega^h(t)$ by
\begin{align*}
\uspacesemidiscale{g}{h}{t} &:= \{ \vec\phi:
\Omega^h(t) \to \R^d,\quad
\vec \phi=\vec{\hat\phi}\circ\vec x_h^{\,-1},\quad\vec{\hat\phi}
\in\uspacesemidiscref{g}{h}\}\,,\\
\pnormspaceale^h(t) &:= \{ \eta: \Omega^h(t)\to \R,\quad
\eta=\hat\eta\circ\vec x_h^{\,-1},\quad\hat\eta\in\pnormspaceref^h \}\,,
\end{align*}
where $\Omega^h(t)=\vec x_h(\D,t)$ and $\vec x_h$ is the discrete ALE mapping
discretized spatially using piecewise linear elements. Thus
$\vec x_h\in [S^h_1]^d$ and each simplex with straight faces in ${\cal
T}^h$ is transformed into a simplex with straight faces belonging to the
triangulation $\mathfrak{T}^h(t)$ of the discretized moving domain
$\Omega^h(t)$.

For what concerns the interface, let $(\Gamma^h(t))_{t\in[0,T]}$  be a family of
$(d-1)$-dimensional polyhedral surfaces approximating the closed surface
$\Gamma(t)$. Let $\Omega^h_+(t)$ denote the exterior of $\Gamma^h(t)$ and let
$\Omega^h_-(t)$ be the interior of $\Gamma^h(t)$, where we assume that
$\Gamma^h(t)$ has no self-intersections. Then $\Omega^h(t) = \Omega_-^h(t) \cup
\Gamma^h(t) \cup \Omega_+^h(t)$, and the fitted nature of our method implies
that
\begin{equation} \label{eq:fittedO_ale}
\overline{\Omega^h_+(t)} = \bigcup_{o \in \mathfrak{T}^h_+(t)} \overline{o}
\quad\text{and}\quad
\overline{\Omega^h_-(t)} = \bigcup_{o \in \mathfrak{T}^h_-(t)} \overline{o} \,,
\end{equation}
where $\mathfrak{T}^h(t) = \mathfrak{T}^h_+(t) \cup \mathfrak{T}^h_-(t)$ and
$\mathfrak{T}^h_+(t) \cap \mathfrak{T}^h_-(t) = \emptyset$.
Let $\vec \nu^h(t)$ denote the piecewise constant unit normal to $\Gamma^h(t)$
such that $\vec\nu^h(t)$ points into $\Omega^h_+(t)$. We also define the
piecewise linear finite element spaces $\Wht$ and $\Vht$, with
$\{\chi^h_k(\cdot,t)\}_{k=1}^{K_\Gamma}$ denoting the standard basis of the
former. Hence $\chi^h_k(\vec q^h_l(t),t) = \delta_{kl}$ for all
$k,l \in \{1,\ldots,K_\Gamma\}$ and $t \in [0,T]$, where
$\{\vec q^h_k(t)\}_{k=1}^{K_\Gamma}$ are the vertices of $\Gamma^h(t)$.
We also recall the discrete interface velocity
\begin{equation*}
\V^h(\vec z, t):=
\sum_{k=1}^{K_\Gamma}\left[\ddt\,\vec q^h_k(t)\right] \chi^h_k(\vec z, t)
\in \Vht\,.
\end{equation*}
The discrete domain velocity $\W^h$ is defined exactly the same but for all the
vertices of $\mathfrak{T}^h(t)$.
\footnote{Marco: Better: First define  $\W^h$, and then say that on $\Gamma^h$,
$\V^h$ is just the trace of $\W^h$, and refer to previous definition label.}
Finally, let
$\langle\cdot,\cdot\rangle_{\Gamma^h(t)}^h$ be the mass lumped inner product on
$\Gamma^h(t)$, see (\ref{eq:masslump}), and let
$\langle\cdot,\cdot\rangle_{\Gamma^h(t)}$ denote the standard $L^2$--inner
product on $\Gamma^h(t)$.

Then, given $\Gamma^h(0)$ and $\vec U^h(0)\in \uspacesemidiscale{g}{h}{0}$, for
$t\in (0,T]$ find $\Gamma^h(t)$ and $(\vec U^h(t), P^h(t),
\V^h(t), \kappa^h(t)) \in \uspacesemidiscale{g}{h}{t} \times \pnormspaceale^h(t)
\times \Vht \times \Wht$ such that
\begin{subequations}
\begin{align}
& \left(\rho^h\,\left.\vec U^h_t\right|_{\D}, \vec \xi \right)_{\Omega^h(t)} +
\left(\rho^h((\vec U^h-\W^h)\,.\,\nabla)\,\vec U^h\,,
\,\vec \xi \right)_{\Omega^h(t)} \nonumber \\
& \qquad +2\left(\mu^h\,\mat D(\vec U^h), \mat D(\vec \xi) \right)_{\Omega^h(t)}
- \left(P^h, \nabla\,.\,\vec \xi\right)_{\Omega^h(t)} \nonumber \\
& \qquad - \gamma\,\left\langle \kappa^h\,\vec\nu^h,
\vec\xi\right\rangle_{\Gamma^h(t)} =
\left(\rho^h\,\vec f_1^h+\vec f_2^h, \vec \xi\right)_{\Omega^h(t)}
\quad \forall\ \vec\xi \in \uspacesemidiscale{0}{h}{t}\,,
\label{eq:ns_semi_alea}\\
& \left(\nabla\,.\,\vec U^h, \varphi\right)_{\Omega^h(t)}  = 0
\quad \forall\ \varphi \in \pnormspaceale^h(t)\,, \label{eq:ns_semi_aleb} \\
& \left\langle \V^h , \chi\,\vec\nu^h
\right\rangle_{\Gamma^h(t)}^h - \left\langle \vec U^h, \chi\,\vec\nu^h
\right\rangle_{\Gamma^h(t)} = 0 \quad \forall\ \chi \in \Wht\,,
\label{eq:ns_semi_alec} \\
& \left\langle \kappa^h\,\vec\nu^h, \vec\eta \right\rangle_{\Gamma^h(t)}^h
+ \left\langle \nabs\,\vec\id, \nabs\,\vec \eta \right\rangle_{\Gamma^h(t)} = 0
\quad \forall\ \vec\eta \in \Vht\,. \label{eq:ns_semi_aled}
\end{align}
\end{subequations}
Here we have defined
$\vec f_i^h(\cdot) := \vec I^h_2\,\vec f_i(\cdot,t)$, $i=1,2$,
\begin{equation}
\mu^h = \mu_+\,\charfcn{\Omega^h_+(t)} + \mu_-\,\charfcn{\Omega^h_-(t)}\in
S^h_0(t)
\end{equation}
and
\begin{equation}
\rho^h = \rho_+\,\charfcn{\Omega^h_+(t)} + \rho_-\,\charfcn{\Omega^h_-(t)}\in
S^h_0(t)\,.
\end{equation}

The semidiscrete continuous-in-time fitted finite element scheme
(\ref{eq:ns_semi_alea}--d) is a system of ODEs. Indeed, we can define the
following matrices and vectors
\begin{subequations}
\begin{align}
& [\vec M^h_\Omega(t)]_{ij} :=
\left(\left(\rho^h\,\phi_j^{\uspacesimpleale^h}\vec e_q,
\phi_i^{\uspacesimpleale^h}\,\vec e_r \right)_{\Omega^h(t)}
\right)_{q,r=1}^d\,, \label{eq:ns_ale_mata}\\
& [\vec D^h_\Omega\big(t,\vec U^h,\W^h\big)]_{ij} :=
\left(\left(\rho^h\big(\big(\vec U^h-\W^h\big)\,.\,\nabla\big)
\phi_j^{\uspacesimpleale^h} \vec e_q, \phi_i^{\uspacesimpleale^h}\,\vec e_r
\right)_{\Omega^h(t)} \right)_{q,r=1}^d\,,\label{eq:ns_ale_matb}\\
& [\vec B^h_\Omega(t)]_{ij} :=
2\left(\left(\mu^h\,\mat D(\phi_j^{\uspacesimpleale^h}\vec e_q),
\mat D(\phi_i^{\uspacesimpleale^h}\,\vec e_r) \right)_{\Omega^h(t)}
\right)_{q,r=1}^d\,,\label{eq:ns_ale_matc}\\
& \vec c^h_i(t,\vec f_1^h,\vec f_2^h) := \left( \rho^h\,\vec f_1^h + \vec f_2^h,
\phi_i^{\uspacesimpleale^h}\right)_{\Omega^h(t)}\,,\label{eq:ns_ale_matd}\\
&[\vec C^h_\Omega(t)]_{ip} := - \left(
\left(\nabla\,.\,(\phi_i^{\uspacesimpleale^h}\,\vec e_q), \phi_p^{\pspaceale^h}
\right)_{\Omega^h(t)} \right)_{q=1}^d,\label{eq:ns_ale_mate}\\
& [\Nbulk^h(t)]_{il} := \left\langle \phi_i^{\uspacesimpleale^h}, \chi^h_l
\,\vec\nu^h \right\rangle_{\Gamma^h(t)} \,,\label{eq:ns_ale_matf}\\
& [\vec N_\Gamma^h(t)]_{kl} := \left\langle \chi^h_l, \chi^h_k\,\vec\nu^h
\right\rangle_{\Gamma^h(t)}^h \,,\label{eq:ns_ale_matg}\\
& [\vec A^h_\Gamma(t)]_{kl} := \left\langle \nabs\,\chi^h_l, \nabs\,\chi^h_k
\right\rangle_{\Gamma^h(t)} \,\vec\id \,,\label{eq:ns_ale_math}
\end{align}
\end{subequations}
where $\{\vec e_q\}_{q=1}^d$ denotes the standard basis in $\R^d$,
$\{\phi_i^{\uspacesimpleale^h}\}_{i=1}^{K^h_\uspacesimpleale}$ is the basis of
$\uspacesemidiscale{0}{h}{t}$ and
$\{\phi_i^{\pspaceale^h}\}_{i=1}^{K^h_\pspaceale}$ is the  basis of
$\pnormspaceale^h(t)$. Thus (\ref{eq:ns_semi_alea}--d) can be rewritten in
algebraic form as
\begin{subequations}
\begin{align}
& \vec M^h_\Omega(t)\ddt \vec U + \vec B^h_\Omega(t) \vec U +
\vec D^h_\Omega\big(t,\vec U^h,\W^h\big)\vec U \nonumber \\
& \qquad + \vec C^h_\Omega(t) P
- \gamma \Nbulk^h(t) \kappa = \vec c^h(t,\vec f_1^h,\vec f_2^h)\,,
\label{eq:ns_ode_alea} \\
& [\vec C^h_\Omega(t) ]^T \vec U = 0\,, \label{eq:ns_ode_aleb} \\
& [\vec N_\Gamma^h(t)]^T \ddt \vec X - \Nbulk^h(t)\vec U = 0\,,
\label{eq:ns_ode_alec} \\
& \vec N_\Gamma^h(t) \kappa + \vec A^h_\Gamma(t)\vec X = 0\,,
\label{eq:ns_ode_aled}
\end{align}
\end{subequations}
where $\vec U$, $P$, $\kappa$ and $\vec X$ are the unknown vectors for the
velocity, pressure, interface curvature and interface position, respectively.

\section{Finite element approximation}\label{sec:ale_fem}
We consider the partitioning  $0= t_0 < t_1 < \ldots < t_{M-1} < t_M = T$ of
$[0,T]$ into possibly variable time steps $\tau_m := t_{m+1}-t_m$, $m=0
,\ldots, M-1$. It is now straightforward to obtain a fully discrete
finite element
approximation of (\ref{eq:ns_ode_alea}--d), since the only missing part is the
temporal discretization. Therefore, applying a semi-implicit Euler scheme to
(\ref{eq:ns_ode_alea}--d), we obtain
\footnote{Marco: Why is the first $\vec M^h_\Omega(t^{m+1})$ not a
$\vec M^h_\Omega(t^{m})$?}
\begin{subequations}
\begin{align}
& \frac{1}{\tau_m}\vec M^h_\Omega(t^{m+1})\vec U^{m+1} +
\vec B^h_\Omega(t^m) \vec U^{m+1} +
\vec D^h_\Omega\big(t^m,\vec U^{m+1},\W^{m+1}\big)\vec U^{m+1} \nonumber \\
& \qquad + \vec C^h_\Omega(t^m) P^{m+1}
- \gamma \Nbulk^h(t^m) \kappa^{m+1} \nonumber \\
& \qquad = \frac{1}{\tau_m} \vec M^h_\Omega(t^m)\vec U^m
+ \vec c^h(t^m,\vec f_1^{m+1},\vec f_2^{m+1})\,,\label{eq:ns_ode_disc_alea} \\
& [\vec C^h_\Omega(t^m) ]^T \vec U^{m+1} = 0\,,
\label{eq:ns_ode_disc_aleb} \\
& \frac{1}{\tau_m}[\vec N_\Gamma^h(t^m)]^T \vec X^{m+1}
- \Nbulk^h(t^m)\vec U^{m+1} = \frac{1}{\tau_m}[\vec N_\Gamma^h(t^m)]^T
\vec X^m\,, \label{eq:ns_ode_disc_alec}\\
& \vec N_\Gamma^h(t^m) \kappa^{m+1} + \vec A^h_\Gamma(t^m)\vec X^{m+1}
 = 0\,. \label{eq:ns_ode_disc_aled}
\end{align}
\end{subequations}
We notice that there is a small abuse of notation in (\ref{eq:ns_ode_disc_alea})
for $\vec U^{m+1}$ and $\W^{m+1}$ in the term $\vec D^h_\Omega\big(t^{m},\vec
U^{m+1},\W^{m+1}\big)$. Indeed only for this term, they denote the discrete
fluid velocity and the discrete bulk domain velocity, respectively, and not
their associated vectors of degree of freedoms.

We can now substitute (\ref{eq:ns_ale_mata}--h) back into
(\ref{eq:ns_ode_disc_alea}--d). Firstly, let
$\uspacediscale{g}{m}=\uspacesemidiscale{g}{h}{t^m}$ and
$\pnormspaceale^m=\pnormspaceale^h(t^m)$. Then, our ALE finite element
approximation, which is based on the variational formulation
(\ref{eq:ns_weaka_ale}--d), is given as follows.
\footnote{Marco: This is not good enough. I think what we want to do is to
re-write (\ref{eq:ns_ode_disc_alea}--d) in a more intuitive form. But
before you can do that, you have to say what you mean by these inner products
where functions defined on $\Omega^{m+1}$ are integrated over $\Omega^m$.
See the remark 2 in \cite{Formaggia04}.}
Let $\Gamma^0$, an
approximation to $\Gamma(0)$, and $\vec U^0\in \uspacediscale{g}{0}$ be given.
For $m=0,\ldots, M-1$, find $(\vec U^{m+1},P^{m+1}, \vec X^{m+1}, \kappa^{m+1})
\in \uspacediscale{g}{m}\times \pnormspaceale^m \times \Vh \times \Wh$ such that
\begin{subequations}
\begin{align}
& \left( \frac{\rho^m}{\tau_m}\,\vec U^{m+1}, \vec \xi \right)_{\Omega^{m+1}}
+ \left(\rho^m( (\vec U^{m+1}- \W^{m+1})\,.\,\nabla)\,\vec U^{m+1}\,,
\,\vec \xi \right)_{\Omega^m}\nonumber \\
& \qquad + 2\left(\mu^m\,\mat D(\vec U^{m+1}), \mat D(\vec \xi)
\right)_{\Omega^m} - \left(P^{m+1}, \nabla\,.\,\vec \xi\right)_{\Omega^m}
- \gamma\,\left\langle \kappa^{m+1}\,\vec\nu^m , \vec\xi\right\rangle_{\Gamma^m}
\nonumber \\
& \qquad=  \left( \frac{\rho^m}{\tau_m}\,\vec U^m, \vec \xi \right)_{\Omega^m}
+ \left(\rho^m\,\vec f_1^{m+1}+\vec f_2^{m+1}, \vec \xi\right)_{\Omega^m}
\quad \forall\ \vec\xi \in \uspacediscale{0}{m}\,, \label{eq:ale_HGa} \\
& \left(\nabla\,.\,\vec U^{m+1}, \varphi\right)_{\Omega^m}  = 0
\quad \forall\ \varphi \in \pnormspaceale^m\,,\label{eq:ale_HGb} \\
&  \left\langle \frac{\vec X^{m+1} - \vec\id}{\tau_m} ,\chi\,\vec\nu^m
\right\rangle_{\Gamma^m}^h - \left\langle \vec U^{m+1}, \chi\,\vec\nu^m
\right\rangle_{\Gamma^m}  = 0 \quad \forall\ \chi \in \Wh\,,
\label{eq:ale_HGc}\\
& \left\langle \kappa^{m+1}\,\vec\nu^m, \vec\eta \right\rangle_{\Gamma^m}^h
+ \left\langle \nabs\,\vec X^{m+1}, \nabs\,\vec \eta \right\rangle_{\Gamma^m} =
0 \quad \forall\ \vec\eta \in \Vh\,, \label{eq:ale_HGd}
\end{align}
\end{subequations}
and set $\Gamma^{m+1} = \vec X^{m+1}(\Gamma^m)$. Again, we have defined
$\vec f_i^{m+1}(\cdot) := \vec I^m_2\,\vec f_i(\cdot,t_{m+1})$, $i=1,2$,
\begin{equation}
\mu^m = \mu_+\,\charfcn{\Omega^m_+} + \mu_-\,\charfcn{\Omega^m_-}\in S^m_0
\end{equation}
and
\begin{equation}
\rho^m = \rho_+\,\charfcn{\Omega^m_+} + \rho_-\,\charfcn{\Omega^m_-}\in S^m_0\,.
\end{equation}

We notice that the ALE finite element approximation (\ref{eq:ale_HGa}--d)
differs from the standard implicit finite element approximation
(\ref{eq:ns_HGimpa}--d) for the presence of the arbitrary domain
velocity $\W^{m+1}$ in the convective term of the momentum equation
(\ref{eq:ale_HGa}) and for the different time derivative, again in the momentum
equation. More precisely, in order to compute the discrete time derivative in
(\ref{eq:ale_HGa}), the interpolation of the velocity $\vec U^m$ onto
$\Omega^m$ is not needed anymore.
\footnote{Marco: This paragraph needs heavy re-wording, as ALE and non-ALE
schemes are completely different things. For example, how we interpret the
inner products in the first place.}
\footnote{Marco: Should you mention somewhere that you use $\Omega^m$ as the
reference manifold?}

\section{Solution method}\label{sec:ale_solution_method}
The solution technique used to solve the ALE scheme (\ref{eq:ale_HGa}--d) is
very similar to the one used to solve the implicit standard scheme
(\ref{eq:ns_HGimpa}--d). More precisely, using a fixed point iteration, we find
a solution for the scheme (\ref{eq:ale_HGa}--d) as follows. Let $\Gamma^m$
and $\vec U^m\in \uspacediscale{g}{m}$ be given.
Set $\vec U^{m+1,0}=\vec U^m$, $\vec \psi^{m+1,0} =
\vec 0$, $\Omega^{m+1,0}=\Omega^m$ and fix $\epsilon_f > 0$.
Then, for $s \geq 0$,
find $(\vec U^{m+1,s+1},P^{m+1}, \vec X^{m+1}, \kappa^{m+1}) \in
\uspacediscale{g}{m}\times \pnormspaceale^m \times \Vh \times \Wh$ such that
\begin{subequations}
\begin{align}
& \left( \frac{\rho^m}{\tau_m}\,\vec U^{m+1,s+1}, \vec
\xi\right)_{\Omega^{m+1,s}} + \left((\rho^m\left(\vec
U^{m+1,s}-\frac{\vec\psi^{m+1,s}}{\tau_m}\right)\,.\,\nabla)\,
\vec U^{m+1,s+1}\,, \,\vec \xi \right)_{\Omega^m}\nonumber \\
& \qquad + 2\left(\mu^m\,\mat D(\vec U^{m+1,s+1}), \mat D(\vec \xi)
\right)_{\Omega^m} - \left(P^{m+1}, \nabla\,.\,\vec \xi\right)_{\Omega^m}
- \gamma\,\left\langle \kappa^{m+1}\,\vec\nu^m,\vec\xi\right\rangle_{\Gamma^m}
\nonumber \\
& \qquad = \left( \frac{\rho^m}{\tau_m}\,\vec U^m, \vec \xi \right)_{\Omega^m}
+ \left(\rho^m\,\vec f_1^{m+1}+\vec f_2^{m+1}, \vec \xi\right)_{\Omega^m}
\quad \forall\ \vec\xi \in \uspacediscale{0}{m}\,, \label{eq:ns_HGalea} \\
& \left(\nabla\,.\,\vec U^{m+1,s+1}, \varphi\right)_{\Omega^m}  = 0
\quad \forall\ \varphi \in \pnormspaceale^m\,,\label{eq:ns_HGaleb} \\
&  \left\langle \frac{\vec X^{m+1} - \vec\id}{\tau_m} ,\chi\,\vec\nu^m
\right\rangle_{\Gamma^m}^h - \left\langle \vec U^{m+1,s+1}, \chi\,\vec\nu^m
\right\rangle_{\Gamma^m}  = 0 \quad \forall\ \chi \in \Wh\,,
\label{eq:ns_HGalec}\\
& \left\langle \kappa^{m+1}\,\vec\nu^m, \vec\eta \right\rangle_{\Gamma^m}^h
+ \left\langle \nabs\,\vec X^{m+1}, \nabs\,\vec \eta \right\rangle_{\Gamma^m} =
0 \quad \forall\ \vec\eta \in \Vh\,. \label{eq:ns_HGaled}
\end{align}
\end{subequations}
The iteration is repeated
until $\|U^{m+1,s+1}-U^{m+1,s}\|_{L^\infty} \leq\epsilon_f$
and $\|\vec \psi^{m+1,s+1}-\vec \psi^{m+1,s}\|_{L^\infty} \leq\epsilon_f$.
Here the
bulk displacement $\vec \psi^{m+1,s+1}$ is computed solving the linear
elasticity problem (\ref{eq:elasta}--b), see \S\ref{sec:stokes_smoothing} for
more details, and $\Omega^{m+1,s+1}$ is updated accordingly. Finally, set
${\Gamma^{m+1} = \vec X^{m+1}(\Gamma^m)}$, $\Omega^{m+1,s+1}=\Omega^{m+1}$ and
$\vec U^{m+1}= \vec U^{m+1,s+1}$. Clearly (\ref{eq:ns_HGalea}--d), at each
time level $m$, is a fixed point iteration, which consists of solving a coupled
linear system of equations for the unknowns $(\vec U^{m+1,s+1}, P^{m+1}, \vec
X^{m+1}, \kappa^{m+1})$ at each step $s$ until the $L^\infty$--error of the
velocity $\|U^{m+1,s+1}-U^{m+1,s} \|_{L^\infty}$ and the $L^\infty$--error of
the bulk displacement $\|\vec \psi^{m+1,s+1}-\vec \psi^{m+1,s}\|_{L^\infty}
\leq\epsilon_f$ are smaller than the required tolerance $\epsilon_f$. The
arbitrary domain velocity $\W^{m+1}$ is chosen to be the solution of the linear
elasticity problem (\ref{eq:elasta}--b). Obviously, no mesh smoothing is
performed on $\Omega^{m+1}$ otherwise all the benefits of the ALE scheme would
be lost since a velocity interpolation would be required. Nevertheless, since
the mesh smoothing is already embedded in the arbitrary domain velocity, the
bulk mesh quality is preserved. In practice, we also require $s>1$, which means
that we do at least two full iterations of the fixed point scheme.

At every step, the linear algebraic system arising from (\ref{eq:ns_HGalea}--d)
is identical to the two-phase Stokes algebraic system
(\ref{eq:stokes_algebraic}) but with the following terms
\begin{align*}
& [\vec B_\Omega]_{ij} := \left( \frac{\rho^m}{\tau_m}
\phi_j^{\uspacesimpleale^m},
\phi_i^{\uspacesimpleale^m} \right)_{\Omega^{m+1,s}}\mat \id
+ 2\left(\left(\mu^m\,\mat D(\phi_j^{\uspacesimpleale^m} \vec e_q),
\mat D(\phi_i^{\uspacesimpleale^m}\,\vec e_r)\right)_{\Omega^m}
\right)_{q,r=1}^d\\
& \qquad + \left( \left(\rho^m \left(\vec U^{m+1,s} -
\frac{\vec \psi^{m+1,s}}{\tau_m}\right)\,.\,\nabla \right)
\phi_j^{\uspacesimpleale^m},\phi_i^{\uspacesimpleale^m}
\right)_{\Omega^m}\mat \id \,,\\
& \vec c_i := \left( \frac{\rho^m}{\tau_m}\vec U^m +
\rho^m\,\vec f_1^{m+1}+\vec
f_2^{m+1},\phi_i^{\uspacesimpleale^m}\right)_{\Omega^m}\,,\\
& [\vec C_\Omega]_{ip} := - \left(
\left(\nabla\,.\,(\phi_i^{\uspacesimpleale^m}\,\vec
e_q), \phi_p^{\pspaceale^m} \right)_{\Omega^m} \right)_{q=1}^d,\quad
[\Nbulk]_{il} := \left\langle \phi_i^{\uspacesimpleale^m}, \chi^m_l \,\vec\nu^m
\right\rangle_{\Gamma^m} \,,\\
& \vec \beta_i :=
\frac{\left(\phi_i^{\pspaceale^m},1\right)_{\Omega^m}}{\left(1,1\right)
_{\Omega^m}} \left\langle \vec I_2^m\,\vec
g,\unitn\right\rangle_{\partial_1\Omega^m}\,,\quad
[\vec N_\Gamma]_{kl} := \left\langle \chi^m_l, \chi^m_k\,\vec\nu^m
\right\rangle_{\Gamma^m}^h \,,\\
& [\vec A_\Gamma]_{kl} := \left\langle \nabs\,\chi^m_l, \nabs\,\chi^m_k
\right\rangle_{\Gamma^m} \,\vec\id \,.
\end{align*}
Then the resulting algebraic system can be solved using the same technique
introduced in \S\ref{sec:stokes_solution_method}.
