\chapter{\sc Two--Phase Stokes Flow}
\label{ch:2}

\section[Mathematical model]{Mathematical model}\label{sec:stokes_model}
We consider two-phase Stokes flow in a given domain
$\Omega\subset\mathbb{R}^d$, where $d=2$ or $d=3$. The domain $\Omega$ contains
two different immiscible, incompressible, viscous fluids (liquid-liquid or
liquid-gas) which for all $t\in[0,T]$ occupy time dependent regions
$\Omega_+(t)$ and $\Omega_-(t):=\Omega\setminus\overline{\Omega}_+(t)$ and
which are separated by an interface
$(\Gamma(t))_{t\in[0,T]}$, $\Gamma(t)\subset\Omega$.
See Figure~\ref{fig:sketch} for an illustration.
\begin{figure}
\begin{center}
\unitlength15mm
\psset{unit=\unitlength,linewidth=1pt}
\begin{picture}(4,4)(0,0)
\psline(0,0)(4,0)(4,4)(0,4)(0,0)
\psellipse(2,2)(1,1)
\psline{->}(3,2)(3.5,2)
\put(3.25,1.75){$\vec\nu$}
\put(1.75,0.75){{$\Gamma(t)$}}
\put(1.75,2){{$\Omega_-(t)$}}
\put(0.5,3.25){{$\Omega_+(t)$}}
\end{picture}
\end{center}
\caption{The domain $\Omega$ in the case $d=2$.}
\label{fig:sketch}
\end{figure}
For later use, we assume that $(\Gamma(t))_{t\in [0,T]}$ is a sufficiently
smooth evolving hypersurface without boundary that is parameterized by
$\vec x(\cdot,t):\Upsilon\to\R^d$, where $\Upsilon\subset \R^d$ is a given
reference manifold, i.e.\ $\Gamma(t) = \vec x(\Upsilon,t)$. Then
\begin{equation} \label{eq:V}
\vec{\mathcal{V}}(\vec z, t) := \vec x_t(\vec q, t) \qquad
\forall\ \vec z = \vec x(\vec q,t) \in \Gamma(t)
\end{equation}
defines the velocity of $\Gamma(t)$, and $\vec{\mathcal{V}} \,.\,\vec{\nu}$ is
the normal velocity of the evolving hypersurface $\Gamma(t)$,
where $\vec\nu(t)$ is the unit normal on $\Gamma(t)$ pointing into
$\Omega_+(t)$.

Denoting the velocity and pressure by $\vec u$ and $p$, respectively, we
introduce the stress tensor
\begin{equation} \label{eq:stress_tensor}
\mat\sigma = \mu \,(\nabla\,\vec u + (\nabla\,\vec u)^T) - p\,\mat\id
= 2\,\mu\, \mat D(\vec u)-p\,\mat\id\,,
\end{equation}
where $\mu(t) = \mu_+\,\charfcn{\Omega_+(t)} + \mu_-\,\charfcn{\Omega_-(t)}$,
with $\mu_\pm \in \R_{>0}$, denotes the dynamic viscosities in the two phases,
$\mat\id \in \R^{d \times d}$ is the identity matrix and
$\mat D(\vec u):=\frac12\, (\nabla\vec u+(\nabla\vec u)^T)$
is the rate-of-deformation tensor.

We consider a two-phase Stokes problem, and so the equations
governing the fluid are
\begin{subequations}
\begin{alignat}{2}
- \nabla\,.\,\mat\sigma & = \vec f \qquad &&\mbox{in } \Omega_\pm(t)\,,
\label{eq:stokes_momentum} \\
\nabla\,.\,\vec u & = 0 \qquad &&\mbox{in } \Omega_\pm(t)\,,
\label{eq:stokes_mass}
\end{alignat}
\end{subequations}
where $\vec f$ is a possible forcing term.

On the free surface $\Gamma(t)$, the following conditions need to hold:
\begin{subequations}
\begin{alignat}{2}
[\vec u]_-^+ & = \vec 0 \qquad &&\mbox{on } \Gamma(t)\,,
\label{eq:interface_jump_velocity} \\
[\mat\sigma\,\vec \nu]_-^+ & = -\gamma\,\varkappa\,\vec\nu \qquad
&&\mbox{on } \Gamma(t)\,, \label{eq:interface_jump_stress} \\
\vec{\mathcal{V}}\,.\,\vec\nu &= \vec u\,.\,\vec \nu \qquad
&&\mbox{on } \Gamma(t)\,, \label{eq:interface_velocity}
\end{alignat}
\end{subequations}
where $\gamma>0$ is the surface tension coefficient and $\varkappa$ denotes the
mean curvature of $\Gamma(t)$, i.e.\ the sum of the principal curvatures of
$\Gamma(t)$, where we have adopted the sign convention that $\varkappa$ is
negative where $\Omega_-(t)$ is locally convex. In particular,
see e.g.\ \cite{DeckelnickDE05}, it holds that
\begin{equation} \label{eq:LBop}
\Delta_s\, \vec \id = \varkappa\, \vec\nu \qquad \mbox{on $\Gamma(t)$}\,,
\end{equation}
where $\Delta_s = \nabs\,.\,\nabs$ is the Laplace--Beltrami operator on
$\Gamma(t)$ with $\nabs\,.\,$ and $\nabs$ denoting surface divergence and
surface gradient on $\Gamma(t)$, respectively. Moreover, as usual,
$[\vec u]_-^+ := \vec u_+ - \vec u_-$ and
$[\mat\sigma\,\vec\nu]_-^+ := \mat\sigma_+\,\vec\nu - \mat\sigma_-\,\vec\nu$
denote the jumps in velocity and normal stress across the interface
$\Gamma(t)$. Here and throughout, we employ the shorthand notation
$\vec g_\pm := \vec g\!\mid_{\Omega_\pm(t)}$ for a function
$\vec g : \Omega \times [0,T] \to \R^d$; and similarly for scalar and
matrix-valued functions. To close the system we prescribe the initial data
$\Gamma(0) = \Gamma_0$ and the boundary condition $\vec u = \vec 0$ on
$\partial \Omega$.

Therefore the total system can be rewritten as follows:
\begin{subequations}
\begin{alignat}{2}
-2 \mu\,\nabla\,.\,\mat D(\vec u)+ \nabla\,p & = \vec f
\quad &&\mbox{in } \Omega_\pm(t)\,, \label{eq:full_momentum} \\
\nabla\,.\,\vec u & = 0 \quad &&\mbox{in } \Omega_\pm(t)\,,
\label{eq:full_mass} \\
\vec u & = \vec 0 \qquad &&\mbox{on } \partial\Omega\,,
\label{eq:full_initial_velocity} \\
[\vec u]_-^+ & = \vec 0 \quad &&\mbox{on } \Gamma(t)\,,
\label{eq:full_jump_velocity} \\
[2\mu \,\mat D(\vec u)\,.\,\vec\nu - p\,\vec \nu]_-^+
& = -\gamma\,\varkappa\,\vec\nu
\quad &&\mbox{on } \Gamma(t)\,, \label{eq:full_jump_stress} \\
(\vec{\mathcal{V}}-\vec u)\,.\,\vec{\nu} & = 0
\quad &&\mbox{on } \Gamma(t)\,,\label{eq:full_velocity}  \\
\Gamma(0) & = \Gamma_0 \,.\label{eq:full_initial_interface}
\end{alignat}
\end{subequations}

\section[Weak formulation]{Weak formulation}\label{sec:stokes_weak}
In order to obtain a weak formulation, we define the function spaces
\begin{align*}
\uspace &:= [H^1_0(\Omega)]^d\,,\qquad \pspace := L^2(\Omega) \qquad
\mbox{and}\qquad
\widehat\pspace := \{\eta\in\pspace : \int_\Omega\eta\dL{d}=0 \}\,,
\end{align*}
and let $(\cdot,\cdot)$ and $\langle \cdot, \cdot \rangle_{\Gamma(t)}$ denote
the $L^2$--inner products on $\Omega$ and $\Gamma(t)$, respectively. In
addition, we let $\mathcal{L}^d$ and $\mathcal{H}^{d-1}$ denote the Lebesgue
measure in $\R^d$ and the $(d-1)$-dimensional Hausdorff measure, respectively.

Multiplying (\ref{eq:LBop}) with a test function, and performing integration by
parts, yields
$$
\left\langle \varkappa\,\vec\nu, \vec\eta \right\rangle_{\Gamma(t)}
+ \left\langle \nabs\,\vec \id, \nabs\,\vec \eta \right\rangle_{\Gamma(t)}
= 0  \quad \forall\ \vec\eta \in [H^1(\Gamma(t))]^d\,.
$$
Moreover, on noting (\ref{eq:stress_tensor}) and
(\ref{eq:interface_jump_stress}), we have that
\begin{align*}
\int_{\Omega_+(t)\cup\Omega_-(t)} (\nabla\,.\,\mat\sigma)\,.\, \vec \xi \dL{d}
& = - \left(\mat\sigma, \nabla\,\vec \xi\right)
- \left\langle [\mat\sigma\,\vec\nu]_-^+ , \vec \xi
  \right\rangle_{\Gamma(t)} \nonumber \\
& = \left( p , \nabla\,.\,\vec \xi\right)
-2 \left(\mu\,\mat D(\vec u) , \mat D(\vec \xi) \right)
+ \gamma \left\langle \varkappa\,\vec \nu , \vec \xi  \right\rangle_{\Gamma(t)}
\end{align*}
for all $\vec \xi \in [H^1_0(\Omega)]^d$. Hence a possible weak formulation of
(\ref{eq:full_momentum}--g) is given as follows. Given $\Gamma(0) = \Gamma_0$,
for almost all $t\in(0,T)$ find $\Gamma(t)$ and ${(\vec u, p, \varkappa)}$ ${\in
\uspace \times \widehat\pspace \times H^1(\Gamma(t))}$ such that
\begin{subequations}
\begin{align}
& 2\left(\mu\,\mat D(\vec u), \mat D(\vec \xi)\right)
- \left(p, \nabla\,.\,\vec \xi\right)
- \gamma\,\left\langle \varkappa\,\vec\nu, \vec\xi\right\rangle_{\Gamma(t)}
= \left(\vec f, \vec \xi\right)\quad \forall\ \vec\xi \in \uspace \,,
\label{eq:weaka}\\
& \left(\nabla\,.\,\vec u, \varphi\right) = 0
\quad \forall\ \varphi \in \widehat\pspace\,, \label{eq:weakb} \\
&  \left\langle \vec{\mathcal{V}}
- \vec u, \chi\,\vec\nu \right\rangle_{\Gamma(t)} = 0
\quad \forall\ \chi \in H^1(\Gamma(t))\,, \label{eq:weakc} \\
& \left\langle \varkappa\,\vec\nu, \vec\eta \right\rangle_{\Gamma(t)}
+ \left\langle \nabs\,\vec \id, \nabs\,\vec \eta \right\rangle_{\Gamma(t)}
= 0  \quad \forall\ \vec\eta \in [H^1(\Gamma(t))]^d\,\label{eq:weakd}
\end{align}
\end{subequations}
holds for almost all times $t \in (0,T]$. Here we have observed that if
$p \in \pspace$ is part of a solution to (\ref{eq:full_momentum}--g), then so is
$p + c$ for an arbitrary $c\in \R$.

\section[Energy bound and volume conservation]{Energy bound and volume
conservation}\label{sec:stokes_energy}
It is straightforward to show an a priori energy bound and a volume
conservation property for the system (\ref{eq:weaka}--d). For the former, we
recall from e.g.\ \cite[Lemma~2.1]{DeckelnickDE05} that
\begin{equation}\label{eq:dtarea}
\ddt\,\mathcal{H}^{d-1}(\Gamma(t)) = -
\left\langle \varkappa,\vec{\mathcal{V}}\,.\,\vec\nu\right\rangle_{\Gamma(t)}.
\end{equation}
Hence, on choosing $\vec\xi = \vec u$ in (\ref{eq:weaka}), and noting
(\ref{eq:weakb},c), we obtain that
\begin{align}
\gamma\, \ddt\,\mathcal{H}^{d-1}(\Gamma(t)) = -
\gamma\,\left\langle \varkappa\,\vec\nu, \vec u\right\rangle_{\Gamma(t)}
=  - 2\left(\mu\,\mat D(\vec u), \mat D(\vec u)\right) +
\left(\vec f, \vec u\right) \,,
\label{eq:ap1}
\end{align}
and so in the absence of outer forces, the interfacial energy is monotonically
decreasing.

In order to show the volume conservation property, we recall from e.g.\
\cite[Lemma~2.1]{DeckelnickDE05} that
\begin{align}
\ddt \vol(\Omega_-(t)) & = \left\langle \vec{\mathcal{V}}, \vec\nu
\right\rangle_{\Gamma(t)}\,.
\end{align}
Hence it follows immediately from the incompressibility condition
(\ref{eq:weakb}) and (\ref{eq:weakc}) that
\begin{align}
\ddt \vol(\Omega_-(t)) & = \left\langle \vec u , \vec\nu
\right\rangle_{\Gamma(t)}
 = \int_{\Omega_-(t)} \nabla\,.\,\vec u \dL{d} =0\,. \label{eq:conserved}
\end{align}
It will be our aim to introduce a fitted finite element approximation for
two-phase Stokes flow that satisfies discrete analogous of
(\ref{eq:ap1}) and (\ref{eq:conserved}).

\section[Finite element approximation]{Finite element
approximation}\label{sec:stokes_fem}
We consider the partitioning  $0= t_0 < t_1 < \ldots < t_{M-1} < t_M = T$ of
$[0,T]$ into possibly variable time steps
$\tau_m := t_{m+1}-t_m$, $m=0 ,\ldots, M-1$. Moreover, let
${\cal T}^m$, $\forall m\ge 0$, be a regular partitioning of the domain
$\Omega$ into disjoint open simplices
$\sigmaO^m_j$, $j = 1 ,\ldots, J^m_\Omega$. On ${\cal T}^m$ we define the
finite element spaces
\begin{equation*}
S^m_k := \{\chi \in C(\overline{\Omega}) : \chi\!\mid_{\sigmaO^m}
\in \mathcal{P}_k(\sigmaO^m) \ \forall\ \sigmaO^m \in {\cal T}^m\}
\subset H^1(\Omega),\quad k \in \mathbb{N}\,,
\end{equation*}
where $\mathcal{P}_k(\sigmaO^m)$ denotes the space of polynomials of degree $k$
on $\sigmaO^m$. Moreover, $S^m_0$ is the space of piecewise constant functions
on ${\cal T}^m$.

Let $\uspace^m\subset\uspace$ and $\pspace^m\subset\pspace$ be the finite
element spaces we use for the approximation of velocity and pressure,
and let $\widehat\pspace^m:= \pspace^m \cap \widehat\pspace$.
The spaces $(\uspace^m,\pspace^m)$ satisfy the LBB inf-sup condition if there
exists a constant $C_0 \in \R_{>0}$, independent of $\mathcal{T}^m$, such that
\begin{equation} \label{eq:LBB}
\inf_{\varphi \in \widehat\pspace^m} \sup_{\vec \xi \in \uspace^m}
\frac{( \varphi, \nabla \,.\,\vec \xi)} {\|\varphi\|_0\,\|\vec \xi\|_1}
\geq C_0 > 0\,,
\end{equation}
see \cite[p.~114]{GiraultR86}. Here $\|\cdot\|_0 := (\cdot,\cdot)^\frac12$ and
$\|\cdot\|_1 := \|\cdot\|_0 + \|\nabla\,\cdot\|_0$ denote the $L^2$--norm and
the $H^1$--norm on $\Omega$, respectively. Throughout this thesis, we will
assume that $S^m_0\subset\pspace^m$. Then, for $d=2$, possible pairs
$(\uspace^m,\pspace^m)$ that satisfy (\ref{eq:LBB}) are P2--P0 and P2--(P1+P0),
i.e.\ we set $\uspace^m=[S^m_2]^d\cap\uspace$ and either $\pspace^m = S^m_0$ or
$S^m_1+S^m_0$. We note that the choice P2--(P1+P0) requires the weak constraint
that all simplices have a vertex in $\Omega$, see \cite{BoffiCGG12}. For $d=3$,
pairs of spaces that satisfy $S^m_0\subset\pspace^m$ and (\ref{eq:LBB}) are
the P3--(P2+P0) element, see \cite{BoffiCGG12}, or stabilized spaces such as
P1$^{\mbox{face bubble}}$--P0, see \cite[Remark~8.7.1]{BoffiBF13}, which is
also called the SMALL element.

In this thesis we consider a fitted finite element approximation for the
evolution of the interface $\Gamma(t)$. Let $\Gamma^{m}\subset\R^d$ be a
$(d-1)$-dimensional polyhedral surface approximating the closed surface
$\Gamma(t_m)$, $m=0 ,\ldots, M$. Let $\Omega^m_+$ denote the exterior of
$\Gamma^m$ and let $\Omega^m_-$ be the interior of $\Gamma^m$, where we assume
that $\Gamma^m$ has no self-intersections. Then
$\Omega = \Omega_-^m \cup \Gamma^m \cup \Omega_+^m$, and the fitted nature of
our method implies that
\begin{equation} \label{eq:fittedO}
\overline{\Omega^m_+} = \bigcup_{o \in \mathcal{T}^m_+} \overline{o}
\quad\text{and}\quad
\overline{\Omega^m_-} = \bigcup_{o \in \mathcal{T}^m_-} \overline{o} \,,
\end{equation}
where $\mathcal{T}^m = \mathcal{T}^m_+ \cup \mathcal{T}^m_-$ and
$\mathcal{T}^m_+ \cap \mathcal{T}^m_- = \emptyset$.
Let $\vec{\nu}^m$ denote the piecewise constant unit normal to $\Gamma^m$
such that $\vec\nu^m$ points into $\Omega^m_+$.

In order to define the parametric finite element spaces on $\Gamma^m$, we
assume that $\Gamma^m=\bigcup_{j=1}^{J_\Gamma} \overline{\sigma^m_j}$, where
$\{\sigma^m_j\}_{j=1}^{J_\Gamma}$ is a family of mutually disjoint open
$(d-1)$-simplices with vertices $\{\vec{q}^m_k\}_{k=1}^{K_\Gamma}$. Following
the notation in \cite{spurious}, see also \cite{gflows3d}, we define
$\Vh := \{\vec\chi \in [C(\Gamma^m)]^d:\vec\chi\!\mid_{\sigma^m_j}
\in \mathcal{P}_1(\sigma^m_j), j=1,\ldots, J_\Gamma\} =: [\Wh]^d$,
where $\Wh \subset H^1(\Gamma^m)$ is the space of scalar continuous
piecewise linear functions on $\Gamma^m$, with $\{\chi^m_k\}_{k=1}^{K_\Gamma}$
denoting the standard basis of $\Wh$.

Moreover we define $\pi^m: C(\Gamma^m)\to \Wh$ the standard interpolation
operator at the nodes $\{\vec{q}_k^m\}_{k=1}^{K_\Gamma}$, and similarly
$\vec\pi^m: [C(\Gamma^m)]^d\to \Vh$. Throughout this thesis, we parameterize
the new surface $\Gamma^{m+1}$ over $\Gamma^m$ using a parameterization
$\vec{X}^{m+1} \in \Vh$, so that $\Gamma^{m+1} = \vec{X}^{m+1}(\Gamma^m)$.
Before we can state our numerical method, we need to introduce a mass lumped
inner product on $\Gamma^m$, that is crucial for the desired tangential motion
of vertices on $\Gamma^m$. This induced tangential motion
will lead to good interface mesh properties.
If $v,w \in L^\infty(\Gamma^m)$ are piecewise continuous, with possible jumps
across the edges of $\{\sigma_j^m\}_{j=1}^{J_\Gamma}$, we define the mass
lumped inner product $\langle\cdot,\cdot\rangle_{\Gamma^m}^h$ as
\begin{equation} \label{eq:masslump}
\left\langle v, w \right\rangle^h_{\Gamma^m} :=
\tfrac1d \sum_{j=1}^{J_\Gamma} \mathcal{H}^{d-1}(\sigma^m_j)\,
\sum_{k=1}^{d} (v\,w)((\vec{q}^m_{j_k})^-),
\end{equation}
where $\{\vec{q}^m_{j_k}\}_{k=1}^{d}$ are the vertices of $\sigma^m_j$, and
where we define the limit $v((\vec{q}^m_{j_k})^-)
:= \underset{\sigma^m_j\ni \vec{p}\to \vec{q}^m_{j_k}}{\lim}\, v(\vec{p})$. We
naturally extend (\ref{eq:masslump}) to vector valued functions. Similarly, we
let $\langle\cdot,\cdot\rangle_{\Gamma^m}$ denote the standard $L^2$--inner
product on $\Gamma^m$.

Then our finite element approximation, which is based on the variational
formulation (\ref{eq:weaka}--d), is given as follows. Let $\Gamma^0$ be an
approximation to $\Gamma(0)$. For $m=0,\ldots, M-1$, find $(\vec U^{m+1},
P^{m+1}, \vec{X}^{m+1}, \kappa^{m+1}) \in \uspace^m\times \widehat\pspace^m
\times \Vh \times \Wh$ such that
\begin{subequations}
\begin{align}
& 2\left(\mu^m\,\mat D(\vec U^{m+1}), \mat D(\vec \xi) \right)
- \left(P^{m+1}, \nabla\,.\,\vec \xi\right) - \gamma\,\left\langle
\kappa^{m+1}\,\vec\nu^m, \vec\xi\right\rangle_{\Gamma^m}= \left(\vec f^{m+1},
\vec \xi\right) \nonumber \\
& \hspace{10cm} \quad \forall\ \vec\xi \in \uspace^m \,,\label{eq:HGa}\\
& \left(\nabla\,.\,\vec U^{m+1}, \varphi\right)  = 0
\quad \forall\ \varphi \in \widehat\pspace^m\,,\label{eq:HGb} \\
&  \left\langle \frac{\vec X^{m+1} - \vec\id}{\tau_m} ,\chi\,\vec\nu^m
\right\rangle_{\Gamma^m}^h - \left\langle \vec U^{m+1}, \chi\,\vec\nu^m
\right\rangle_{\Gamma^m}  = 0 \quad \forall\ \chi \in \Wh\,, \label{eq:HGc} \\
& \left\langle \kappa^{m+1}\,\vec\nu^m, \vec\eta \right\rangle_{\Gamma^m}^h
+ \left\langle \nabs\,\vec X^{m+1}, \nabs\,\vec \eta \right\rangle_{\Gamma^m} =
0 \quad \forall\ \vec\eta \in \Vh \label{eq:HGd}
\end{align}
\end{subequations}
and set $\Gamma^{m+1} = \vec{X}^{m+1}(\Gamma^m)$. Here we have defined
$\vec f^{m+1}(\cdot) := \vec I^m_2\,\vec f(\cdot,t_{m+1})$, where $\vec I^m_2$
is the standard interpolation operator onto $[S^m_2]^d$ and $\mu^m =
\mu_+\,\charfcn{\Omega^m_+} + \mu_-\,\charfcn{\Omega^m_-}\in S^m_0$.
We observe that (\ref{eq:HGa}--d) is a linear scheme in that it leads to a
coupled linear system of equations for the unknowns
$(\vec U^{m+1}, P^{m+1}, \vec{X}^{m+1}, \kappa^{m+1})$ at each time level.
We also note that the scheme (\ref{eq:HGa}--d), in the context of an
unfitted finite element approximation, has been considered in \cite{spurious}.
In particular, most of the theoretical results presented in the following
are a direct consequence of the results in \cite{spurious}.
