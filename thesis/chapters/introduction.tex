\chapter{\sc Introduction}\label{ch:introduction}
Fluid flow problems with an unknown interface are ubiquitous in physics and
engineering. This introductory chapter is devoted to a description of the
mathematical model used to model this type of systems together with a brief
overview of the possible approaches to describe the unknown interface.

The chapter is organised as follows: in \S\ref{sec:free_boundary_flows}
we define what a free-boundary fluid flow is and we introduce a generic setting
for the problem; in \S\ref{sec:one_phase_model} we present the one-phase
Navier--Stokes equations and their physical meaning; in
\S\ref{sec:two_phase_model} we discuss the two-phase Navier--Stokes case
and the interface conditions needed for the well posedness of the problem;
in \S\ref{sec:interface_treatment} we briefly overview the different ways to
treat the unknown interface; in \S\ref{sec:numerical_challenges} we detail
the various numerical challenges posed by these types of flows; in
\S\ref{sec:numerical_implementation} we finally list the libraries used to
implement the numerical schemes described in this thesis and we provide the
links to the code developed in order to make all the results presented in this
work reproducible.

\section{Free-boundary fluid flows}\label{sec:free_boundary_flows}
Fluid flow problems with a moving interface are encountered in many
applications in physics, engineering and biophysics. Typical applications are
drops and bubbles, die swell, dam break, liquid storage tanks, dendritic growth,
ink-jet printing, fuel injection. For this reason, developing robust and
efficient numerical methods for these flows is an important problem and
has attracted tremendous interest over the last decade.

In these type of problems, apart from the flow solution in the bulk domain, the
position of a portion of the boundary is also unknown. This boundary can either
be an external boundary or an interface between sub-domains. At the
boundary/interface, certain boundary conditions need to be fulfilled, which
specify the position of the boundary. These conditions relate the variables of
the flow, velocity and pressure, across the domains under consideration of
external influences, such as for example surface tension. Numerically, in order
to be able to compute the flow solution as well as the boundary/interface
geometry, a measure to track the boundary starting from an initial position
needs to be incorporated.

The scenario we consider are two-phase flows, liquid-liquid or liquid-gas, in a
$d$-dimensional computational domain $\Omega\subset\R^d$. This domain
$\Omega$ contains two different immiscible fluids which, for all $t\in[0,T]$,
occupy time dependent regions $\Omega_+(t)$ and
$\Omega_-(t):=\Omega\setminus\overline{\Omega}_+(t)$ and
which are separated by an interface $(\Gamma(t))_{t\in[0,T]}$,
$\Gamma(t)\subset\Omega$. For later use, we always assume that
$(\Gamma(t))_{t\in [0,T]}$ is a sufficiently smooth evolving hypersurface
without boundary. See Figure~\ref{fig:two_phase_sketch} for an illustration.
\begin{figure}
\begin{center}
\unitlength15mm
\psset{unit=\unitlength,linewidth=1pt}
\begin{picture}(4,4)(0,0)
\psline(0,0)(4,0)(4,4)(0,4)(0,0)
\psline{->}(3,2)(3.5,2)
\pscustom[fillstyle=solid,fillcolor=lightgray!50]
{
\psellipse(2,2)(1,1)
}
\put(3.25,1.75){$\vec\nu$}
\put(1.75,0.75){{$\Gamma$}}
\put(1.75,2){{$\Omega_-$}}
\put(0.5,3.25){{$\Omega_+$}}
\end{picture}
\end{center}
\caption[Two-phase flow]{Two-phase flow setting.}
\label{fig:two_phase_sketch}
\end{figure}
Therefore both the sub-domains themselves and the flow field are part of the
solution. Indeed the main task in this context is to account for the interface
$\Gamma(t)$, which separates the distinct fluid domains and is generally in
motion. Apart from its exact position, the computation of geometrical quantities
of $\Gamma(t)$ such as its normal vector $\vec\nu$ and its mean curvature
$\varkappa$ are of interest.

\section{One-phase Navier--Stokes model}\label{sec:one_phase_model}
Before describing the full two-phase Navier--Stokes model, we start introducing
the simpler one-phase case where the full domain $\Omega$ is occupied by one
fluid only. We restrict ourselves to isothermal conditions, incompressible
fluids and we assume that there is no change of phase. For more details see
\cite{GrossR11}.

Denoting the velocity and pressure by $\vec u$ and $p$, respectively, we
introduce the Newtonian stress tensor
\begin{equation} \label{eq:stress_tensor}
\mat\sigma = \mu \,(\nabla\,\vec u + (\nabla\,\vec u)^T) - p\,\mat\id
= 2\,\mu\, \mat D(\vec u)-p\,\mat\id\,,
\end{equation}
where $\mat\id \in \R^{d \times d}$ is the identity matrix and
$\mat D(\vec u):=\frac{1}{2}\, (\nabla\vec u+(\nabla\vec u)^T)$
is the rate-of-deformation tensor with $[\nabla\vec u ]_{ij} =
\frac{\partial u_i}{\partial z_j}$, $1\leq i,j \leq d$. Moreover, let
$\mu$, with $\mu \in \R_{>0}$, be the fluid dynamic viscosity and let $\rho$,
with $\rho \in \R_{>0}$, be the fluid density.

Imposing the conservation of momentum and mass, the following standard
Navier--Stokes governing equations can be derived:
\begin{subequations}
\begin{alignat}{2}
\rho\,(\vec u_t +(\vec u \,.\, \nabla) \vec u)- \nabla\,.\,\mat\sigma
& = \vec f \quad &&\mbox{in } \Omega\,,
\label{eq:ns_momentum_one_phase} \\
\nabla\,.\,\vec u & = 0 \quad &&\mbox{in } \Omega\,,
\label{eq:ns_mass_one_phase}
\end{alignat}
\end{subequations}
where $\vec f$ includes all external body forces referred to the unit mass of
fluid. For later use, we define the divergence of a tensor $\mat H:\R^d\to\R^{d
\times d}$ as the vector with entries
\begin{equation}
\big[\nabla\,.\,\mat H\big]_i=\sum_{j=1}^d{
\frac{\partial\big[\mat H\big]_{ij}}{\partial z_j}}\,, \quad 1\leq i\leq d\,.
\end{equation}

Substituting (\ref{eq:stress_tensor}) in (\ref{eq:ns_momentum_one_phase}), we
obtain the well known one-phase incompressible Navier--Stokes model:
\begin{subequations}
\begin{alignat}{2}
\rho\,(\vec u_t +(\vec u \,.\, \nabla)\vec u) -2 \mu\,\nabla\,.\,\mat D(\vec u)+
\nabla\,p & = \vec f \quad &&\mbox{in } \Omega\,,
\label{eq:ns_momentum_bis_one_phase} \\
\nabla\,.\,\vec u & = 0 \quad &&\mbox{in } \Omega\,.
\label{eq:ns_mass_bis_one_phase}
\end{alignat}
\end{subequations}

To better grasp the physical meaning of the terms appearing in the momentum
equation (\ref{eq:ns_momentum_bis_one_phase}) we briefly describe them. The
inertia term $(\vec u \,.\, \nabla) \vec u$ is a convection term arising from
the conservation of momentum. The momentum of each portion of fluid needs to be
conserved therefore it needs to move with the fluid which means that it is
convected with the fluid. The pressure term $\nabla\,p$  is originated by the
Newtonian constitutive equation (\ref{eq:stress_tensor}) and it represent all
the forces resulting from pressure differences within the fluid. Finally, the
friction term $\nabla\,.\,\mat D(\vec u)$, which again arises from the Newtonian
constitutive equation (\ref{eq:stress_tensor}), is a diffusion operator
equalizing the velocity of neighbouring elements. The more viscous the fluid,
the stronger is the friction between neighbouring particles and thus the
equalizing effect.

The Reynolds number $\Rey$ is an important dimensionless quantity in fluid
mechanics and it is defined as
\begin{equation}\label{eq:reynolds}
\Rey=\frac{\rho\,u_c\,L_c}{\mu}\,,
\end{equation}
where $L_c$ is the characteristic dimension of the problem and $u_c$ is the
characteristic velocity of the fluid. It quantifies the ratio of inertial
forces to viscous forces within the fluid, therefore it is used to predict
flow patterns in different fluid flow situations. When $\Rey \ll 1$ the flow is
laminar due to the fact that viscous forces are dominant and it is
characterized by smooth, constant fluid motion. Instead, when $\Rey \gg 1$ the
flow is turbulent due to the fact that inertial forces are dominant and it
tends to produce chaotic eddies, vortices and other flow instabilities.

In the case of laminar flow, the advective term in the momentum equation
(\ref{eq:ns_momentum_one_phase}) can be neglected, giving rise to the so called
Stokes equations
\begin{subequations}
\begin{alignat}{2}
- \nabla\,.\,\mat\sigma & = \vec f \quad &&\mbox{in } \Omega\,,
\label{eq:momentum_one_phase} \\
\nabla\,.\,\vec u & = 0 \quad &&\mbox{in } \Omega\,,
\label{eq:mass_one_phase}
\end{alignat}
\end{subequations}
which for Newtonian fluids are expressed as
\begin{subequations}
\begin{alignat}{2}
-2 \mu\,\nabla\,.\,\mat D(\vec u)+ \nabla\,p & = \vec f \quad &&\mbox{in }
\Omega\,,
\label{eq:momentum_bis_one_phase} \\
\nabla\,.\,\vec u & = 0 \quad &&\mbox{in } \Omega\,.
\label{eq:mass_bis_one_phase}
\end{alignat}
\end{subequations}

We finally observe that, using the incompressibility condition
(\ref{eq:ns_mass_bis_one_phase}) we have that
\begin{equation}
2\nabla\,.\,\mat D(\vec u) = \nabla\,.\big(\nabla\vec u+(\nabla\vec u)^T\big)=
\Delta \vec u + \nabla(\nabla\,.\vec u) = \Delta \vec u\,.
\end{equation}
Therefore the system (\ref{eq:ns_momentum_bis_one_phase}--b) can be rewritten as
\begin{subequations}
\begin{alignat}{2}
\rho\,(\vec u_t +(\vec u \,.\, \nabla) \vec u) - \mu\,\Delta\vec u+\nabla\,p
& = \vec f \quad \mbox{in } \Omega\,, \\
\nabla\,.\,\vec u & = 0 \quad \mbox{in } \Omega\,,
\end{alignat}
\end{subequations}
which is the canonical form of the one-phase Navier--Stokes problem and the
system (\ref{eq:momentum_bis_one_phase}--b) can be rewritten as
\begin{subequations}
\begin{alignat}{2}
- \mu\,\Delta\vec u+\nabla\,p & = \vec f \quad \mbox{in } \Omega\,, \\
\nabla\,.\,\vec u & = 0 \quad \mbox{in } \Omega\,,
\end{alignat}
\end{subequations}
which is the canonical form of the one-phase Stokes problem.

For the full Navier--Stokes problem (\ref{eq:ns_momentum_bis_one_phase}-b), a
divergence-free velocity field for the whole computational domain is needed as
an initial condition. Therefore we impose
\begin{equation}\label{eq:ns_ic}
\vec u(0) = \vec u_0 \quad \mbox{in } \Omega\,.
\end{equation}
The initial condition is not needed in the Stokes case
(\ref{eq:momentum_bis_one_phase}-b) since it is a stationary problem.

\sloppy In order to obtain a well-posed system, boundary conditions have to be
imposed on the bulk domain boundary $\partial\Omega$. Let $\partial\Omega$ be
partitioned as ${\partial\Omega=\partial_1\Omega \cup \partial_2\Omega}$ with
$\partial_1\Omega \cap \partial_2\Omega = \emptyset$. On $\partial_1 \Omega$
we use the Dirichlet condition
\begin{equation}\label{eq:ns_bc_dirichlet}
\vec u = \vec g \quad \mbox{on } \partial_1 \Omega\,,
\end{equation}
and we notice that the case $\vec g = \vec 0$ corresponds to the so called
no-slip condition. Instead, on $\partial_2 \Omega$ we prescribe the free-slip
condition
\begin{equation}\label{eq:ns_bc_freeslip}
\vec u \,.\,\unitn = 0\,,\quad \mat\sigma\,\unitn\,.\,\unitt = 0
\quad\forall \unitt \in \{\unitn\}^\perp \quad \mbox{on } \partial_2 \Omega\,,
\end{equation}
with $\unitn$ denoting the outer unit normal of $\partial \Omega$ and
$\{\unitn\}^\perp := \{ \unitt \in \R^d : \unitt \,.\,\unitn = 0\}$. The
second term in (\ref{eq:ns_bc_freeslip}) can be derived from the balance of
kinetic energy prescribing a null power transfer through $\partial_2\Omega$.
Therefore, (\ref{eq:ns_bc_freeslip}) is an energy preserving boundary condition,
see \cite{Bothe2013} for more details.

\section{Two-phase Navier--Stokes model}\label{sec:two_phase_model}
We now consider two-phase flows, i.e., the domain $\Omega$ contains two
different immiscible incompressible phases which may move in time
and have different densities and dynamic viscosities. As mentioned before, we
assume isothermal conditions and both phases to be pure substances. Furthermore,
we do not consider reaction, mass transfer or phase transition. For more details
see again \cite{GrossR11}.

Each fluid phase is governed by the Navier--Stokes equations
(\ref{eq:ns_momentum_bis_one_phase}--b), or by the Stokes equations
(\ref{eq:momentum_bis_one_phase}--b). Therefore the two-phase Navier--Stokes
model is simply
\begin{subequations}
\begin{alignat}{2}
\rho\,(\vec u_t +(\vec u \,.\, \nabla)\vec u) -2 \mu\,\nabla\,.\,\mat D(\vec u)+
\nabla\,p & = \vec f \quad &&\mbox{in } \Omega_\pm(t)\,,
\label{eq:ns_momentum_bis} \\
\nabla\,.\,\vec u & = 0 \quad &&\mbox{in } \Omega_\pm(t)\,,
\label{eq:ns_mass_bis}
\end{alignat}
\end{subequations}
where $\mu(t) = \mu_+\,\charfcn{\Omega_+(t)} + \mu_-\,\charfcn{\Omega_-(t)}$,
with $\mu_\pm \in \R_{>0}$, is the dynamic viscosity and
$\rho(t) = \rho_+\,\charfcn{\Omega_+(t)} + \rho_-\,\charfcn{\Omega_-(t)}$,
with $\rho_\pm \in \R_{>0}$, is the fluid density. Here and throughout,
$\charfcn{\D}$ denotes the characteristic function for a set $\D$. Analogously,
the two-phase Stokes model is
\begin{subequations}
\begin{alignat}{2}
-2 \mu\,\nabla\,.\,\mat D(\vec u)+ \nabla\,p & = \vec f \quad &&\mbox{in }
\Omega_\pm(t)\,,
\label{eq:momentum_bis} \\
\nabla\,.\,\vec u & = 0 \quad &&\mbox{in } \Omega_\pm(t)\,.
\label{eq:mass_bis}
\end{alignat}
\end{subequations}

In addition to the initial condition (\ref{eq:ns_ic}), which is needed
only in the Navier--Stokes case, and to the boundary conditions
(\ref{eq:ns_bc_dirichlet}) and (\ref{eq:ns_bc_freeslip}), some interface
conditions are also needed on the unknown interface $\Gamma(t)$, which couple
the velocity and stress between the two domains. Viscosity of the fluids leads
to the continuity condition
\begin{equation}\label{eq:interface_jump_velocity}
[\vec u]_-^+ = \vec 0 \quad \mbox{on } \Gamma(t)\,,
\end{equation}
where $[\vec u]_-^+ := \vec u_+ - \vec u_-$ denote the jump in velocity across
the interface $\Gamma(t)$. Here and throughout, we employ the shorthand notation
$\vec h_\pm := \vec h\!\mid_{\Omega_\pm(t)}$ for a function
$\vec h : \Omega \times [0,T] \to \R^d$; and similarly for scalar and
matrix-valued functions.

The assumption that there is no change of phase leads to the dynamic interface
condition
\begin{equation}\label{eq:interface_velocity}
\V\,.\,\vec\nu = \vec u\,.\,\vec \nu \quad \mbox{on }
\Gamma(t)\,,
\end{equation}
given that $\V$ is the velocity of the evolving interface
$\Gamma(t)$ and where $\vec\nu(t)$ is the unit normal on $\Gamma(t)$ pointing
into $\Omega_+(t)$. Thus the normal velocity of the interface needs to
match the flow normal velocity across the interface $\Gamma(t)$ which means
that no particle of fluid can cross the interface. Therefore, given that the
interface separates the two phases, it imposes no change of phase.

The momentum conservation in a small material volume that intersects the
interface leads to the stress balance condition
\begin{equation}\label{eq:interface_jump_stress_div}
[\mat\sigma\,\vec \nu]_-^+ = -\nabs\,.\mat\sigma_{\Gamma} \quad \mbox{on }
\Gamma(t)\,,
\end{equation}
where $\mat\sigma_{\Gamma}$ is the interface stress tensor and $\nabs\,.$ is
the surface divergence on $\Gamma(t)$. The operator $\nabs$ is the surface
gradient on $\Gamma(t)$ and it is the orthogonal projection of the usual
gradient operator $\nabla$ on the tangent space of the surface $\Gamma(t)$.
Therefore, given a smooth function $h$ defined on a neighbourhood of
$\Gamma(t)$, it is defined as
\begin{equation}\label{eq:surface_gradient}
\nabs\,h(\vec z)=\mat P(\vec z)\,\nabla\,h(\vec z)=\nabla\,h(\vec z)\,-
\,\nabla\,h(\vec z)\,.\,\vec\nu(\vec z)\,\vec\nu(\vec z)\,,\quad \vec z
\in\Gamma(t)\,,
\end{equation}
with the usual projection operator $\mat P$ defined as
\begin{equation}\label{eq:surface_projection}
\mat P=\mat\id\,-\,\vec\nu\,\vec\nu\,^T\quad \mbox{on } \Gamma(t)\,.
\end{equation}
It can be shown that the definition of $\nabs\,h$ does not depend on the chosen
extension of $h$, but only on the value of $h$ on $\Gamma(t)$. See e.g.\
\cite[\S2.1]{DeckelnickDE05} for details.
For later use, we also define the surface
Laplacian, also known as Laplace--Beltrami operator, $\Delta_s$ on $\Gamma(t)$
as
\begin{equation}\label{eq:surface_laplacian}
\Delta_s = \nabs\,.\,\nabs\,.
\end{equation}

We restrict ourselves to the case that the only force acting on the interface
is the surface tension contact force. Therefore the interface stress tensor
$\mat\sigma_\Gamma$ has the following constitutive equation
\begin{equation}\label{eq:interface_stress_tensor}
\mat\sigma_{\Gamma}=\gamma\,\mat P\,,
\end{equation}
where $\gamma$ is the surface tension coefficient. Substituting
(\ref{eq:interface_stress_tensor}) in (\ref{eq:interface_jump_stress_div}) we
obtain
\begin{align}\label{eq:interface_stress_tensor_divergence}
[\mat\sigma\,\vec \nu]_-^+ &= -\nabs\,.(\gamma\,\mat P)
= -\gamma\,\nabs\,.\,\mat P-\nabs\,\gamma
= \gamma\nabs\,.\,(\vec\nu\,\vec\nu^T)-\nabs\,\gamma \nonumber \\
& = \gamma\nabs\,.\,\vec\nu\,\vec\nu + \gamma\nabs\,\vec\nu\,.\,\vec\nu
- \nabs\,\gamma = \gamma\nabs\,.\,\vec\nu\,\vec\nu - \nabs\,\gamma \nonumber \\
& = -\gamma\,\varkappa\,\vec\nu - \nabs\,\gamma \quad \mbox{on } \Gamma(t)\,,
\end{align}
where we have used the fact that the surface gradient $\nabs$ is, by definition,
always orthogonal to the surface normal $\vec\nu$ and that $\varkappa =
-\nabs\,.\vec\nu$ is the mean curvature of $\Gamma(t)$. The projection $\mat P$
is used, since $\mat\sigma_{\Gamma}$ should represent only contact forces that
are tangential to the surface. We assume that $\gamma$ is constant and
therefore, from (\ref{eq:interface_stress_tensor_divergence}), we obtain the
so called clean interface model for the interfacial forces
\begin{equation}
[\mat\sigma\,\vec \nu]_-^+ = -\gamma\,\varkappa\,\vec\nu \quad \mbox{on }
\Gamma(t)\,,
\end{equation}
which, in the case of Newtonian stress tensor (\ref{eq:stress_tensor}), assumes
the form
\begin{equation}\label{eq:interface_jump_stress}
[2\mu \,\mat D(\vec u)\,.\,\vec\nu - p\,\vec \nu]_-^+
= -\gamma\,\varkappa\,\vec\nu \quad \mbox{on } \Gamma(t)\,.
\end{equation}

Moreover, in order to have a well-posed problem, one needs suitable initial
conditions for the interface. Therefore we set $\Gamma(0)=\Gamma_0$.

Finally, we can formulate the two-phase Navier--Stokes model, with the
necessary initial, boundary and interface conditions, as
\begin{subequations}
\begin{alignat}{2}
\rho\,(\vec u_t +(\vec u \,.\, \nabla)\vec u) -2 \mu\,\nabla\,.\,\mat D(\vec u)+
\nabla\,p & = \vec f \quad &&\mbox{in } \Omega_\pm(t)\,, \\
\nabla\,.\,\vec u & = 0 \quad &&\mbox{in } \Omega_\pm(t)\,, \\
\vec u & = \vec g \quad &&\mbox{on } \partial_1\Omega\,, \\
\vec u \,.\,\unitn = 0\,,\quad \mat\sigma\,\unitn\,.\,\unitt & = 0
\quad\forall \unitt \in \{\unitn\}^\perp \quad &&
\mbox{on }\partial_2\Omega\,,\\
[\vec u]_-^+ & = \vec 0 \quad &&\mbox{on } \Gamma(t)\,, \\
[2\mu \,\mat D(\vec u)\,.\,\vec\nu - p\,\vec \nu]_-^+
& = -\gamma\,\varkappa\,\vec\nu \quad &&\mbox{on } \Gamma(t)\,, \\
(\V-\vec u)\,.\,\vec \nu & = 0 \quad &&\mbox{on } \Gamma(t)\,, \\
\Gamma(0) & = \Gamma_0 \,, \\
\vec u(0) & = \vec u_0 \,.
\end{alignat}
\end{subequations}

\section{Interface treatment}\label{sec:interface_treatment}
The dynamics of the interface are determined by the condition
(\ref{eq:interface_velocity}) which, however, describes the dynamics in a
strongly implicit way since the velocity field $\vec u$ depends on the
position of the interface itself. In numerical simulations, there are
several strategies to deal with this problem, which can be divided into two
categories depending on the viewpoint: interface capturing and interface
tracking.

Interface capturing methods use an Eulerian description of the interface, which
is defined implicitly. A characteristic scalar field $\psi$ is used to identify
the two phases as well as the interface along the boundaries of the individual
fluid domains. Depending on the method, this scalar field may be, for example,
a discontinuous Heaviside function or a signed-distance function. The interface
motion is taken into account using a standard advection equation
\begin{equation}
\frac{\partial \psi}{\partial t}+\vec u\,.\nabla\,\psi=0 \quad \mbox{in }
\Omega\,,
\end{equation}
which transports the scalar field with the fluid velocity $\vec u$. The most
important methods, which belong to this category, are the particle method,
the volume-of-fluid method, the level-set method and the phase-field method.
The particle method utilizes mass-less particles distributed in an Eulerian
mesh to capture the fluid flow and in particular the interface position, see
e.g \cite{Girault1976}. In the volume-of-fluid method, the characteristic
function of one of the phases is approximated numerically, see e.g.
\cite{HirtN81,RenardyR02,Popinet09}. In the level-set method, the interface is
given as the level set of a function, which has to be determined, see e.g.
\cite{SussmanSO94,Sethian99,OsherF03,GrossR07,Svacek17}. Instead, the
phase-field method works with diffuse interfaces, and therefore the transition
layer between the phases has a finite size. There is no tracking mechanism for
the interface, but the phase state is included implicitly in the governing
equations. The interface is associated with a smooth, but highly localized
variation of the so-called phase-field variable. We refer to
\cite{HohenbergH77,AndersonMW98,LowengrubT98,Feng06,KaySW08,AbelsGG12,GrunK14}
for details. In general, the great advantage of the interface capturing
approaches is that they are inherently able to deal with topological changes of
the interface. This allows a much more flexible interface description than in
interface tracking approaches. On the other hand, it is challenging to treat
accurately discontinuous quantities across the interface, to have mass
conservation within each phase and to apply boundary conditions along the
interface.

Interface tracking approaches, instead, use a Lagrangian description of the
interface which is described explicitly. The idea is to take a (virtual)
particle on the interface at time $t=t_0$ with Eulerian coordinates $\vec z(0)
\in \Gamma_0$. Let $\vec z(t)$ be the Eulerian coordinates of this particle at
time $t$, with $t\geq t_0$. The particles on the interface are transported by
the flow velocity $\vec u$, therefore it satisfies the equation
\begin{equation}
\frac{\partial \vec z(t)}{\partial t}=\vec u(\vec z(t),t)\,,
\end{equation}
which determines the path of a material particle and consequently it
characterizes the evolution of the interface. This interface representation
forms the basis of the interface tracking methods. In these methods a
collection of markers is put on a given interface $\Gamma_0$ and then
transported (numerically) by the flow velocity $\vec u$ to obtain the markers
on the interface $\Gamma(t)$. Usually, the collection of markers on $\Gamma_0$
are the set of vertices of the triangulation of $\Gamma_0$. The great
advantage of interface tracking approaches is that they offer an accurate and
computationally efficient approximation of the evolving interface. Moreover,
the imposition of boundary conditions at the interface is simple since the mesh
nodes lie on the interface itself. However, firstly the mesh quality will
usually degrade quickly in the case of large deformations and secondly
topological changes require a special treatment. In order to keep an
acceptable mesh quality throughout the simulation, mesh smoothing techniques
are applied and, when they are not enough, a full remeshing is used. If
topological changes occur, the only way to tackle the issue is to
perform a remeshing with respect to the new interface. The main consequence of
remeshing is the need to project the field values from the old to the new
mesh, a ceaseless source for errors, and at the same time computationally
costly especially in dimension 3. Therefore, it is desirable to keep the
frequency of remeshing as low as possible. We refer e.g. to
\cite{UnverdiT92,Bansch01,Tryggvason_etal01,GanesanMT07,GanesanT08,spurious,
fluidfbp} for further details, and to \cite{LevequeL97,Peskin02} for the
related immersed boundary method, which is used to simulate fluid-structure
interactions using Eulerian coordinates for the fluid and Lagrangian
coordinates for the structure.

In the following chapters, we always use an interface tracking approach to
describe the interface evolution. In particular, we adopt a fitted mesh approach
which means that the discrete interface is composed by bulk element faces.
Consequently there is a strong coupling between the bulk mesh and the interface
mesh.

\section{Numerical challenges}\label{sec:numerical_challenges}
Two-phase flows pose enormous challenges to numerical simulation tools and
they can not be solved reliably and accurately by the commercial software that
are available nowadays. Therefore, compared to one-phase flow solvers, special
tailor-made numerical schemes have to be developed. Below we address some causes
of the very high numerical complexity of this problem class.

The evolution of the unknown interface is determined by the simple dynamic
condition (\ref{eq:interface_velocity}). Unfortunately, it turns out that an
accurate numerical approximation of the geometric object representing the
interface and its evolution is a very difficult task. This task become even
more complicated in cases of geometric singularities such as droplet break up or
collision. Although several techniques have been developed, an accurate
interface approximation remains a challenging task.

For high Reynolds number, the flow model is strongly nonlinear. Therefore the
coupling between fluid dynamics and interface evolution turns out to be strongly
nonlinear. This nonlinearity cause difficulties for the construction of
accurate numerical schemes for time discretization.

Usually several quantities are discontinuous across the interface. Indeed, for
example, the value of density and viscosity have jumps across the interface,
since the interface separates two different fluids. Moreover, due to surface
tension forces, also the pressure is discontinuous across the interface. In
most applications the stress balance (\ref{eq:interface_jump_stress_div}) and a
jump in the viscosity across the interface typically induce a discontinuity
across the interface of the normal derivative of the velocity. Of particular
importance is the precise inclusion of surface tension terms, and the correct
handling of discontinuity jumps in the material properties and in the pressure
at the interface, in order to suppress spurious velocities, which are also
called parasitic currents. Apart from stationary problems, the unknown
interface is changing as a function of time and therefore these discontinuities
are moving. The numerical treatment of moving discontinuities often causes
severe difficulties.

The full discretization in space and in time results in a very large nonlinear
system of equations at each time step. Most of the total computing time is
spent solving these large nonlinear systems and therefore it is crucial to use
efficient iterative solvers. The efficiency can be largely improved by using
special techniques that are adapted to the problem class, in the sense that the
solver makes use of certain structural properties of the problem.

\section{Numerical implementation}\label{sec:numerical_implementation}
We implement all numerical schemes from scratch in \verb|C++| within the
\verb|DUNE| framework, see \cite{dunegridpaperI08,dunegridpaperII08}.
\verb|DUNE|, the Distributed and Unified Numerics Environment is a modular and
extensible toolbox for solving partial differential equations with grid-based
methods. It supports different schemes based on finite elements, finite volumes
and finite differences. We use the following \verb|DUNE| modules:
\verb|dune-common| (build system, infrastructure and foundation classes),
\verb|dune-istl| (algebraic solvers), \verb|dune-geometry| (geometric
entities), \verb|dune-grid| (grid manager interface) and \verb|dune-fem| (FEM
toolbox), see \cite{dunefempaper10}. More information on the library can be
found on the official website \myurl{https://www.dune-project.org/}.

We actively contributed to the various modules of the library providing
upstream patches for new features and bug fixing with more than $4\cdot 10^4$
lines of changes.

As grid manager we use \verb|ALBERTA|, see \cite{Alberta}, available here
\myurl{http://www.alberta-fem.de/}. We also use the sparse factorization
package \verb|SuiteSparse|, see \cite{Davis04}, available here
\myurl{http://faculty.cse.tamu.edu/davis/suitesparse.html}.

\sloppy We implement the scheme described in Chapter \ref{ch:geometric_pdes} as
a \verb|DUNE| module named \verb|dune-geometric-pde|, available here
\myurl{https://github.com/magnese/dune-geometric-pde}, while the schemes
described in Chapter \ref{ch:stokes} and Chapter \ref{ch:navier_stokes}
are implemented as a \verb|DUNE| module called
\verb|dune-navier-stokes-two-phase|, available here
\myurl{https://github.com/magnese/dune-navier-stokes-two-phase}.

All the meshes are created with the library \verb|Gmsh|, see
\cite{GeuzaineR09}, available here \myurl{http://gmsh.info/} and the plots are
generated with either \verb|ParaView|, available here
\myurl{http://www.paraview.org/}, or \verb|Gnuplot|, available here
\myurl{http://www.gnuplot.info/}.

All the simulations we report on are performed on a Linux cluster equipped with
\verb|Ubuntu 14.04.1 LTS| and compiled with \verb|g++ 5.3.0|. The cluster is
heterogeneous and consist of 32 computing nodes. Each node is equipped
with dual Xeon CPUs containing either 4 or 6 processor cores with frequency
ranging from 2.40GHz to 3.00GHz and memory between 16GB and 48GB. The CPU times
are measured in seconds and correspond to the wall time of a single thread
computation.
