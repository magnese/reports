\chapter{\sc Introduction}
\label{ch:introduction}

\section[Free-boundary fluid flows]{Free-boundary fluid flows}
Fluid flow problems with a moving interface are encountered in many
applications in physics, engineering and biophysics. Typical applications are
drops and bubbles, die swell, dam break, liquid storage tanks, dendritic growth,
ink-jet printing, fuel injection. For this reason, developing robust and
efficient numerical methods for these flows is an important problem and
has attracted tremendous interest over the last decade.

In these type of problems, apart from the flow solution in the bulk domain, the
position of a portion of the boundary is also unknown. This boundary can either
be an external boundary or an interface between sub-domains. At the
boundary/interface, certain boundary conditions need to be fulfilled, which
specify the position of the boundary. These conditions relate the variables of
the flow, velocity and pressure, across the domains under consideration of
external influences, such as for example surface tension. Numerically, in order
to be able to compute the flow solution as well as the boundary/interface
geometry, a measure to track the boundary starting from an initial position
needs to be incorporated.

The scenario we consider are two-phase flows, liquid-liquid or liquid-gas, in a
$d$-dimensional computational domain $\Omega\subset\mathbb{R}^d$. This domain
$\Omega$ contains two different immiscible fluids which occupy time dependent
regions $\Omega_+(t)$ and $\Omega_-(t)$ and which are separated by an interface
$\Gamma(t)$. See Figure~\ref{fig:two_phase} for an illustration.
\begin{figure}
\begin{center}
\unitlength15mm
\psset{unit=\unitlength,linewidth=1pt}
\begin{picture}(4,4)(0,0)
\psline(0,0)(4,0)(4,4)(0,4)(0,0)
\pscustom[fillstyle=solid,fillcolor=lightgray!50]
{
\psarcn(0.75,2.5){0.75}{180}{0}
\psarc(2.75,2.5){1.25}{180}{360}
\psline(4,2.5)(4,0)(0,0)(0,2.5)
}
\psline{->}(2.75,1.25)(2.75,2)
\put(2.9,1.5){$\vec\nu$}
\put(1.6,2.25){{$\Gamma$}}
\put(0.5,1){{$\Omega_-$}}
\put(2.5,3){{$\Omega_+$}}
\end{picture}
\end{center}
\caption{Two-phase flow setting.}
\label{fig:two_phase}
\end{figure}
Therefore both the sub-domains themselves and the flow field are part of the
solution. Indeed the main task in this context is to account for the interface
$\Gamma(t)$, which separates the distinct fluid domains and is generally in
motion. Apart from its exact position, the computation of geometrical quantities
of $\Gamma(t)$ such as its normal vector $\nu$ and its curvature $\varkappa$ are
of interest.