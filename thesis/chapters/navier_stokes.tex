\chapter{\sc Two-Phase Navier--Stokes Flow}\label{ch:navier_stokes}

\section{Mathematical model}\label{sec:navier_stokes_model}
We now consider two-phase Navier--Stokes flow in a given domain
$\Omega\subset\R^d$, where $d=2$ or $d=3$. As already described in
\S\ref{sec:free_boundary_flows} and \S\ref{sec:stokes_model}, the domain
$\Omega$ contains two different immiscible, incompressible fluids
(liquid-liquid or liquid-gas) which for all $t\in[0,T]$ occupy time
dependent regions $\Omega_+(t)$ and
$\Omega_-(t):=\Omega\setminus\overline{\Omega}_+(t)$ and which are separated by
an interface $(\Gamma(t))_{t\in[0,T]}$, $\Gamma(t)\subset\Omega$.
Consistently with two-phase Stokes flow, we limit ourselves to interfaces
formed by closed hypersurfaces, see Figure~\ref{fig:two_phase_sketch_closed}
for a pictorial representation in dimension $d=2$.

We treat the interface $\Gamma(t)$ with the same technique used for the
two-phase Stokes, see Chapter~\ref{ch:stokes}, therefore we use a front tracking
approach, see \S\ref{sec:front_tracking_approach}, which parametrizes the
unknown interface $\Gamma(t)$ as $\vec x(\cdot,t):\Upsilon\to\R^d$ where
$\Upsilon\subset\R^d$ is a given reference manifold such that $\Gamma(t) = \vec
x(\Upsilon,t)$. As usual, we require that the evolving hypersurface is
sufficiently smooth and without boundary. The velocity $\V$ of $\Gamma(t)$ is
defined by the equation (\ref{eq:V}) which we report here for the seek of
completeness
\begin{equation*}
\V(\vec z, t) := \vec x_t(\vec q, t) \quad
\forall\ \vec z = \vec x(\vec q,t) \in \Gamma(t)\,,
\end{equation*}
where $\V \,.\,\vec{\nu}$ is the normal velocity of the evolving hypersurface
$\Gamma(t)$ and $\vec\nu(t)$ is the unit normal on $\Gamma(t)$ pointing into
$\Omega_+(t)$.

The fluid dynamics in the bulk domain $\Omega$ is governed by the two-phase
Navier--Stokes model (\ref{eq:ns_momentum_bis}--b) which describes the velocity
$\vec u$ and pressure $p$ fields of the fluid. The velocity and stress tensor,
see (\ref{eq:stress_tensor}), needs to be coupled across the free surface
$\Gamma(t)$ therefore we impose the interface conditions
(\ref{eq:interface_jump_velocity}), (\ref{eq:interface_jump_stress}) and
(\ref{eq:interface_velocity}). In order to close the system, we prescribe the
initial data $\Gamma(0) = \Gamma_0$ and the initial velocity
$\vec u(0) = \vec u_0$. Finally, we impose an homogeneous Dirichlet boundary
condition on $\partial_1 \Omega$ and a free-slip condition on $\partial_2
\Omega$, with $\partial\Omega =\partial_1\Omega \cup \partial_2\Omega$ and
$\partial_1\Omega \cap \partial_2\Omega = \emptyset$. Therefore the total system
can be rewritten as follows:
\begin{subequations}
\begin{alignat}{2}
\rho\,(\vec u_t +\vec u \,.\, \nabla \vec u) -2 \mu\,\nabla\,.\,\mat D(\vec u)+
\nabla\,p & = \vec f
\quad &&\mbox{in } \Omega_\pm(t)\,, \label{eq:ns_full_momentum} \\
\nabla\,.\,\vec u & = 0 \quad &&\mbox{in } \Omega_\pm(t)\,,
\label{eq:ns_full_mass} \\
\vec u & = \vec 0 \quad &&\mbox{on } \partial_1\Omega\,,
\label{eq:ns_full_dirichlet} \\
\vec u \,.\,\unitn & = 0 \quad &&\mbox{on } \partial_2\Omega\,,
\label{eq:ns_full_freeslip} \\
[\vec u]_-^+ & = \vec 0 \quad &&\mbox{on } \Gamma(t)\,,
\label{eq:ns_full_jump_velocity} \\
[2\mu \,\mat D(\vec u)\,.\,\vec\nu - p\,\vec \nu]_-^+
& = -\gamma\,\varkappa\,\vec\nu
\quad &&\mbox{on } \Gamma(t)\,, \label{eq:ns_full_jump_stress} \\
(\V-\vec u)\,.\,\vec{\nu} & = 0
\quad &&\mbox{on } \Gamma(t)\,,\label{eq:ns_full_velocity}  \\
\Gamma(0) & = \Gamma_0 \,,\label{eq:ns_full_initial_interface} \\
\vec u(0) & = \vec u_0 \,,\label{eq:ns_full_initial_velocity}
\end{alignat}
\end{subequations}
where $\unitn$ denotes the outer unit normal of $\partial \Omega$.
Finally let $\rho(t) = \rho_+\,\charfcn{\Omega_+(t)}+
\rho_-\,\charfcn{\Omega_-(t)}$, with $\rho_\pm \in \R_{>0}$, be the fluid
density in the two phases, let $\mu(t) = \mu_+\,\charfcn{\Omega_+(t)} +
\mu_-\,\charfcn{\Omega_-(t)}$, with $\mu_\pm \in \R_{>0}$, be the dynamic
viscosities in the two phases, let $\mat D(\vec u):=\frac12\, (\nabla\vec
u+(\nabla\vec u)^T)$ be the rate-of-deformation tensor, let $\vec f$ be a
possible forcing term, let $\gamma>0$ be the surface tension coefficient and let
$\varkappa$ be the mean curvature of $\Gamma(t)$. See
Chapter~\ref{ch:introduction} for more details.

\section{Standard weak formulation}\label{sec:ns_weak_standard}
A standard weak formulation of (\ref{eq:ns_full_momentum}--i) can be obtained
proceeding analogously to the Stokes case, see \S~\ref{sec:stokes_weak}. We
define the same function spaces
\begin{align*}
\uspace &:= [H^1_0(\Omega)]^d\,,\quad \pspace := L^2(\Omega) \quad
\mbox{and}\quad
\pnormspace := \{\eta\in\pspace : \int_\Omega\eta\dL{d}=0 \}\,,
\end{align*}
and, as usual, let $(\cdot,\cdot)$ and $\langle \cdot, \cdot
\rangle_{\Gamma(t)}$ denote the $L^2$--inner products on $\Omega$ and
$\Gamma(t)$, respectively. In addition, we let $\vol$ and $\surfvol$ denote the
Lebesgue measure in $\R^d$ and the $(d-1)$-dimensional Hausdorff measure,
respectively.

The weak counterpart of the differential geometry identity (\ref{eq:LBop})
can be obtained multiplying (\ref{eq:LBop}) with a test function $\vec\eta$,
and performing integration by parts, yielding
\begin{equation*}
\left\langle \varkappa\,\vec\nu, \vec\eta \right\rangle_{\Gamma(t)}
+ \left\langle \nabs\,\vec \id, \nabs\,\vec \eta \right\rangle_{\Gamma(t)}
= 0  \quad \forall\ \vec\eta \in [H^1(\Gamma(t))]^d\,.
\end{equation*}
Moreover, on noting (\ref{eq:stress_tensor}) and
(\ref{eq:interface_jump_stress}), we have that
\begin{align*}
\int_{\Omega_+(t)\cup\Omega_-(t)} (\nabla\,.\,\mat\sigma)\,.\, \vec \xi \dL{d}
& = - \left(\mat\sigma, \nabla\,\vec \xi\right)
- \left\langle [\mat\sigma\,\vec\nu]_-^+ , \vec \xi
  \right\rangle_{\Gamma(t)} \nonumber \\
& = \left( p , \nabla\,.\,\vec \xi\right)
-2 \left(\mu\,\mat D(\vec u) , \mat D(\vec \xi) \right)
+ \gamma \left\langle \varkappa\,\vec \nu , \vec \xi  \right\rangle_{\Gamma(t)}
\end{align*}
for all $\vec \xi \in [H^1_0(\Omega)]^d$.

Hence, the standard weak formulation of is given as follows. Given $\Gamma(0) =
\Gamma_0$ and $\vec u = \vec u_0$, for almost all $t\in(0,T)$ find $\Gamma(t)$
and ${(\vec u, p, \varkappa)}$ ${\in \uspace \times \widehat\pspace \times
H^1(\Gamma(t))}$ such that
\begin{subequations}
\begin{align}
& \left(\rho\,\vec u_t, \vec \xi\right) + \left(\rho\,\vec u \,.\, \nabla \vec
u,\vec \xi\right) + 2\left(\mu\,\mat D(\vec u), \mat D(\vec \xi)\right)
\nonumber \\
& - \left(p, \nabla\,.\,\vec \xi\right)
- \gamma\,\left\langle \varkappa\,\vec\nu, \vec\xi\right\rangle_{\Gamma(t)}
= \left(\vec f, \vec \xi\right)\quad \forall\ \vec\xi \in \uspace \,,
\label{eq:ns_weaka}\\
& \left(\nabla\,.\,\vec u, \varphi\right) = 0
\quad \forall\ \varphi \in \pnormspace\,, \label{eq:ns_weakb} \\
&  \left\langle \V
- \vec u, \chi\,\vec\nu \right\rangle_{\Gamma(t)} = 0
\quad \forall\ \chi \in H^1(\Gamma(t))\,, \label{eq:ns_weakc} \\
& \left\langle \varkappa\,\vec\nu, \vec\eta \right\rangle_{\Gamma(t)}
+ \left\langle \nabs\,\vec \id, \nabs\,\vec \eta \right\rangle_{\Gamma(t)}
= 0  \quad \forall\ \vec\eta \in [H^1(\Gamma(t))]^d\,\label{eq:ns_weakd}
\end{align}
\end{subequations}
holds for almost all times $t \in (0,T]$. Again, we notice that if
$p \in \pspace$ is part of a solution to (\ref{eq:ns_full_momentum}--i), then
so is $p + c$ for an arbitrary $c\in \R$.
