\chapter[Two-Phase Navier--Stokes Flow Standard FEMs]
{\sc Standard Finite Element Approximations for Two-Phase Navier--Stokes
Flow}\label{ch:navier_stokes}

\section{Mathematical model}\label{sec:navier_stokes_model}
We now consider two-phase Navier--Stokes flow in a given domain
$\Omega\subset\R^d$, where $d=2$ or $d=3$. As already described in
\S\ref{sec:free_boundary_flows} and \S\ref{sec:stokes_model}, the domain
$\Omega$ contains two different immiscible, incompressible fluids
(liquid-liquid or liquid-gas) which for all $t\in[0,T]$ occupy time
dependent regions $\Omega_+(t)$ and
$\Omega_-(t):=\Omega\setminus\overline{\Omega}_+(t)$ and which are separated by
an interface $(\Gamma(t))_{t\in[0,T]}$, $\Gamma(t)\subset\Omega$.
Consistently with two-phase Stokes flow, we limit ourselves to interfaces
formed by closed hypersurfaces, see Figure~\ref{fig:two_phase_sketch}
for a pictorial representation in dimension $d=2$.

We treat the interface $\Gamma(t)$ with the same technique used for the
two-phase Stokes, see Chapter~\ref{ch:stokes}, therefore we use a front tracking
approach, see \S\ref{sec:front_tracking_approach}, which parametrizes the
unknown interface $\Gamma(t)$ as $\vec x(\cdot,t):\Upsilon\to\R^d$ where
$\Upsilon\subset\R^d$ is a given reference manifold such that $\Gamma(t) = \vec
x(\Upsilon,t)$. As usual, we require that the evolving hypersurface is
sufficiently smooth and without boundary. The velocity $\V$ of $\Gamma(t)$ is
defined by the equation (\ref{eq:V}) which we report here for the sake of
completeness
\begin{equation*}
\V(\vec z, t) := \vec x_t(\vec q, t) \quad
\forall\ \vec z = \vec x(\vec q,t) \in \Gamma(t)\,,
\end{equation*}
where $\V \,.\,\vec \nu$ is the normal velocity of the evolving hypersurface
$\Gamma(t)$ and $\vec\nu(t)$ is the unit normal on $\Gamma(t)$ pointing into
$\Omega_+(t)$.

The fluid dynamics in the bulk domain $\Omega$ is governed by the two-phase
Navier--Stokes model (\ref{eq:ns_momentum_bis}--b) which describes the velocity
$\vec u$ and pressure $p$ fields of the fluid. The velocity and stress tensor,
see (\ref{eq:stress_tensor}), need to be coupled across the free surface
$\Gamma(t)$ therefore we impose the interface conditions
(\ref{eq:interface_jump_velocity}), (\ref{eq:interface_jump_stress}) and
(\ref{eq:interface_velocity}). In order to close the system, we prescribe the
initial data $\Gamma(0) = \Gamma_0$ and the initial velocity
$\vec u(0) = \vec u_0$. Finally, we impose the Dirichlet condition $\vec u =
\vec g$ on $\partial_1 \Omega$ and the free-slip condition $\vec u \,.\,\unitn =
0$ and $\mat\sigma\,\unitn\,.\,\unitt = 0$, $\forall \unitt \in
\{\unitn\}^\perp$, on $\partial_2 \Omega$, with $\unitn$ denoting the outer
unit normal of $\partial \Omega$ and $\{\unitn\}^\perp := \{ \unitt \in \R^d :
\unitt \,.\,\unitn = 0\}$ As usual, it holds that
$\partial\Omega =\partial_1\Omega \cup \partial_2\Omega$ and $\partial_1\Omega
\cap \partial_2\Omega = \emptyset$. Therefore the total system can be rewritten
as follows:
\begin{subequations}
\begin{alignat}{2}
\rho\,(\vec u_t +(\vec u \,.\, \nabla)\vec u) -2 \mu\,\nabla\,.\,\mat D(\vec u)+
\nabla\,p & = \vec f
\quad &&\mbox{in } \Omega_\pm(t)\,, \label{eq:ns_full_momentum} \\
\nabla\,.\,\vec u & = 0 \quad &&\mbox{in } \Omega_\pm(t)\,,
\label{eq:ns_full_mass} \\
\vec u & = \vec g \quad &&\mbox{on } \partial_1\Omega\,,
\label{eq:ns_full_dirichlet} \\
\vec u \,.\,\unitn = 0 \,,\quad \mat\sigma\,\unitn\,.\,\unitt & = 0
\quad\forall \unitt \in \{\unitn\}^\perp \quad && \mbox{on } \partial_2\Omega\,,
\label{eq:ns_full_freeslip}\\
[\vec u]_-^+ & = \vec 0 \quad &&\mbox{on } \Gamma(t)\,,
\label{eq:ns_full_jump_velocity} \\
[2\mu \,\mat D(\vec u)\,.\,\vec\nu - p\,\vec \nu]_-^+
& = -\gamma\,\varkappa\,\vec\nu
\quad &&\mbox{on } \Gamma(t)\,, \label{eq:ns_full_jump_stress} \\
(\V-\vec u)\,.\,\vec \nu & = 0
\quad &&\mbox{on } \Gamma(t)\,,\label{eq:ns_full_velocity}  \\
\Gamma(0) & = \Gamma_0 \,,\label{eq:ns_full_initial_interface} \\
\vec u(0) & = \vec u_0 \,.\label{eq:ns_full_initial_velocity}
\end{alignat}
\end{subequations}
Here we let $\rho(t) = \rho_+\,\charfcn{\Omega_+(t)}+
\rho_-\,\charfcn{\Omega_-(t)}$, with $\rho_\pm \in \R_{>0}$, be the fluid
density in the two phases, let $\mu(t) = \mu_+\,\charfcn{\Omega_+(t)} +
\mu_-\,\charfcn{\Omega_-(t)}$, with $\mu_\pm \in \R_{>0}$, be the dynamic
viscosities in the two phases, let $\mat D(\vec u):=\frac12\, (\nabla\vec
u+(\nabla\vec u)^T)$ be the rate-of-deformation tensor, let $\vec f$ be a
possible forcing term, let $\gamma>0$ be the surface tension coefficient and let
$\varkappa$ be the mean curvature of $\Gamma(t)$. See
Chapter~\ref{ch:introduction} for more details.

\section{Weak formulation}\label{sec:ns_weak}
A standard weak formulation of (\ref{eq:ns_full_momentum}--i) can be obtained
proceeding analogously to the Stokes case, see \S\ref{sec:stokes_weak}. We
define the same function spaces
\begin{align*}
\uspace b &:= \{\vec \phi\in[H^1(\Omega)]^d:
\vec \phi =\vec b\quad \mbox{on }\partial_1\Omega\,,
\quad \vec \phi\,.\,\unitn=0 \quad \mbox{on }\partial_2\Omega\}\,,\\
\pspace &:= L^2(\Omega)\,,\\
\pnormspace &:= \{\eta\in\pspace : \int_\Omega\eta\dL{d}=0 \}\,,
\end{align*}
for a given $\vec b \in [H^1(\Omega)]^d$. As usual, let $(\cdot,\cdot)$ and
$\langle \cdot, \cdot \rangle_{\Gamma(t)}$ denote the $L^2$--inner products on
$\Omega$ and $\Gamma(t)$, respectively. In addition, we let $\vol$ and
$\surfvol$ denote the Lebesgue measure in $\R^d$ and the $(d-1)$-dimensional
Hausdorff measure, respectively.

Using (\ref{eq:weak_nabs_id}) and (\ref{eq:weak_stress_jump}), we can write the
standard weak formulation of (\ref{eq:ns_full_momentum}--i) as follows. Given
$\Gamma(0) = \Gamma_0$ and $\vec u = \vec u_0$, for almost all $t\in(0,T)$ find
$\Gamma(t)$ and ${(\vec u, p, \varkappa)}$ ${\in \uspace g \times \pnormspace
\times H^1(\Gamma(t))}$ such that
\begin{subequations}
\begin{align}
& \left(\rho\,\vec u_t, \vec \xi\right) + \left(\rho\,(\vec u \,.\, \nabla) \vec
u,\vec \xi\right) + 2\left(\mu\,\mat D(\vec u), \mat D(\vec \xi)\right)
\nonumber \\
& \qquad - \left(p, \nabla\,.\,\vec \xi\right)
- \gamma\,\left\langle \varkappa\,\vec\nu, \vec\xi\right\rangle_{\Gamma(t)}
= \left(\vec f, \vec \xi\right)\quad \forall\ \vec\xi \in \uspace 0 \,,
\label{eq:ns_weaka}\\
& \left(\nabla\,.\,\vec u, \varphi\right) = 0
\quad \forall\ \varphi \in \pnormspace\,, \label{eq:ns_weakb} \\
&  \left\langle \V
- \vec u, \chi\,\vec\nu \right\rangle_{\Gamma(t)} = 0
\quad \forall\ \chi \in H^1(\Gamma(t))\,, \label{eq:ns_weakc} \\
& \left\langle \varkappa\,\vec\nu, \vec\eta \right\rangle_{\Gamma(t)}
+ \left\langle \nabs\,\vec \id, \nabs\,\vec \eta \right\rangle_{\Gamma(t)}
= 0  \quad \forall\ \vec\eta \in [H^1(\Gamma(t))]^d\,\label{eq:ns_weakd}
\end{align}
\end{subequations}
holds for almost all times $t \in (0,T]$. Again, we notice that if
$p \in \pspace$ is part of a solution to (\ref{eq:ns_full_momentum}--i), then
so is $p + c$ for an arbitrary $c\in \R$.

An alternative to the weak formulation (\ref{eq:ns_weaka}--d) can be obtained
following the paper \cite{fluidfbp}. We refer to this alternative formulation
as antisymmetric weak formulation. The authors show that it is easy to derive
an energy bound not only on the continuous level but also on discrete level, as
well as existence/uniqueness on discrete level. In order to obtain this
alternative weak formulation, we want to rewrite the term $\left(\rho\,\vec u_t,
\vec \xi\right) + \left(\rho\,(\vec u \,.\, \nabla) \vec u,\vec \xi\right)$ in
(\ref{eq:ns_weaka}) in an equivalent antisymmetric form.

First, we observe that for arbitrary functions $\vec v$,
$\vec \omega$, $\vec \xi \in [H^1(\Omega)]^d$ it holds that
\begin{align}\label{eq:tripleterm}
& [(\vec v\,.\,\nabla)\,\vec \omega]\,.\,\vec \xi
 = (\vec v\,.\,\nabla)\,(\vec \omega\,.\,\vec \xi) -
[(\vec v\,.\,\nabla)\,\vec \xi]\,.\,\vec \omega \nonumber \\
& = \tfrac{1}{2}\,(\vec v\,.\,\nabla)\,
(\vec \omega\,.\,\vec \xi) + \tfrac{1}{2}\,(\vec v\,.\,\nabla)\,
(\vec \omega\,.\,\vec \xi) -\tfrac{1}{2}[(\vec v\,.\,\nabla)\,\vec
\xi] \,.\,\vec \omega -\tfrac{1}{2}[(\vec v\,.\,\nabla)\,\vec
\xi] \,.\,\vec \omega \nonumber \\
& = \tfrac{1}{2}\,(\vec v\,.\,\nabla)\, (\vec \omega\,.\,\vec \xi) +
\tfrac{1}{2}[(\vec v\,.\,\nabla)\,\vec \omega]\,.\,\vec \xi -
\tfrac{1}{2}[(\vec v\,.\,\nabla)\,\vec \xi] \,.\,\vec \omega\,.
\end{align}

Choosing $\vec v = \vec \omega = \vec u$ in (\ref{eq:tripleterm}) and
substituting it in the convection term $\left(\rho\,(\vec u \,.\, \nabla)
\vec u,\vec \xi\right)$ we obtain
\begin{equation}\label{eq:ns_advect_tripleterm}
( \rho\,(\vec u \,.\,\nabla)\,\vec u, \vec \xi) =
\tfrac{1}{2}(\rho,(\vec u\,.\,\nabla)(\vec u\,.\,\vec \xi))
+ \tfrac{1}{2}(\rho\,(\vec u \,.\,\nabla)\,\vec u, \vec \xi)
-\tfrac{1}{2}(\rho\,(\vec u\,.\,\nabla)\,\vec \xi,\vec u)\,.
\end{equation}

Then we have for any $\vec v \in [H^1(\Omega)]^d$ and
$\phi\in W_0^{1,\frac{3}{2}}(\Omega)$ that
\begin{align}\label{eq:ibp0}
( \rho,(\vec v \,.\,\nabla)\,\phi) & = (\rho, \nabla\,.\,(\phi\,\vec v))
- (\rho\,(\nabla\,.\,\vec v), \phi) \nonumber \\ & =
- \left\langle [\rho]_-^+\,\vec v\,.\,\vec \nu,
  \phi \right\rangle_{\Gamma(t)}
- (\rho\,(\nabla\,.\,\vec v), \phi)\,.
\end{align}
Here we notice that (\ref{eq:ibp0}) is well defined. Indeed, $\phi\in
W_0^{1,\frac{3}{2}}(\Omega)$ implies that its trace is in
$L^{\frac{3}{2}}(\partial\Omega)$. Moreover, given that
$\vec v \in [H^1(\Omega)]^d$, it holds that $\vec v\,.\,\vec\nu \in
[H^{\frac{1}{2}}(\partial\Omega)]^d$, which implies, thanks to the continuous
embedding theorems \cite[Theorem~6.5 and 6.7]{DINEZZA2012}, that $\vec
v\,.\,\vec\nu \in [L^4(\partial\Omega)]^d$. But we know, using the
H\"{o}lder inequality, that the term $\left\langle [\rho]_-^+\,\vec v\,.\,
\vec \nu, \phi \right\rangle_{\Gamma(t)}$ is well defined if
$\vec v\,.\,\vec\nu$ is at least in $L^3(\partial\Omega)$, which is indeed the
case.

Moreover, if $\vec u\in [H^1(\Omega)]^d$ and $\vec \xi \in \uspace{0}$ then
$\vec u\,.\,\vec\xi \in W_0^{1,\frac{3}{2}}(\Omega)$. Indeed, $H^1(\Omega)$ is
compactly embedded in $L^6(\Omega)$, for $d=2$ or $d=3$. By the
generalized H\"{o}lder inequality we know that $(\nabla h)\omega\in
L^{\frac{3}{2}}(\Omega)$ if $h\in H^1(\Omega)$ and $\omega\in L^6(\Omega)$.
Therefore $\nabla(\vec u\,.\,\vec \xi)\in L^{\frac{3}{2}}(\Omega)$ which implies
$\vec u\,.\,\vec \xi \in W_0^{1,\frac{3}{2}}(\Omega)$. Hence, it follows from
taking $\phi = \vec u\,.\,\vec\xi$ and $\vec v = \vec u$ in (\ref{eq:ibp0}) and
applying it to the first term on the right-hand side of
(\ref{eq:ns_advect_tripleterm}) that
\begin{align}\label{eq:fulladvect}
( \rho\,(\vec u \,.\,\nabla)\,\vec u, \vec \xi) = &
-\tfrac{1}{2}\langle [\rho]_-^+\vec u\,.\,\vec \nu, \vec u\,.\,\vec \xi
\rangle_{\Gamma(t)}
-\tfrac{1}{2}(\rho\,(\nabla\,.\,\vec u ),\vec u\,.\,\vec\xi) \nonumber \\
& \qquad +\tfrac{1}{2}(\rho\,(\vec u\,.\,\nabla)\,\vec u, \vec \xi)
-\tfrac{1}{2}(\rho\,(\vec u\,.\,\nabla)\,\vec \xi,\vec u)
\quad \forall\ \vec \xi \in \uspace 0\,,
\end{align}
and, using the incompressibility condition (\ref{eq:ns_full_mass}), we obtain
\begin{align}\label{eq:advect}
( \rho\,(\vec u \,.\,\nabla)\,\vec u, \vec \xi)
= & \tfrac{1}{2} [ (\rho\,(\vec u\,.\,\nabla)\,\vec u, \vec \xi) -
(\rho\,(\vec u\,.\,\nabla)\,\vec \xi,\vec u) \nonumber \\
& \qquad -\langle [\rho]_-^+\vec u\,.\,\vec \nu, \vec u\,.\,\vec \xi
\rangle_{\Gamma(t)}] \quad \forall\ \vec \xi \in \uspace 0\,.
\end{align}

Next, we can use Reynolds transport theorem, together with
(\ref{eq:ns_full_velocity}), to rewrite the time derivative term
$(\rho\,\vec u_t,\vec \xi)$ and obtain
\begin{align}\label{eq:rhot1}
\ddt(\rho \,\vec u, \vec \xi) & =
\ddt\left(\rho_+\,\int_{\Omega_+(t)} \vec u \,.\,\vec \xi \dL{d}
+ \rho_-\,\int_{\Omega_-(t)} \vec u\,.\,\vec \xi \dL{d}  \right) \nonumber \\
& =  (\rho\,\vec u_t, \vec \xi)
-\left\langle [\rho]_-^+\,\V\,.\,\vec\nu, \vec u \,.\,\vec \xi
\right\rangle_{\Gamma(t)} \nonumber \\
&= (\rho\,\vec u_t, \vec \xi)- \left\langle [\rho]_-^+\,\vec u\,.\,\vec \nu,
\vec u \,.\,\vec \xi \right\rangle_{\Gamma(t)}
\quad \forall \vec \xi \in \uspace 0\,.
\end{align}
Therefore, it follows from (\ref{eq:rhot1}) that
\begin{equation*}
(\rho\,\vec u_t, \vec \xi) =
\tfrac{1}{2} \left[
\ddt (\rho\,\vec u,\vec \xi) + (\rho\,\vec u_t, \vec \xi)
+ \left\langle [\rho]_-^+\,\vec u\,.\,\vec \nu,
\vec u \,.\,\vec \xi \right\rangle_{\Gamma(t)}
\right]
\quad \forall\ \vec \xi \in \uspace 0\,,
\end{equation*}
which on combining with (\ref{eq:advect}) yields that
\begin{align} \label{eq:rhot3}
& (\rho\,[\vec u_t + (\vec u\,.\,\nabla)\,\vec u], \vec \xi) \nonumber \\
& = \tfrac{1}{2}\bigg[ \ddt (\rho\,\vec u, \vec \xi)
+ (\rho\,\vec u_t, \vec \xi) + (\rho, [(\vec u\,.\,\nabla)\,\vec u]\,.\,\vec \xi
- [(\vec u\,.\,\nabla)\,\vec \xi]\,.\,\vec u) \bigg]
\quad \forall \vec \xi \in \uspace 0\,.
\end{align}

Hence, the antisymmetric weak formulation of (\ref{eq:ns_full_momentum}--i) is
given as follows. Given $\Gamma(0) = \Gamma_0$ and $\vec u = \vec u_0$, for
almost all $t\in(0,T)$ find $\Gamma(t)$ and ${(\vec u, p, \varkappa)}$ ${\in
\uspace g \times \pnormspace \times H^1(\Gamma(t))}$ such that
\begin{subequations}
\begin{align}
& \tfrac{1}{2}\bigg[ \ddt (\rho\,\vec u, \vec \xi) + (\rho\,\vec u_t, \vec \xi)
+ (\rho, [(\vec u\,.\,\nabla)\,\vec u]\,.\,\vec \xi
- [(\vec u\,.\,\nabla)\,\vec \xi]\,.\,\vec u)\bigg] \nonumber \\
& \qquad +2\left(\mu\,\mat D(\vec u), \mat D(\vec \xi)\right)
- \left(p, \nabla\,.\,\vec \xi\right)\nonumber \\
& \qquad - \gamma\,\left\langle \varkappa\,\vec\nu, \vec\xi
\right\rangle_{\Gamma(t)}
= \left(\vec f, \vec \xi\right)\quad \forall\ \vec\xi \in \uspace 0 \,,
\label{eq:ns_weaka_antisym}\\
& \left(\nabla\,.\,\vec u, \varphi\right) = 0
\quad \forall\ \varphi \in \pnormspace\,, \label{eq:ns_weakb_antisym} \\
&  \left\langle \V
- \vec u, \chi\,\vec\nu \right\rangle_{\Gamma(t)} = 0
\quad \forall\ \chi \in H^1(\Gamma(t))\,, \label{eq:ns_weakc_antisym} \\
& \left\langle \varkappa\,\vec\nu, \vec\eta \right\rangle_{\Gamma(t)}
+ \left\langle \nabs\,\vec \id, \nabs\,\vec \eta \right\rangle_{\Gamma(t)}
= 0  \quad \forall\ \vec\eta \in [H^1(\Gamma(t))]^d\,\label{eq:ns_weakd_antisym}
\end{align}
\end{subequations}
holds for almost all times $t \in (0,T]$.

\section{Antisymmetric finite element approximation}\label{sec:ns_fem_antisym}
In order to obtain a fully practical finite element approximation of the
antisymmetric weak formulation (\ref{eq:ns_weaka_antisym}--d) we proceed
analogously to the Stokes case, see \S\ref{sec:stokes_fem}. For the benefit of
the reader, we state again all the hypothesis and the notations used.

Here, in order to have a consistent finite element approximation, we consider
the partitioning $t_m =m\,\tau$, $m=0,\ldots, M$, of $[0,T]$ into uniform time
steps $\tau=\frac{T}{M}$, see \cite{fluidfbp}. Moreover, let ${\cal T}^m$,
$\forall m\ge 0$, be a regular partitioning of the domain $\Omega$ into
disjoint open simplices $\sigmaO^m_j$, $j = 1 ,\ldots, J^m_\Omega$. From now
on, the domain $\Omega$ which we consider is the polyhedral domain defined by
the triangulation ${\cal T}^m$ On ${\cal T}^m$ we define the finite element
spaces
\begin{equation*}
S^m_k := \{\chi \in C(\overline{\Omega}) : \chi\!\mid_{\sigmaO^m}
\in \mathcal{P}_k(\sigmaO^m) \ \forall\ \sigmaO^m \in {\cal T}^m\}\,,
\quad k \in \mathbb{N}\,,
\end{equation*}
where $\mathcal{P}_k(\sigmaO^m)$ denotes the space of polynomials of degree $k$
on $\sigmaO^m$. Moreover, $S^m_0$ is the space of piecewise constant functions
on ${\cal T}^m$ and let $\vec I^m_k$ be the standard interpolation operator
onto $[S^m_k]^d$.

Let $\uspacedisc{g}{m}\subset\uspacesimple(\vec I_k^m\vec g)$ and
$\pspace^m\subset\pspace$ be the finite element spaces we use for the
approximation of velocity and pressure, and let $\pnormspace^m:= \pspace^m \cap
\pnormspace$. The spaces $(\uspacedisc{0}{m},\pspace^m)$ satisfy the LBB
inf-sup. See (\ref{eq:LBB}) for more details.

We consider a fitted finite element approximation for the evolution of the
interface $\Gamma(t)$. Let $\Gamma^m\subset\R^d$ be a $(d-1)$-dimensional
polyhedral surface approximating the closed surface $\Gamma(t_m)$, $m=0
,\ldots, M$. Let $\Omega^m_+$ denote the exterior of $\Gamma^m$ and let
$\Omega^m_-$ be the interior of $\Gamma^m$, where we assume that $\Gamma^m$ has
no self-intersections. Then $\Omega = \Omega_-^m \cup \Gamma^m \cup
\Omega_+^m$, and the fitted nature of
our method implies that
\begin{equation*}
\overline{\Omega^m_+} = \bigcup_{o \in \mathcal{T}^m_+} \overline{o}
\quad\text{and}\quad
\overline{\Omega^m_-} = \bigcup_{o \in \mathcal{T}^m_-} \overline{o} \,,
\end{equation*}
where $\mathcal{T}^m = \mathcal{T}^m_+ \cup \mathcal{T}^m_-$ and
$\mathcal{T}^m_+ \cap \mathcal{T}^m_- = \emptyset$. Let $\vec \nu^m$ denote the
piecewise constant unit normal to $\Gamma^m$ such that $\vec\nu^m$ points into
$\Omega^m_+$.

In order to define the parametric finite element spaces on $\Gamma^m$, we
proceed analogously to the mean curvature flow and surface diffusion problems,
see \S\ref{sec:geometric_pdes_fem}. Therefore we assume that
$\Gamma^m=\bigcup_{j=1}^{J_\Gamma} \overline{\sigma^m_j}$, where
$\{\sigma^m_j\}_{j=1}^{J_\Gamma}$ is a family of mutually disjoint open
$(d-1)$-simplices with vertices $\{\vec q^m_k\}_{k=1}^{K_\Gamma}$. Then
we define $\Vh := \{\vec\chi \in [C(\Gamma^m)]^d:\vec\chi\!\mid_{\sigma^m_j}
\in \mathcal{P}_1(\sigma^m_j), j=1,\ldots, J_\Gamma\} =: [\Wh]^d$, where $\Wh
\subset H^1(\Gamma^m)$ is the space of scalar continuous piecewise linear
functions on $\Gamma^m$, with $\{\chi^m_k\}_{k=1}^{K_\Gamma}$ denoting the
standard basis of $\Wh$. As usual, we parametrize the new surface
$\Gamma^{m+1}$ over $\Gamma^m$ using a parametrization $\vec X^{m+1} \in \Vh$,
so that $\Gamma^{m+1} = \vec X^{m+1}(\Gamma^m)$. Finally, let
$\langle\cdot,\cdot\rangle_{\Gamma^m}^h$ be the mass lumped inner product on
$\Gamma^m$, see (\ref{eq:masslump}), and let
$\langle\cdot,\cdot\rangle_{\Gamma^m}$ denote the standard $L^2$--inner product
on $\Gamma^m$.

We need to pay particular attention to the discretization of the time
derivative in (\ref{eq:ns_weaka_antisym}). Indeed, the discrete counterpart of
$\tfrac{1}{2} \ddt (\rho\,\vec u, \vec \xi)$ is
\begin{equation}\label{eq:antysim_time_derivative_a}
\frac{1}{2}\left(\frac{\rho^m\,\vec U^{m+1} - \vec I^m_0\,\rho^{m-1}
\vec I^m_2\,\vec U^m}{\tau}, \vec \xi \right)\,,
\end{equation}
while the discrete version of $\tfrac{1}{2}(\rho\,\vec u_t, \vec \xi)$ is
\begin{equation}\label{eq:antysim_time_derivative_b}
\frac{1}{2}\left( \vec I^m_0\,\rho^{m-1}\, \frac{\vec U^{m+1} - \vec I^m_2\,
\vec U^m}{\tau}, \vec \xi \right)\,.
\end{equation}
We notice that in (\ref{eq:antysim_time_derivative_b}), instead of using the
actual density $\rho^m$, we use the density at the previous time step
$\rho^{m-1}$ and we interpolate it on the current bulk grid $\Omega_\pm^m$. This
is needed to prove stability over one time step. Since
$\rho^m = \rho_+\,\charfcn{\Omega^m_+} + \rho_-\,\charfcn{\Omega^m_-}\in
S^m_0$, we can rewrite the discretization of the weak time derivative
$\tfrac{1}{2}\big[ \ddt (\rho\,\vec u, \vec \xi)+(\rho\,\vec u_t, \vec\xi)\big]$
in the simpler form
\begin{equation}
\left(\frac{\tfrac{1}{2}(\vec I^m_0\,\rho^{m-1}+\rho^m)\,\vec U^{m+1} -
\vec I^m_0\,\rho^{m-1} \vec I^m_2\,\vec U^m}{\tau}, \vec \xi \right)\,.
\end{equation}

Then our antisymmetric finite element approximation is given as follows. Let
$\Gamma^0$, an approximation to $\Gamma(0)$, and $\vec U^0\in
\uspacedisc{g}{0}$ be given. For $m=0,\ldots, M-1$, find $(\vec U^{m+1},P^{m+1},
\vec X^{m+1}, \kappa^{m+1}) \in \uspacedisc{g}{m}\times \pnormspace^m \times
\Vh \times \Wh$ such that
\begin{subequations}
\begin{align}
& \left(\frac{\tfrac{1}{2}(\vec I^m_0\,\rho^{m-1}+\rho^m)\,\vec U^{m+1} -
\vec I^m_0\,\rho^{m-1} \vec I^m_2\,\vec U^m}{\tau}, \vec \xi \right)\nonumber \\
& \qquad + \frac{1}{2}\left((\rho^m\,\vec I^m_2\,\vec U^m\,.\,\nabla)\,
\vec U^{m+1}\,,\,\vec \xi \right) - \frac{1}{2} \left((\rho^m\,
\vec I^m_2 \, \vec U^m\,.\,\nabla)\,\vec \xi\,,\,\vec U^{m+1} \right)
\nonumber \\
& \qquad - 2\left(\mu^m\,\mat D(\vec U^{m+1}), \mat D(\vec \xi) \right)
- \left(P^{m+1}, \nabla\,.\,\vec \xi\right) \nonumber \\
& \qquad - \gamma\,\left\langle
\kappa^{m+1}\,\vec\nu^m,\vec\xi\right\rangle_{\Gamma^m}
= \left(\vec f^{m+1}, \vec \xi\right)  \quad \forall\ \vec\xi \in
\uspacedisc{0}{m}\,,\label{eq:ns_HGa_antisym} \\
& \left(\nabla\,.\,\vec U^{m+1}, \varphi\right)  = 0
\quad \forall\ \varphi \in \pnormspace^m\,,\label{eq:ns_HGb_antisym} \\
&  \left\langle \frac{\vec X^{m+1} - \vec\id}{\tau} ,\chi\,\vec\nu^m
\right\rangle_{\Gamma^m}^h - \left\langle \vec U^{m+1}, \chi\,\vec\nu^m
\right\rangle_{\Gamma^m}  = 0 \quad \forall\ \chi \in \Wh\,,
\label{eq:ns_HGc_antisym}\\
& \left\langle \kappa^{m+1}\,\vec\nu^m, \vec\eta \right\rangle_{\Gamma^m}^h
+ \left\langle \nabs\,\vec X^{m+1}, \nabs\,\vec \eta \right\rangle_{\Gamma^m} =
0 \quad \forall\ \vec\eta \in \Vh \label{eq:ns_HGd_antisym}
\end{align}
\end{subequations}
and set $\Gamma^{m+1} = \vec X^{m+1}(\Gamma^m)$.  Here we have defined
$\vec f^{m+1}(\cdot) := \vec I^m_2\,\vec f(\cdot,t_{m+1})$,
\begin{equation}
\mu^m = \mu_+\,\charfcn{\Omega^m_+} + \mu_-\,\charfcn{\Omega^m_-}\in S^m_0
\end{equation}
and
\begin{equation}
\rho^m = \rho_+\,\charfcn{\Omega^m_+} + \rho_-\,\charfcn{\Omega^m_-}\in S^m_0\,.
\end{equation}

\section{Existence, uniqueness and stability of a discrete solution}
\label{sec:ns_existence}
TODO: Discuss interpolation of $\vec U^m$, stress difference to unfitted case,
where stability if no coarsening in bulk mesh. Describe strategies for
interpolation. State that stability holds only without remeshing
(interpolation!)

\section{Standard finite element approximation}\label{sec:ns_fem}
An alternative to the antisymmetric discrete finite element approximation
(\ref{eq:ns_HGa_antisym}--d) can be obtained using the standard weak
formulation (\ref{eq:ns_weaka}--d). We use the same hypothesis and
definitions employed in \S\ref{sec:ns_fem_antisym}. Then only difference is
that we consider the more general partitioning  $0= t_0 < t_1 < \ldots <
t_{M-1} < t_M = T$ of $[0,T]$ into possibly variable time steps $\tau_m :=
t_{m+1}-t_m$, $m=0 ,\ldots, M-1$.

Then our explicit standard finite element approximation, which is based on the
variational formulation (\ref{eq:ns_weaka}--d), is given as follows. Let
$\Gamma^0$, an approximation to $\Gamma(0)$, and $\vec U^0\in \uspacedisc{g}{0}$
be given. For $m=0,\ldots, M-1$, find $(\vec U^{m+1},P^{m+1}, \vec X^{m+1},
\kappa^{m+1}) \in \uspacedisc{g}{m}\times \pnormspace^m \times \Vh \times \Wh$
such that
\begin{subequations}
\begin{align}
& \left( \rho^m\,\frac{\vec U^{m+1} - \vec I^m_2\,\vec U^m}{\tau_m}, \vec
\xi \right) + \left(\rho^m(\vec I^m_2\vec U^m\,.\,\nabla)\,\vec U^{m+1}\,,
\,\vec \xi \right)\nonumber \\
& \qquad - 2\left(\mu^m\,\mat D(\vec U^{m+1}), \mat D(\vec \xi) \right)
- \left(P^{m+1}, \nabla\,.\,\vec \xi\right) \nonumber \\
& \qquad - \gamma\,\left\langle \kappa^{m+1}\,\vec\nu^m ,
\vec\xi\right\rangle_{\Gamma^m}
= \left(\vec f^{m+1}, \vec \xi\right)  \quad \forall\ \vec\xi \in
\uspacedisc{0}{m}\,, \label{eq:ns_HGa} \\
& \left(\nabla\,.\,\vec U^{m+1}, \varphi\right)  = 0
\quad \forall\ \varphi \in \pnormspace^m\,,\label{eq:ns_HGb} \\
&  \left\langle \frac{\vec X^{m+1} - \vec\id}{\tau_m} ,\chi\,\vec\nu^m
\right\rangle_{\Gamma^m}^h - \left\langle \vec U^{m+1}, \chi\,\vec\nu^m
\right\rangle_{\Gamma^m}  = 0 \quad \forall\ \chi \in \Wh\,, \label{eq:ns_HGc}\\
& \left\langle \kappa^{m+1}\,\vec\nu^m, \vec\eta \right\rangle_{\Gamma^m}^h
+ \left\langle \nabs\,\vec X^{m+1}, \nabs\,\vec \eta \right\rangle_{\Gamma^m} =
0 \quad \forall\ \vec\eta \in \Vh \label{eq:ns_HGd}
\end{align}
\end{subequations}
and set $\Gamma^{m+1} = \vec X^{m+1}(\Gamma^m)$. Again, we have defined $\vec
f^{m+1}(\cdot) := \vec I^m_2\,\vec f(\cdot,t_{m+1})$,
\begin{equation}
\mu^m = \mu_+\,\charfcn{\Omega^m_+} + \mu_-\,\charfcn{\Omega^m_-}\in S^m_0
\end{equation}
and
\begin{equation}
\rho^m = \rho_+\,\charfcn{\Omega^m_+} + \rho_-\,\charfcn{\Omega^m_-}\in S^m_0\,.
\end{equation}

Since the convective term is treated explicitly, the scheme
(\ref{eq:ns_HGa}--d) is a linear scheme that leads to a coupled linear system of
equations for the unknowns $(\vec U^{m+1}, P^{m+1}, \vec X^{m+1}, \kappa^{m+1})$
at each time level.

We notice that the antisymmetric scheme (\ref{eq:ns_HGa_antisym}--d) differs
from the explicit standard scheme (\ref{eq:ns_HGa}--d), for the antisymmetric
rewrite of the convective term and for the density coefficients in the time
derivative. Indeed, instead of using the actual density $\rho^m$, it uses an
average between the actual density and the interpolation of the density at the
previous time step $\tfrac{1}{2}(\vec I^m_0\,\rho^{m-1} + \rho^m)$ for what
concern $\vec U^{m+1}$ and it uses directly the density interpolation
$\vec I^m_0\,\rho^{m-1}$ for $\vec I^m_2\vec U^m$.

Alternatively, the convective term in (\ref{eq:ns_HGimpa}) can be treated
implicitly and the finite element approximation (\ref{eq:ns_HGa}--d) becomes
\begin{subequations}
\begin{align}
& \left( \rho^m\,\frac{\vec U^{m+1} - \vec I^m_2\,\vec U^m}{\tau_m}, \vec
\xi \right) + \left(\rho^m( \vec U^{m+1}\,.\,\nabla)\,\vec U^{m+1}\,,
\,\vec \xi \right)\nonumber \\
& \qquad- 2\left(\mu^m\,\mat D(\vec U^{m+1}), \mat D(\vec \xi) \right)
- \left(P^{m+1}, \nabla\,.\,\vec \xi\right) \nonumber \\
& \qquad - \gamma\,\left\langle \kappa^{m+1}\,\vec\nu^m ,
\vec\xi\right\rangle_{\Gamma^m}
= \left(\vec f^{m+1}, \vec \xi\right)  \quad \forall\ \vec\xi \in
\uspacedisc{0}{m}\,, \label{eq:ns_HGimpa} \\
& \left(\nabla\,.\,\vec U^{m+1}, \varphi\right)  = 0
\quad \forall\ \varphi \in \pnormspace^m\,,\label{eq:ns_HGimpb} \\
&  \left\langle \frac{\vec X^{m+1} - \vec\id}{\tau_m} ,\chi\,\vec\nu^m
\right\rangle_{\Gamma^m}^h - \left\langle \vec U^{m+1}, \chi\,\vec\nu^m
\right\rangle_{\Gamma^m}  = 0 \quad \forall\ \chi \in \Wh\,,
\label{eq:ns_HGimpc}\\
& \left\langle \kappa^{m+1}\,\vec\nu^m, \vec\eta \right\rangle_{\Gamma^m}^h
+ \left\langle \nabs\,\vec X^{m+1}, \nabs\,\vec \eta \right\rangle_{\Gamma^m} =
0 \quad \forall\ \vec\eta \in \Vh\,. \label{eq:ns_HGimpd}
\end{align}
\end{subequations}
The scheme (\ref{eq:ns_HGimpa}--d) is a nonlinear scheme that leads to a coupled
nonlinear system of equations for the unknowns $(\vec U^{m+1}, P^{m+1}, \vec
X^{m+1}, \kappa^{m+1})$ at each time level.

\section{ALE weak formulation}\label{sec:ns_weak_ale}
The system of governing equations (\ref{eq:ns_full_momentum}--i) is expressed in
term of Eulerian coordinates $\vec z$. However, it is possible to rewrite the
velocity time derivative $\vec u_t$ with respect to the so called Arbitrary
Lagrangian Eulerian (ALE) coordinate $\vec q$.

Analogously to what is done in \S\ref{sec:front_tracking_approach} to
parametrize the interface $\Gamma(t)$, we can define the fixed reference
manifold $\D$ and extend the map (\ref{eq:interface_parametrization}) to
parametrize $\Omega$ as $\Omega=\vec x(\D,t)$. This extended map obviously
still satisfy $\Gamma(t) = \vec x(\Upsilon,t)$, given that $\Upsilon\subset\D$.

Now, let $h:\Omega\times [0,T]\to \R$ be a function defined on the Eulerian
frame, the corresponding function on the ALE frame $\hat h$ is defined as
\begin{equation}\label{eq:h}
\hat h:\D\times [0,T]\to \R,\quad \hat h(\vec q,t)=h(\vec x(\vec q,t),t).
\end{equation}
In order to compute the time derivative of (\ref{eq:h}) with respect to the ALE
frame, using the chain rule, we have
\begin{equation}
\frac{\partial\hat h(\vec q,t)}{\partial t}=\frac{\partial h(\vec x(\vec
q,t),t)}{\partial t}=\frac{\partial h(\vec z,t)}{\partial t}+\vec x_t(\vec q,t)
\cdot \nabla h(\vec z,t),
\end{equation}
therefore it holds that
\begin{equation}
\frac{\partial h(\vec z,t)}{\partial t} =
\frac{\partial\hat h(\vec q,t)}{\partial t}-
\vec x_t(\vec q,t) \cdot \nabla h(\vec z,t).
\end{equation}
Finally, introducing the domain velocity
\begin{equation} \label{eq:W}
\W(\vec z, t) := \vec x_t(\vec q, t) \quad \forall\ \vec z = \vec x(\vec q,t)
\in \Omega
\end{equation}
and the time derivative in the ALE frame
\begin{equation} \label{eq:ale_derivative}
\left.\frac{\partial h(\vec z,t)}{\partial t}\right|_{\D}:=
\frac{\partial\hat h(\vec q,t)}{\partial t} \quad
\forall\ \vec z = \vec x(\vec q,t) \in \Omega
\end{equation}
we obtain
\begin{equation}\label{eq:ale_dt}
h_t =\left.h_t\right|_{\D} -\vec{\mathcal{W}} \cdot \nabla h.
\end{equation}
The identity (\ref{eq:ale_dt}) is naturally extended to vector valued functions.
We stress the fact that the domain velocity $\W$ for the interface points is
consistent with the interface velocity $\V$. Therefore it holds
\begin{equation}
\left.\W \right|_{\Gamma(t)}=\V\,.
\end{equation}

Using (\ref{eq:ale_dt}) in (\ref{eq:ns_full_momentum}), we can rewrite the
momentum equation in the ALE frame as:
\begin{align}
\rho\,(\left.\vec u_t\right|_{\D} +(\vec u - \W)\,.\, \nabla \vec u)\,
-2 \mu\,\nabla\,.\, \mat D(\vec u)+ \nabla\,p & = \vec f
\quad &&\mbox{in } \Omega_\pm(t)\,. \label{eq:ns_full_momentum_ale} \\
\end{align}
We can notice that, with respect to the original formulation, there is a
convective-type term due to the domain movement and the time derivative is
computed in the fixed reference frame $\D$. Obviously, if the domain is fixed,
this additional convective term is zero and the time derivative in the ALE
frame coincide with the usual time derivative in the Eulerian frame. Therefore
$\W=\vec 0$ corresponds to a pure Eulerian method, while $\W=\vec u$ corresponds
to a fully Lagrangian scheme.

In order to define the ALE weak formulation, we need to use a different
functional setting for the test functions with respect to the one used for the
standard weak formulation and antisymmetric weak formulation, see
\S\ref{sec:ns_weak}. More precisely we consider the admissible spaces of test
functions on the reference domain
\begin{align*}
\widehat{\uspace{b}} &:= \{\vec \phi\in[H^1_0(\D)]^d:\vec \phi =\vec
g\quad \mbox{on }\partial_1\D\}\,,\\
\hat{\pspace} &:= L^2(\D)\,,\\
\widetilde{\hat\pspace} &:= \{\eta\in\hat\pspace : \int_\D\eta\dL{d}=0 \}\,,
\end{align*}
for a given $\vec b \in [H^1_0(\D)]^d$. We notice that the boundary
of $\Omega$ is not time dependent which implies that
$\partial_1\Omega=\partial_1\D$.

Hence, the ALE weak formulation of (\ref{eq:ns_full_momentum_ale},
\ref{eq:ns_full_mass}--i) is given as follows. Given $\Gamma(0) = \Gamma_0$ and
$\vec u = \vec u_0$, for almost all $t\in(0,T)$ find $\Gamma(t)$ and ${(\vec u,
p, \varkappa)\in \uspace g \times \pnormspace \times H^1(\Gamma(t))}$
such that
\begin{subequations}
\begin{align}
& \left(\rho\,\left.\vec u_t\right|_{\D}, \vec \xi\right) +
\left(\rho\,(\vec u - \W)\,.\, \nabla \vec u,\vec \xi\right)
+ 2\left(\mu\,\mat D(\vec u), \mat D(\vec \xi)\right) \nonumber \\
& - \left(p, \nabla\,.\,\vec \xi\right)
- \gamma\,\left\langle \varkappa\,\vec\nu, \vec\xi\right\rangle_{\Gamma(t)}
= \left(\vec f, \vec \xi\right)\quad \forall\ \vec\xi \in \uspace 0 \,,
\label{eq:ns_weaka_ale}\\
& \left(\nabla\,.\,\vec u, \varphi\right) = 0
\quad \forall\ \varphi \in \pnormspace\,, \label{eq:ns_weakb_ale} \\
&  \left\langle \V
- \vec u, \chi\,\vec\nu \right\rangle_{\Gamma(t)} = 0
\quad \forall\ \chi \in H^1(\Gamma(t))\,, \label{eq:ns_weakc_ale} \\
& \left\langle \varkappa\,\vec\nu, \vec\eta \right\rangle_{\Gamma(t)}
+ \left\langle \nabs\,\vec \id, \nabs\,\vec \eta \right\rangle_{\Gamma(t)}
= 0  \quad \forall\ \vec\eta \in [H^1(\Gamma(t))]^d\,\label{eq:ns_weakd_ale}
\end{align}
\end{subequations}
holds for almost all times $t \in (0,T]$.

\section{ALE finite element approximation}\label{sec:ns_fem_ale}

\section{Solution method}\label{sec:ns_solution_method}
The linear algebraic system arising from (\ref{eq:ns_HGa}--d) and
(\ref{eq:ns_HGa_antisym}--d) is identical to the two-phase Stokes
algebraic system (\ref{eq:stokes_algebraic}) with the only modification in
the terms $\vec B_{\Omega}$ and $\vec c$. Indeed, for explicit standard scheme
(\ref{eq:ns_HGa}--d) these terms become
\begin{align*}
[\vec B_\Omega]_{ij} := & \left( \frac{\rho^m}{\tau_m} \phi_j^{\uspacesimple^m},
\phi_i^{\uspacesimple^m} \right)\mat \id
+ 2\left(\left(\mu^m\,\mat D(\phi_j^{\uspacesimple^m} \vec e_q),
\mat D(\phi_i^{\uspacesimple^m}\,\vec e_r) \right) \right)_{q,r=1}^d\\
& + \left( \left(\rho^m \vec I_2^m\vec U^m\,.\,\nabla \right)
\phi_j^{\uspacesimple^m},
\phi_i^{\uspacesimple^m} \right)\mat \id \,,\\
\vec c_i := &\left( \frac{\rho^m}{\tau_m}\vec I_2^m\vec U^m +
\vec f^{m+1},\phi_i^{\uspacesimple^m}\right)\,,
\end{align*}
while for the antisymmetric scheme (\ref{eq:ns_HGa_antisym}--d) these terms
become
\begin{align*}
[\vec B_\Omega]_{ij} := & \left( \frac{\rho^m}{\tau_m} \phi_j^{\uspacesimple^m},
\phi_i^{\uspacesimple^m} \right)\mat \id
+ 2\left(\left(\mu^m\,\mat D(\phi_j^{\uspacesimple^m} \vec e_q),
\mat D(\phi_i^{\uspacesimple^m}\,\vec e_r) \right) \right)_{q,r=1}^d\\
& + \frac{1}{2}\left( \left(\rho^m \vec I_2^m\vec U^m\,.\,\nabla \right)
\phi_j^{\uspacesimple^m}, \phi_i^{\uspacesimple^m} \right)\mat \id
-\frac{1}{2}\left( \left(\rho^m \vec I_2^m\vec U^m\,.\,\nabla \right)
\phi_i^{\uspacesimple^m}, \phi_j^{\uspacesimple^m} \right)\mat \id \,,\\
\vec c_i := &\left( \frac{\vec I_0^m\rho^{m-1}+\rho^m}{2\tau_m}\vec I_2^m
\vec U^m + \vec f^{m+1},\phi_i^{\uspacesimple^m}\right)\,.
\end{align*}
Then the resulting algebraic system can be solved using the same technique
introduced in \S\ref{sec:stokes_solution_method}.

Instead, in order to solve the implicit standard scheme (\ref{eq:ns_HGimpa}--d)
we use a fixed point iteration. Therefore we can reformulate the scheme
(\ref{eq:ns_HGimpa}--d) as follow. Let $\Gamma^0$, an approximation to
$\Gamma(0)$, and $\vec U^0\in \uspacedisc{g}{0}$ be given. For $m=0,\ldots,
M-1$, set $\vec U^{m+1,0}=\vec I^m_2\,\vec U^m$, $s=0$ and $\epsilon_f
\geq 0$. Then, find $(\vec U^{m+1,s+1},P^{m+1}, \vec X^{m+1}, \kappa^{m+1}) \in
\uspacedisc{g}{m}\times \pnormspace^m \times \Vh \times \Wh$ such that
\begin{subequations}
\begin{align}
& \left( \rho^m\,\frac{\vec U^{m+1,s+1} - \vec I^m_2\,\vec U^m}{\tau_m}, \vec
\xi \right) + \left((\rho^m\vec U^{m+1,s}\,.\,\nabla)\,\vec U^{m+1,s+1}\,,
\,\vec \xi \right)\nonumber \\
& - 2\left(\mu^m\,\mat D(\vec U^{m+1,s+1}), \mat D(\vec \xi) \right)
- \left(P^{m+1}, \nabla\,.\,\vec \xi\right) \nonumber \\
& - \gamma\,\left\langle \kappa^{m+1}\,\vec\nu^m,\vec\xi\right\rangle_{\Gamma^m}
= \left(\vec f^{m+1}, \vec \xi\right)  \quad \forall\ \vec\xi \in
\uspacedisc{0}{m}\,,
\label{eq:ns_HGfixedpointa} \\
& \left(\nabla\,.\,\vec U^{m+1,s+1}, \varphi\right)  = 0
\quad \forall\ \varphi \in \pnormspace^m\,,\label{eq:ns_HGfixedpointb} \\
&  \left\langle \frac{\vec X^{m+1} - \vec\id}{\tau_m} ,\chi\,\vec\nu^m
\right\rangle_{\Gamma^m}^h - \left\langle \vec U^{m+1,s+1}, \chi\,\vec\nu^m
\right\rangle_{\Gamma^m}  = 0 \quad \forall\ \chi \in \Wh\,,
\label{eq:ns_HGfixedpointc}\\
& \left\langle \kappa^{m+1}\,\vec\nu^m, \vec\eta \right\rangle_{\Gamma^m}^h
+ \left\langle \nabs\,\vec X^{m+1}, \nabs\,\vec \eta \right\rangle_{\Gamma^m} =
0 \quad \forall\ \vec\eta \in \Vh\,, \label{eq:ns_HGfixedpointd}
\end{align}
\end{subequations}
and increment $s$, until $\|U^{m+1,s+1}-U^{m+1,s}\|_{L^\infty} \leq\epsilon_f$.
Finally, set ${\Gamma^{m+1} = \vec X^{m+1}(\Gamma^m)}$ and $\vec U^{m+1}= \vec
U^{m+1,s+1}$. The scheme (\ref{eq:ns_HGfixedpointa}--d), at each time
level $m$, is a fixed point iteration which consists of solving a coupled
linear system of equations for the unknowns $(\vec U^{m+1,s+1}, P^{m+1},
\vec X^{m+1}, \kappa^{m+1})$ at each step $s$ until $L^\infty$--error of the
velocity $\|U^{m+1,s+1}-U^{m+1,s} \|_{L^\infty}$ is smaller than the required
tolerance $\epsilon_f$. Obviously, in the case that the fixed point iteration
performs only one step, (\ref{eq:ns_HGfixedpointa}--d) reduces to the explicit
standard scheme (\ref{eq:ns_HGa}--d). Again, at every step, the linear algebraic
system arising from (\ref{eq:ns_HGfixedpointa}--d) is identical to the two-phase
Stokes algebraic system (\ref{eq:stokes_algebraic}) with the only modification
in the terms $\vec B_{\Omega}$ and $\vec c$ which become
\begin{align*}
[\vec B_\Omega]_{ij} := & \left( \frac{\rho^m}{\tau_m} \phi_j^{\uspacesimple^m},
\phi_i^{\uspacesimple^m} \right)\mat \id
+ 2\left(\left(\mu^m\,\mat D(\phi_j^{\uspacesimple^m} \vec e_q),
\mat D(\phi_i^{\uspacesimple^m}\,\vec e_r) \right) \right)_{q,r=1}^d\\
& + \left( \left(\rho^m \vec U^{m+1,s}\,.\,\nabla \right)
\phi_j^{\uspacesimple^m},
\phi_i^{\uspacesimple^m} \right)\mat \id \,,\\
\vec c_i := &\left( \frac{\rho^m}{\tau_m}\vec I_2^m\vec U^m +
\vec f^{m+1},\phi_i^{\uspacesimple^m}\right)\,.
\end{align*}
