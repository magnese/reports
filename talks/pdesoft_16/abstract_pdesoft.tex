\documentclass{article}
\usepackage[utf8]{inputenc}
\usepackage{amsmath,amsfonts,amssymb}
\usepackage{graphicx}
\usepackage{times}
\parindent0pt
\parskip 1.5ex plus 1ex minus .5ex
%%%%%%%%%%%%%%%%%%%%%%%%%%%%%%%%%%%%%%%%%%%%%%%%%%%%%%%%%%%%%%%%%%%%%%
% Do not change above this line! Do not add packages! You may comment
% out packages missing in your installation.
%
% Everything before "\begin{document}" will be ignored! Do not define
%   macros!
%%%%%%%%%%%%%%%%%%%%%%%%%%%%%%%%%%%%%%%%%%%%%%%%%%%%%%%%%%%%%%%%%%%%%%
\begin{document}
\title{FEM approximation of Two-Phase Navier--Stokes Flow using DUNE-FEM}
\author{%
  Marco Agnese\thanks{Imperial College London,
    \texttt{m.agnese13@imperial.ac.uk}}
}
\maketitle

We propose a novel fitted finite element method for two-phase Navier--Stokes
flow problems that uses piecewise linear finite elements to approximate the
moving interface. The meshes describing the discrete interface in general do
not deteriorate in time, which means that in numerical simulations a smoothing
or a remeshing of the interface mesh is not necessary. We present several
numerical experiments for our numerical method, which demonstrate the accuracy
and robustness of the proposed algorithm. The discretization is implemented
within the DUNE-FEM framework.

Link: \emph{http://www2.warwick.ac.uk/fac/sci/maths/research/events/2015-16/
nonsymposium/pde}

\begin{thebibliography}{10}

\bibitem{Paper}
M.~Agnese, R.~N\"urnberg, Internat. J. Numer. Methods Fluids,
\emph{ Fitted finite element discretization of two--phase Stokes flow},(2016),
to appear.
\end{thebibliography}

\end{document}
