\documentclass[a4paper,12pt,onecolumn]{article}

\usepackage[utf8]{inputenc}
\usepackage[english]{babel}
\usepackage{amssymb,amsmath,amsthm,array,epsfig,a4,verbatim,pstricks,url,subfig,
hyperref}

\newcommand{\R}{{\mathbb R}}
\newcommand{\spa}{\operatorname{span}}
\newcommand{\vol}{\mathcal{L}^d}
\newcommand{\dH}[1]{\;{\rm d}{\cal H}^{#1}} % Hausdorff measure
\newcommand{\dL}[1]{\;{\rm d}{\cal L}^{#1}} % Lebesgue measure
\newcommand{\bigchi}{\ensuremath{\mathrm{\mathcal{X}}}}
\newcommand{\charfcn}[1]{\bigchi_{#1}} % characteristic function
\newcommand{\Vh}{\underline{V}(\Gamma^m)}
\newcommand{\Wh}{W(\Gamma^m)}
\newcommand{\Vht}{\underline{V}(\Gamma^h(t))}
\newcommand{\Wht}{W(\Gamma^h(t))}
\newcommand{\uspace}{\mathbb{U}}
\newcommand{\pspace}{\mathbb{P}}
\newcommand{\kspace}{\mathbb{K}}
\newcommand{\xspace}{\mathbb{X}}
\newcommand{\sigmaO}{o}
\newcommand{\nabs}{\nabla_{\!s}}
\newcommand{\Id}{I\!d}
\newcommand{\id}{\rm id}
\newcommand{\ddt}{\frac{\rm d}{{\rm d}t}}
\newcommand{\NbulkT}{\vec{N}_{\Gamma,\Omega}^T}
\newcommand{\Nbulk}{\vec{N}_{\Gamma,\Omega}}
\newcommand{\errorXx}{\|\Gamma^h - \Gamma\|_{L^\infty}}
\newcommand{\LerrorUu}[1]{\|\vec U - I^h_{#1}\,\vec u\|_{L^2(\Omega_T)}}
\newcommand{\errorUu}[1]{\|\vec U - I^h_{#1}\,\vec u\|_{L^\infty}}
\newcommand{\errorPp}[1]{\|P - I^h_{#1}\,p\|_{L^\infty}}
\newcommand{\LerrorPp}{\|P - p\|_{L^2(\Omega_T)}}
\newcommand{\unitn}{\vec{\rm n}}
\newcommand{\mat}[1]{\underline{\underline{#1}}\rule{0pt}{0pt}}

\renewcommand{\theequation}{\arabic{section}.\arabic{equation}}

\title{Preconditioners}
\author{Marco Agnese \and Robert N\"urnberg}

\begin{document}

\maketitle

\section{Algebraic system}\label{sec:algebraic_system}
\setcounter{equation} 0

\textcolor{magenta}{TODO: I assemble both $\vec C_\Omega$ and $\vec C^T_\Omega$ 
in my code. Maybe assemble only once but pay attention to Dirichlet BC!}

Find $(\vec U^{m+1},P^{m+1}, \kappa^{m+1},\delta\vec{X}^{m+1})\in
(\R^d)^{K^m_\uspace}\times \R^{K^m_\pspace} \times \R^{K_\Gamma}\,\times
(\R^d)^{K_\Gamma}$, where $\vec X^{m+1} = \vec X^m+ \delta\vec X^{m+1}$, such
that
\begin{equation}
\begin{pmatrix}
\vec B_\Omega & \vec C_\Omega & -\gamma\,\Nbulk & 0 \\
\vec C^T_\Omega & 0 & 0 & 0 \\
\NbulkT & 0 & 0 & -\frac1{\tau_m}\,\vec{N}_\Gamma^T \\
0 & 0 & \vec{N}_\Gamma & \vec{A}_\Gamma
\end{pmatrix}
\begin{pmatrix}
\vec U^{m+1} \\
P^{m+1} \\
\kappa^{m+1} \\
\delta\vec{X}^{m+1}
\end{pmatrix}
=
\begin{pmatrix}
\vec g \\
0 \\
0 \\
-\vec{A}_\Gamma\,\vec{X}^{m}
\end{pmatrix} \,,
\label{eq:lin}
\end{equation}
where $(\vec U^{m+1},P^{m+1},\kappa^{m+1},\delta\vec{X}^{m+1})$ here denote the
coefficients of these finite element functions with respect to the standard
bases of $\uspace^m$, $\pspace^m$, $\Wh$ and $\Vh$, respectively. Moreover,
$\vec X^m$ denotes the coefficients of $\vec\id\!\mid_{\Gamma^m}$ with respect
to the basis of $\Vh$.

\section{Schur complement approach}\label{sec:shur}
\setcounter{equation} 0

For the the solution of (\ref{eq:lin}) we use a Schur complement approach that
eliminates $(\kappa^{m+1}, \delta \vec X^{m+1})$ from (\ref{eq:lin}), and then
use an iterative solver for the remaining system in $(\vec U^{m+1}, P^{m+1})$.
This approach has the advantage that for the reduced system well-known solution
methods for finite element discretizations for the standard Stokes equations may
be employed. The desired Schur complement approach for eliminating
$(\kappa^{m+1},\delta \vec X^{m+1})$ from (\ref{eq:lin}) can be obtained as
follows. Let
\begin{equation} \label{eq:Xi}
\Xi_\Gamma:= 
\begin{pmatrix}
 0 & - \frac1{\tau_m}\,\vec{N}_\Gamma^T \\
\vec{N}_\Gamma & \vec{A}_\Gamma
\end{pmatrix} \,.
\end{equation}
Then (\ref{eq:lin}) can be reduced to
\begin{subequations}
\begin{equation} \label{eq:SchurkX}
\begin{pmatrix}
\vec B_\Omega + \gamma\,(\Nbulk \ 0)\,\Xi_\Gamma^{-1}\,
\binom{\NbulkT}{0} & \vec C_\Omega \\
\vec C_\Omega^T & 0
\end{pmatrix}
\begin{pmatrix}
\vec U^{m+1} \\ 
P^{m+1}
\end{pmatrix}
= 
\begin{pmatrix}
\vec g
-\gamma\,(\Nbulk \ 0)\, \Xi_\Gamma^{-1}\,
\binom{0}{\vec{A}_\Gamma\,\vec{X}^{m}} \\
0
\end{pmatrix}
\end{equation}
and
\begin{equation} \label{eq:SchurkXb}
\binom{\kappa^{m+1}}{\delta\vec{X}^{m+1}} = \Xi_\Gamma^{-1}\,
\binom{-\NbulkT\,\vec U^{m+1}}{-\vec{A}_\Gamma\,\vec{X}^{m}}\,.
\end{equation}
\end{subequations}

We notice that when $\gamma = 0$ the linear system (\ref{eq:SchurkX}) becomes
\begin{equation} \label{eq:stokes_system}
\begin{pmatrix}
\vec B_\Omega & \vec C_\Omega \\
\vec C_\Omega^T & 0
\end{pmatrix}
\begin{pmatrix}
\vec U^{m+1} \\ 
P^{m+1}
\end{pmatrix}
= 
\begin{pmatrix}
\vec g \\
0
\end{pmatrix}\,,
\end{equation}
which for $\mu_+=\mu_-$ corresponds to a discretization of the one--phase
Navier--Stokes problem.

Finally we observe that, in order to obtain the curvature of a curve, it is 
sufficient to solve the system
\begin{equation}
\Xi_\Gamma\,\binom{\kappa^{m+1}}{\delta\vec{X}^{m+1}} = 
\binom{0}{\vec{A}_\Gamma\,\vec{X}^{m}}
\end{equation}
which can be rewritten as
\begin{equation}
\vec{N}_\Gamma\,\kappa^{m+1} = \vec{A}_\Gamma\,\vec{X}^{m}
\end{equation}
which is the discrete form of the identity
\begin{equation}
\Delta_s\, \vec \id = \varkappa\, \vec\nu\,.
\end{equation}

\section{Preconditioning}\label{sec:preconditioning}
\setcounter{equation} 0

\textcolor{magenta}{TODO: Why I am calculating twice the factorization of 
(\ref{eq:Xi}) in my code?? Why $S$ is the schur complement operator? Write 
algorithm GMRES to understand better!}

For the solution of (\ref{eq:SchurkX}) we employ a preconditioned GMRES
iterative solver, where for the inverse $\Xi_\Gamma^{-1}$ we employ a sparse
$L\,U$ decomposition, which we obtain with the help of the sparse factorization
package UMFPACK, see \cite{Davis04}. Having obtained $(\vec U^{m+1}, P^{m+1})$
from (\ref{eq:SchurkX}), we solve (\ref{eq:SchurkXb}) for
$(\kappa^{m+1}, \delta\vec X^{m+1})$.

As preconditioner we use the matrix
\begin{equation} \label{eq:ESW}
\mathcal{P} = 
\begin{pmatrix}
\vec{\mathcal{P}}_{\vec B} & \vec C_\Omega \\
0 & -\mathcal{P}_S
\end{pmatrix}\,,
\end{equation}
where $\vec{\mathcal{P}}_{\vec B}$ is some preconditioner for the matrix $\vec
B_\Omega$, and $\mathcal{P}_S$ acts as a preconditioner for the Schur complement
operator $S=\vec C^T_\Omega \,\vec B_\Omega^{-1}\,\vec C_\Omega$.

An application of the preconditioner (\ref{eq:ESW}) amounts to solving the
equations
\begin{equation}
\begin{pmatrix}
\vec{\mathcal{P}}_{\vec B} & \vec C_\Omega \\
0 & -\mathcal{P}_S
\end{pmatrix}
\begin{pmatrix} 
\vec U \\ 
P 
\end{pmatrix}
= 
\begin{pmatrix} 
\vec v \\ 
q 
\end{pmatrix}
\iff
\mathcal{P}_S\,P = -q\,,\quad \vec{\mathcal{P}}_{\vec B}\,\vec U = \vec v -
\vec C_\Omega\,P\,.
\end{equation}

We choose $\vec B_\Omega$ for $\vec{\mathcal{P}}_{\vec B}$ and $M_\Omega$ for 
$\mathcal{P}_S$ where $M_\Omega$ is the pressure mass matrix defined as
\begin{equation}
[M_\Omega]_{ij} = \left(\,\phi_j^{\pspace^m},\, \phi_i^{\pspace^m}\right)\,.
\end{equation}

ADD GMRES PARAMETERS! 

\section{Benchmark solution methods for the preconditioner} 
\label{sec:preconditioner_solver_benchmark}
\setcounter{equation} 0

\begin{table*}
 \center
\begin{tabular}{llll}
\hline
 & full UMFPACK & block UMFPACK & block CG \\
\hline
 16 & 0.0478953 (5) & 0.0640469 (21) & \\
 32 & 0.269122 (6) & 0.325345 (23) & \\
 64 & 1.44521 (7) & 1.7618 (25) & \\
128 & 7.9722 (9) & 10.5198 (29) & \\
\hline
\end{tabular}
\caption{Preconditioner solver benchmark.}
\label{tab:prec_solver_benchmark}
\end{table*}

\newpage
\bibliographystyle{siam}
\bibliography{../bib/robert_refs,../bib/marco_refs}
\end{document}
