\chapter*{Abstract}
\addcontentsline{toc}{chapter}{Abstract}
LAPS is a multi-physics plasma simulation framework and code suite which provides data structures and communication infrastructure on massively parallel computers. It leverages heavily on existing parallel solvers provided by or through \verb|PETSc|. A key module of LAPS is the plasma equilibrium solver for both axisymmetric linear and toroidal configurations, with closed flux surfaces or with boundary intercepting open field lines, using the generalized Grad-Shafranov equation, which is a nonlinear elliptic partial differential equation. By the assumption of axisymmetry, the equilibrium solver concerns only two spatial dimensions.  We have chosen a spectral element method (SEM) with a modal basis and Gaussian integration because it ensures an exponential convergence for smooth solutions, which are expected for plasma equilibrium.

Quadrilateral elements are adopted since it is straightforward to generalize the SEM on the domain $\Omega_1\subset\mathbb{R}$ to the domain $\Omega_2\subset\mathbb{R}^2$ using a basis tensor product. We adopt unstructured instead of commonly used structured mesh for quadrilateral elements for the reason that open field line equilibrium at the tokamak edge requires nontrivial multi-block meshing around the X point. An object oriented generic code is written using \verb|C++|. It reads an unstructured mesh created with \verb|GMSH| and makes an optimal partitioning among the processes. The SEM discretized problem is then solved using the \verb|PETSc| library. Instead of solving directly the system of equations, we rewrite the stiffness matrix with the static condensation technique splitting the contribution of the interior modes and edge modes in order to reduce the communication among the processors to maximize the parallelization of the code and the performance.

In our initial implementation, the equilibrium is solved in the domain delimited by the magnetic field generated by the device. Later it will be solved in the entire physical geometry which leads to a free boundary problem since it is unknown an a priori plasma/vacuum interface. A further development of the code will be the implementation of a common interface which will allow the exchange of the results obtained with different methods of LAPS in order to increase the flexibility and an h-p adaptivity for the code to improve the quality of the solution and the order of convergence of the method.
