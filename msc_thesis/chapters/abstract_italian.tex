\chapter*{Sommario}
\addcontentsline{toc}{chapter}{Sommario}
Il LAPS \`{e} un framework per simulazioni multifisiche di plasmi che offre strutture di dati e infrastrutture di comunicazione per grandi cluster di computers paralleli; per fare ci\`{o} esso utilizza ampiamente i solutori paralleli forniti da o attraverso la libreria \verb|PETSc|. Un modulo fondamentale del LAPS \`{e} il solutore per studiare l'equilibrio del plasma sia in configurazioni toroidali, sia in configurazioni con simmetria assiale, risolvendo l'equazione non lineare alle derivate parziali conosciuta come equazione di Grad-Shafranov generalizzata. Grazie all'ipotesi di simmetria assiale, il solutore di equlibrio opera utilizzando solamente due coordinate spaziali. Il metodo numerico scelto \`{e} un metodo agli elementi spettrali (SEM) con una base modale e con integrazione Gaussiana; tale metodo assicura una convergenza esponenziale quando la soluzione \`{e} regolare, fatto altamente probabile traddandosi di un problema d'equlibrio.

Per la discretizzazione del dominio abbiamo adottato una mesh non strutturata con elementi quadrangolari; questa scelta \`{e} stata dettata dal fatto che \`{e} abbastanza semplice generalizzare un metodo SEM da un dominio $\Omega_1\subset\mathbb{R}$ ad un dominio $\Omega_2\subset\mathbb{R}^2$. L'utilizzo di mesh non strutturate dipende invece dal fatto che in un tokamak le linee aperte d'equilibrio richiedono un complicato meshing multiblocco intorno all'X-point. Abbiamo scritto un codice \verb|C++| generico orientato agli oggetti che legge una mesh non strutturata creata con il software \verb|GMSH| e performa un partizionamento ottimale tra i processi. Il problema discretizzato \`{e} poi risolto utilizzando la libreria \verb|PETSc|; tuttavia, invece di risolvere il sistema lineare direttamente, riscriviamo la matrice di rigidezza usando la static condensation technique che consiste nel separare i contributi dei modi interni da quelli esterni riducendo la comunicazione tra i processi. Ci\`{o} massimizza la parallelizzazione del codice e ne aumenta le performance.

Inizialmente l'equilibrio \`{e} risolto in un dominio delimitato dal campo magnetico generato dalla macchina ma in seguito sar\`{a} risolto in tutta la geometria fisica originando un problema di free boundary dato che non si conosce a priori l'interfaccia plasma/vuoto. Un ulteriore sviluppo del codice sar\`{a} l'implementazione di un'interfaccia comune la quale permetter\`{a} lo scambio di risultati ottenuti dalle differenti parti del LAPS per aumentarne la flessibilit\`{a} ed un adattivit\`{a} h-p per migliorare la qualit\`{a} della soluzione e l'ordine di convergenza del metodo.
